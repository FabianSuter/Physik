\documentclass[8pt, a4paper, fleqn, landscape]{scrartcl}
\usepackage[utf8]{inputenc}
\usepackage[ngerman]{babel}
\usepackage{ulem}

\usepackage{hyperref}
\hypersetup{pdfborder = {0 0 0}}%No borders around Hyperlinks

%Layout
\usepackage{multicol, geometry, xcolor}
\geometry{margin=1cm}
\parindent 0pt
\pagestyle{empty}

\newlength{\breite}
\setlength{\breite}{0.5pt}
\setlength{\columnseprule}{\breite}

\usepackage{graphicx}

\usepackage{tikz}
\usepackage{scalerel} %maybe not needed

\usepackage{lastpage, fancyhdr}



%Mathematik-Pakete
\usepackage{amsmath, amstext, amssymb, mathtools, esint, polynom}
\usepackage{bm}
\allowdisplaybreaks %Seitenumbruch in align-Umgebung erlauben


% Farbe der Boxen für Titel / Untertitel
\definecolor{section}{RGB}{79,148,205}
\definecolor{subsection}{RGB}{135,206,250}
\definecolor{titletext}{RGB}{0,0,0}


% section color box
\setkomafont{section}{\mysection}
\newcommand{\mysection}[1]{%
	\Large\sffamily
	\setlength{\fboxsep}{0cm}%already boxed
	\colorbox{section}{%
		\begin{minipage}{\linewidth}%
			\vspace*{2pt}%Space before
			\leftskip2pt %Space left
			\rightskip\leftskip %Space right
			{\color{titletext} #1}
			\vspace*{1pt}%Space after
		\end{minipage}%
}}
% subsection color box
\setkomafont{subsection}{\mysubsection}
\newcommand{\mysubsection}[1]{%
	\Large\sffamily
	\setlength{\fboxsep}{0cm}%already boxed
	\colorbox{subsection}{%
		\begin{minipage}{\linewidth}%
			\vspace*{2pt}%Space before
			\leftskip2pt %Space left
			\rightskip\leftskip %Space right
			{\color{titletext} #1}
			\vspace*{1pt}%Space after
		\end{minipage}%
}}

%Added from Optik Physik 3
%\setlist[itemize]{itemsep=0pt, parsep=0pt, topsep=0pt} % Funktioniert nicht :(


\newcommand{\m}{\mathrm{m}} 
\newcommand{\Hz}{\mathrm{Hz}}
\newcommand{\s}{\mathrm{s}}
\newcommand{\N}{\mathrm{N}}
\newcommand{\kg}{\mathrm{kg}}
\newcommand{\J}{\mathrm{J}}
\newcommand{\rad}{\mathrm{rad}}				
\newcommand{\A}{\mathrm{A}}
\newcommand{\V}{\mathrm{V}}							
\newcommand{\Pa}{\mathrm{Pa}}	
\newcommand{\mol}{\mathrm{mol}}		
\newcommand{\Ohm}{\mathrm{\Omega}}	
\newcommand{\W}{\mathrm{W}}	
\newcommand{\dB}{\mathrm{dB}}


%Header & footer
\pagestyle{fancy}
\fancyhf{}
\setlength{\footskip}{0.5cm}
\fancyfoot[L]{\thepage{} / \pageref{LastPage}}
\fancyfoot[R]{Physik}
\renewcommand{\footrulewidth}{0pt}
\renewcommand{\headrulewidth}{0pt}

				
\providecommand{\diff}{\mathop{} \! \mathrm{d}}

%new commands 
\newcommand{\const}{\mathrm{const}} 
\newcommand{\e}{\mathrm{e}}


%Informationen für den Befehl maketitle
\title{Physik}
\subtitle{Zusammenfassung an der OST}
\author{Fabian Suter, \today}
\date{{\small \url{https://github.com/FabianSuter/Physik.git}}}
 
\begin{document}

	\begin{multicols*}{3}
		\maketitle
	 
		% TODO: bei Gelegenheit alle Einheitenlisten mit \rule{0pt}{10pt} o.ä. überprüfen (passt die Zeilenhöhe an)

		\thispagestyle{fancy}%Pagenumber for first page

		\input{Sections/01_Statik.tex}

		\section{Kinematik}
			
	\subsection{Geradlinige Bewegung (1D)}
		Die Bewegung erfolgt entlang einer Gerade (keine Richtungsänderung) \\
		\\
		\begin{minipage}{0.48\linewidth}
			\begin{equation*}
				x(t) \quad  \underrightarrow{ \frac{d}{dt}} \quad  v(t) \quad  \underrightarrow{ \frac{d}{dt}} \quad a(t)  
			\end{equation*}
		\end{minipage}
		\hfill
		\begin{minipage}{0.48\linewidth}
			\begin{equation*}
				x(t) \quad \underleftarrow{\int dt} \quad v(t) \quad \underleftarrow{\int dt} \quad a(t)
			\end{equation*}
		\end{minipage}

		\subsubsection{Weg $x(t)$}
			Weg mit Zeit parametrisiert: $x = x(t)$ 

		\subsubsection{Geschwindigkeit $v(t) = \frac{\Delta \, x}{\Delta \, t}$}
		
			\begin{tabular}{lll}
				momentane Geschw.: & $\frac{d}{dt} x(t) = \dot{x}(t)$ & (Tangente) \\	
				\\
				mittlere Geschw.: & $\overline{v} = \frac{x_2 -x_1}{t_2 - t_1} =  \frac{x(t_2) - x(t_1)}{t_2 - t_1} $  & (Sekante) \\
			\end{tabular}

		\subsubsection{Beschleunigung $a(t) = \frac{\Delta \, v}{\Delta \, t}$}
			
			\begin{tabular}{ll}
				momentane Beschleunigung: & $\frac{d}{dt} v(t) = \dot{v}(t) = \ddot{x}(t)$ \\	
				\\
				mittlere Beschleunigung: & $\overline{a} = \frac{v_2 -v_1}{t_2 - t_1} =  \frac{v(t_2) - v(t_1)}{t_2 - t_1} $ \\
			\end{tabular}

		\subsubsection{Ruck $j(t)$}
		Änderung der Beschleunigung pro Zeiteinheit: $j(t) = \dot{a}(t) = \dddot{x}(t)$

	\subsection{Gleichförmige Bewegung $a(t) = 0$}
		\begin{tabular}{l}
			$a(t) = 0$ \\
			\\
			$v(t) = v_0 = \; \text{const}$ \\
			\\
			$x(t) = v_0 \cdot t + x_0 $ \\
		\end{tabular}
		
	\subsection{Gleichm. beschleunigte Bewegung $a(t) = $ konst}
		\begin{minipage}{0.48\linewidth}
			\textbf{Allgemein:} \\
				\\
				\begin{tabular}{l}
					$a(t) = a_0 = \text{const}$ \\
					\\
					$v(t) = a_0 \cdot t + v_0$ \\
					\\
					$x(t) = \frac{1}{2} \, a_0 \cdot t^2 + v_0 \cdot t + x_0$
				\end{tabular}
		\end{minipage}
		\hfill
		\begin{minipage}{0.48\linewidth}
			\textbf{Anwendungsfall: Freier Fall} \\
				\\
				\begin{tabular}{ll}
					$a(t) = -g = \text{const}$ \\
					\\
					$v(t) = -g \cdot t $ \\
					\\
					$x(t) = - \frac{1}{2} \, g \cdot t^2 + h_0$
				\end{tabular}
		\end{minipage}

		\subsubsection{Höchsten Punkt $x_{max}$ finden (Extremum)}
		
			Im Extremalpunkt gilt: $\frac{d}{dt} x(t) = v(t) \overset{!}{=} 0$ \\
			\\
			$0 \overset{!}{=} v(t_{max}) = -g \cdot t_{max} + v_0 $ \qquad \qquad $\Rightarrow t_{max} = \frac{v_0}{g}$ \\
			
			Durch einsetzen von $t_{max}$ in $x(t)$ erhält man die maximale Höhe: \\
			$x(t_{max}) = - \frac{1}{2}\, g \cdot t_{max}^2 + v_0 \cdot t_{max} + h_0 = - \frac{v_0^2}{2 \, g}\, + \frac{v_0^2}{g} + h_0 $

	\subsection{Beliebige Bewegung (2D)}
		
		\subsubsection{Geschwindigkeit (tangential zur Bahnkurve)}
		
			\begin{tabular}{ll}
				momentane Geschw.: & $ \vec{v} = \lim \limits_{\Delta t \rightarrow 0} \frac{\Delta \vec{r}}{\Delta t}  \frac{d}{dt} \vec{r} = \dot{\vec{r}}$ \\	
				\\
				mittlere Geschw.: & $\overline{\vec{v}} = \frac{\Delta \vec{r}}{\Delta t} = \frac{\vec{r}(t + \Delta t) - \vec{r}(t)}{\Delta t} $ \\
				\\
				Betrag: & $v = \vert \vec{v} \vert = \lim \limits_{\Delta t \rightarrow 0} \frac{ \vert \Delta \vec{r} \vert}{\Delta t} = \lim \limits_{\Delta t \rightarrow 0} \frac{\Delta s }{\Delta t} = \frac{d}{dt} s$ \\
			\end{tabular}

		\subsubsection{Beschleunigung}
			\begin{tabular}{ll}
				momentane Beschl.: & $ \vec{a} = \frac{d}{dt} \vec{v} = \dot{\vec{v}} = \frac{d^2}{d t^2} \vec{r} = \ddot{\vec{r}}$ \\	
				\\
				mittlere Beschl.: & $\overline{\vec{a}} = \frac{\Delta \vec{v}}{\Delta t} $ \\
			\end{tabular}
		
			\textbf{Die Beschleunigung kann ungleich null sein, auch wenn der Betrag der Geschwindigkeit konstant ist}

	\subsection{Bahnkurven}	
		Die Geschwindigkeitsänderung in einer Bahnkurve wird in zwei \\
		Komponenten aufgeteilt: \\
			$\Delta \vec{v}_{radial}$ \quad und \quad $\Delta \vec{v}_{tangential}$ \\
			\\
		Der tangentiale Anteil ändert ausschliesslich den Betrag der \\
		Geschwindigkeit $\vert \vec{v} \vert$ \\
		Der radiale Anteil ändert ausschliessich die Richtung der \\
		Geschwindigkeit $\vec{v}$ \\

		\begin{minipage}{0.58\linewidth}
			\begin{tikzpicture}
				[
				x=1cm, y=1cm, scale=0.6, font=\footnotesize, >=latex 
				%Voreinstellung für Pfeilspitzen
				]
				%Raster im Hintergrund
				%\draw[step=1, gray, very thin] (0,0) grid (5.5,5.5);
				\draw[thick] (0, 0) circle(2.5);
				\fill[black] (0, 0) circle(5pt) node [midway, above, yshift=7pt, scale=2] {$m$};
				\begin{scope}[xshift=2.5cm, yshift=0cm, rotate=0, scale=1]
					%Kräfte
					\draw [-latex, very thick, green!50!black] (0,0) -- ++(0,-2) node[midway, right, scale=1.2] {$\vec{a}_{tangential}$};
					\draw [-latex, very thick, purple] (0,0) -- ++(-1.5,0) node[midway, above, scale=1.2, xshift=-3pt] {$\vec{a}_{radial}$};
					\draw [-latex, very thick, orange] (0,0) -- ++(-1.5,-2) node[midway, left, scale=1.2] {$\vec{a}_{tot}$};
					\fill [blue!70!white](0,0) circle (0.1) node[midway, right, scale=2] {$P$};;
				\end{scope}		
			\end{tikzpicture}
		\end{minipage}
		\hfill
		\begin{minipage}{0.4\linewidth}
			$a_{tangential} = \frac{dv}{dt} = \dot{v}$ \\
			\\
			$a_{radial} = \frac{v^2}{r}$ \\
			\\
			$ \boxed { F_{zentripetal} = m \, \frac{v^2}{r} }$ \\
		\end{minipage}

	\subsection{Gleichförmige Bewegung $a_{tangential} = 0$}
	
		\begin{minipage}{0.6\linewidth}
			\textbf{tangential (Tacho)} \\
				\\
				$a_{tangential} = 0$ \\
				\\
				$v(t) = v_0 = \text{const}$ \\
				\\
				$s(t) = v_0 \cdot t + s_0$ \\
		\end{minipage}
		\hfill
		\begin{minipage}{0.35\linewidth}
			\textbf{radial} \\
				\\
				$a_{radial} = \frac{v^2}{r}$ \\
				\\
				\\
				\\
				\\
		\end{minipage}

	\subsection{Gleichm. beschl. Bewegung $a_{tangential} = \text{konst}$}		
		\begin{minipage}{0.6\linewidth}
			\textbf{tangential (Tacho)} \\
				\\
				$a_{tangential} = a_0 = \text{const}$ \\
				\\
				$v(t) = a_{tangential} \cdot t + v_0 $ \\
				\\
				$s(t) = \frac{1}{2} \, a_{tangential} \cdot t^2 +  v_0 \cdot t + s_0$ \\
		\end{minipage}
		\hfill
		\begin{minipage}{0.35\linewidth}
			\textbf{radial} \\
				\\
				$a_{radial} = \frac{v^2}{r}$ \\
				\\
				\\
				\\
				\\
		\end{minipage}

		Die Gesamtbeschleunigung eines Systems $\vec{a}_{tot} = \vec{a}_{tangential} + \vec{a}_{radial} $ muss nicht zwingend konstant sein! Bei Änderungen der Richtung ändert die Gesamtbeschleunigung. 
	
	\subsection{Kreisbewegung}
		
		\subsubsection{Winkel $\phi$ (zurückgelegter Weg)}
		
		\begin{minipage}{0.5\linewidth}
			\begin{tikzpicture}
				[
				x=1cm, y=1cm, scale=1.0, font=\footnotesize, >=latex 
				%Voreinstellung für Pfeilspitzen
				]

				%Länge x Achse
				\draw [thick] (0,0) -- ++(2,0) node[below] {};
				\draw [thick] (1,0) node[below] {$r$};
				
				%Zahlen auf x-Achse
				\foreach \x in {0,2}
				\draw[shift={(\x,0)},color=black, thick] (0pt,2pt) -- (0pt,-2pt);
				
				%Winkelgugus
				\begin{scope}[xshift=0cm, yshift=0.25cm, rotate=0, scale=1]
					\filldraw[fill=white, thick] (0,0) -- (2,0) arc (0:30:2) -- cycle 
						node[midway, below, green!50!black, xshift=0pt, yshift=0pt] {};
					\filldraw[fill=white, thick] (0,0) -- (1,0) arc (0:30:1) -- cycle 
						node[midway, xshift=8pt, yshift=-1pt] {$\phi$};
				\end{scope}
				
				\draw [<->, thick] (2.25,0.25) arc (0:30:2.25) node[midway, right, xshift=0pt, yshift=0pt] {$S$};
			\end{tikzpicture}
		\end{minipage}
		\hfill
		\begin{minipage}{0.4\linewidth}
			$\boxed{ \text{Radiant: } \phi = \frac{s}{r} }$ 
		\end{minipage}

		\subsubsection{Winkelgeschwindigkeit $\omega= \frac{\phi}{t}$}
		
			$\omega := \lim \limits_{\Delta t \rightarrow 0} \frac{\phi(t + \Delta t) - \phi(t)}{\Delta t} = \frac{d \phi}{dt} = \dot{\phi}$ \\
			\\
			Der Betrag $v$ der (Bahn-) Geschwinndigkeit entspricht: $v = r \cdot \omega$ \\

			\begin{tabular}{ll}
				Umlaufzeit, Periode $T$ & Umlaufzeit für vollständige Umdrehung \\
				\\
				Drehzahl, Drehfrequenz $f$ & inverse Umlaufzeit   $f = \frac{1}{T}$ \\
				\\
				\\
			\end{tabular}
			
			\textbf{Wichtige Umrechnungsformeln} \\		 
				\\
				\boxed{
					\begin{tabular}{l c l}
						$v = r \cdot \omega$ & $\Leftrightarrow$ & $\omega = \frac{v}{r}$ \\		
						\\
						$f = \frac{1}{T}$ & $\Leftrightarrow$ & $T = \frac{1}{f}$ \\
						\\
						$\omega = \frac{2 \, \pi}{T}$ & $\Leftrightarrow$ & $T = \frac{2 \, \pi}{f}$ \\
						\\
						$\omega = 2 \, \pi \, f$ & $\Leftrightarrow$ & $f = \frac{\omega}{2 \, \pi}$
					\end{tabular}
				}

		\subsubsection{Winkelbeschleunigung $\alpha = \frac{\omega}{t}$}
		
			$\alpha = \lim \limits_{\Delta t \rightarrow 0} \frac{\omega(t + \Delta t) - \omega(t)}{\Delta t} = \frac{d \omega}{dt} = \dot{\omega} = \frac{d^2 \, \phi}{d t^2} \ddot{\phi} $ \\
			\\
			$a_{tangential} = \frac{dv}{dt} = \frac{d}{dt} r \cdot \omega = r \cdot \alpha$

	\subsection{Gleichförmige Kreisbewegung}
		
		\begin{tabular}{l}
			$\alpha(t) = 0$ \\
			\\
			$\omega(t) = \omega_0 = \text{const}$ \\
			\\
			$\phi(t) = \omega_0 \, t + \phi_0$ 
		\end{tabular}

	\subsection{Gleichm. beschleunigte Kreisbewegung}
		
		\begin{tabular}{l}
			$\alpha(t) = \alpha_0 = \text{const} $ \\
			\\
			$\omega(t) = \alpha_0 \cdot t + \omega_0$ \\
			\\
			$\phi(t) = \frac{1}{2} \, \alpha_0 \cdot t^2 + \omega_0 \cdot t + \phi_0$ \\
		\end{tabular}

	\subsection{Senkrechter Wurf}

		\begin{minipage}{0.45\linewidth}
			\begin{tikzpicture}
				[
				x=1cm, y=1cm, scale=0.5, font=\footnotesize, >=latex 
				%Voreinstellung für Pfeilspitzen
				]
				
				%Raster im Hintergrund
				%\draw[step=1, gray!50!white, very thin] (-2,-2) grid (5,5);
				
				%Zahlen auf x-Achse
				\foreach \x in {-0.75,0.25,1.25,2.25,3.25,4.25}
				\draw[shift={(\x,0)},color=gray, very thick] (0pt,0pt) -- (-10pt,-10pt);
				
				\draw [very thick] (0.5,0)--(0,1);
				\draw [very thick] (-0.5,0)--(0,1);
				\draw [very thick] (0,1)--++(0,1);
				\draw [very thick] (0,1.75)--(0.5,1.55)--(1,1.75);
				\fill (0,2.35) circle (0.35);
				\fill [orange](1,1.75) circle (0.15);

				%Länge x Achse
				\draw [thick] (-1,0) -- ++(5.5,0);
				\draw [-latex, very thick, red] (1,1.75)--++(0,1.5) node [midway, right, red, xshift=0pt, yshift=0pt, scale=1.5] {$\vec{v}_0$} {};
				\draw [-latex, very thick, gray] (4,4)--++(0,-1.5) node [midway, left, gray, xshift=0pt, yshift=0pt, scale=1.5] {$-\vec{g}$} {};
				\draw [-latex, very thick] (-1,0)--++(0,4) node [above, xshift=0pt, yshift=0pt, scale=1.5] {$y$} {};
				\draw [<->, very thick, gray] (2,0)--++(0,1.75) node [midway, right, gray, xshift=0pt, yshift=0pt, scale=1.5] {$h_0$} {};
				\draw [dashed, thick] (1,1.75)--++(1,0);
			\end{tikzpicture}
		\end{minipage}
		\hfill
		\begin{minipage}{0.5\linewidth}
			$a = -g = \text{const}$ \\
			\\
			$v(t) = -g \cdot t + v_0$ \\
			\\
			$h(t) = - \frac{1}{2} \, g \cdot t^2 + v_0 \cdot t + h_0$ \\
		\end{minipage}

		\subsubsection{Maximale Flughöhe $h_{max}$ bestimmen}
			Bei der maximalen Flughöhe $h_{max}$ gilt: $v(t) \overset{!}{=} 0$ \\
			\\
			$v_0 - g \cdot t_{max} \overset{!}{=} 0$ \qquad $\Rightarrow$ \qquad $t_{max} = \frac{v_0}{g}$ \\
			\\
			Nun wird $t_{max}$ in $h(t)$ eingesetzt: \\
			\\
			$h_{max} = h(t_{max}) = - \frac{g}{2} \, \frac{v_0^2}{g^2} + v_0 \, \frac{v_0}{g} + h_0 = \frac{v_0^2}{2 \, g} + h_0 $ \\
			\\
			\\
			\textbf{Hinweis: Die maximale Flughöhe kann auch über die potentielle und kinetische Energie berechnet werden!} \\
				$E_{kin} \overset{!}{=} 0 $ \qquad $E_{pot} \overset{!}{=} m \cdot g \cdot h_{max} $ \\
				$\frac{1}{2} \, m \cdot v^2 = m \cdot g \cdot h_{max}$ \quad $\Rightarrow$ \quad $h_{max} = \frac{m \, v^2}{2 \, m  \, g} =  \frac{v^2}{2 \, g} $  \\
				\\
				$\Rightarrow$ für \underline{abgeschlossene} Systeme!
			
\vfill\null
\columnbreak
	
	\subsection{Horizontaler Wurf}
	
		\begin{tikzpicture}
			[
			x=1cm, y=1cm, scale=0.5, font=\footnotesize, >=latex 
			%Voreinstellung für Pfeilspitzen
			]
			
			%Raster im Hintergrund
			\draw[,step=1, gray!50!white, very thin] (0,0) grid (9.5,7.5);
			
			%Länge x-Achse
			\draw [-latex, very thick] (0,0) -- ++(9.5,0) node[below, scale=1.5] {$x$};
			
			%Länge y-Achse
			\draw [-latex, very thick] (0,0) -- ++(0,7.5) node[left, scale=1.5] {$y$};
			
			%Zahlen auf y-Achse 
			\foreach \y in {0,1,2,3,4,5,6,7}
			\draw[shift={(0,\y)}, color=black, thick] (2pt,0pt) -- (-2pt,0pt);
			
			%Zahlen auf x-Achse
			\foreach \x in {0,1,2,3,4,5,6,7,8,9}
			\draw[shift={(\x,0)},color=black, thick] (0pt,2pt) -- (0pt,-2pt);
			
			\draw [] (0,6) node[gray, left, scale=1.5] {$y_{0}$};	
			\draw [] (8,0) node[orange, below, scale=1.5] {$x_{max}$};	
			\fill [orange] (8,0) circle (0.1);
			% die Parable halt
			\draw[green!50!black, very thick] (0,6) parabola bend (0,6) (8,0);			

			%Vektor v0
			\fill [red] (0,6) circle (0.1);
			\draw[-latex, very thick, red] (0,6) -- ++ (2,0) node [midway, above, red, xshift=7pt, yshift=0pt, scale=1.5] {$\vec{v}_0=\vec{v}_x$} node (v) {};		
			
			%Vektor v1
			\fill [red] (4,4.5) circle (0.1);
			\draw[-latex, very thick, red] (4,4.5) -- ++ (2,0) node [midway, above, red, xshift=0pt, yshift=0pt, scale=1.5] {$\vec{v}_x$} node (v) {};
			\draw[-latex, very thick, red] (4,4.5) -- ++ (0,-1.495) node [midway, left, red, xshift=-4pt, yshift=-1pt, scale=1.5] {$\vec{v}_y$} {};
			\draw[-latex, very thick, blue] (4,4.5) -- ++ (2,-1.495) node [midway, below, blue, xshift=0.9cm, yshift=0pt, scale=1.5] {$\vec{v}_r$} {};
			\draw [dashed, gray, thick] (4,3)--(6,3)--(6,4.5);
			
			\draw [-latex, very thick] (9,7)--(9,5.5) node [midway, left, xshift=0cm, yshift=0pt, scale=1.5] {$-\vec{g}$} {};
		\end{tikzpicture}

		Der horizontale Wurf muss komponentenweise beschrieben werden \\
		x-Achse: gleichförmige, unbeschleunigte Bewegung \\
		y-Achse: gleichmässig beschleunigte Bewegung \\
		
		\begin{minipage}{0.45\linewidth}
			\textbf{x-Achse} \\
				\\
				$a_x = 0$ \\
				$v_x = v_0$ \\
				$x = v_0 \cdot t + x_0$ \\
		\end{minipage}	
		\hfill	
		\begin{minipage}{0.45\linewidth}
			\textbf{y-Achse} \\
				\\
				$a_y = -g$ \\
				$v_y = -g \cdot t$ \\
				$y = - \frac{1}{2} \, g \cdot t^2 + y_0$ \\
		\end{minipage}	
		
		\textbf{Tipp:} Lege den Koordinatenursprung in den Abwurf-Ort
	
		\subsubsection{Beschreibung der Flugbahn (Eliminierung von $t$)}	
			Die y-Koordinate soll als Funktion der x-Koordinate ausgedrückt werden: $y = f(x)$ \\
			\\
			$x = v_0 \, t$ \quad $\Leftrightarrow$ \quad $t = \frac{x}{v_0}$ \quad $\Rightarrow$ \quad $y = - \frac{1}{2} \, g \cdot t^2 = - \frac{g}{2} \frac{x^2}{v_0^2} = y(x)$
			
	\subsection{Schiefer Wurf}
	
		\begin{tikzpicture}
			[
			x=1cm, y=1cm, scale=1.3, font=\footnotesize, >=latex 
			%Voreinstellung für Pfeilspitzen
			]
			
			%Raster im Hintergrund
			\draw[step=1, gray!50!white, very thin] (0,0) grid (4.5,1.5);
			
			%Länge x-Achse
			\draw [-latex] (0,0) -- ++(4.5,0) node[below] {$x$};
			
			%Länge y-Achse
			\draw [-latex] (0,0) -- ++(0,1.5) node[left] {$y$};
			
			%Zahlen auf y-Achse 
			\foreach \y in {0,1}
			\draw[shift={(0,\y)}] (2pt,0pt) -- (-2pt,0pt);
			
			%Zahlen auf x-Achse
			\foreach \x in {0,1,2,3,4}
			\draw[shift={(\x,0)},color=black] (0pt,2pt) -- (0pt,-2pt);
			
			\filldraw[fill=green!20!white, draw=green!50!black] (0,0) -- (0.75,0) arc (0:45:0.75) -- cycle node[midway, below, green!50!black, xshift=4pt, yshift=2pt] {$\phi$};
			
			%gestrichelte linie
			\draw [dashed, orange, thick] (2,0) -- (2,1);
			
			\draw [] (0,1) node[gray, left] {$h_{max}$};
			\draw [] (0,0) node[gray, left] {$h_{0}$};	
			\draw [] (0,0) node[orange, below] {$0$};
			\draw [] (4,0) node[orange, below] {$s_{max}$};	
			\draw [] (2,0) node[orange, below] {$\dfrac{s_{max}}{2}$};	
			
			% die Parable halt
			\draw[red,thick] (0,0) parabola bend (2,1) (4,0);
			
			\draw [dashed, gray, thick] (2,1)-- (0,1);		
			
			%Vektor v
			\draw[-latex, thick, blue] (0,0) -- (0.8,0.8) node [midway, above, blue, xshift=-4pt, yshift=-1pt] {$\vec{v}$} node (v) {};		
			
		\end{tikzpicture}

		Der schiefe Wurf muss komponentenweise beschrieben werden \\
		x-Achse: gleichförmige, unbeschleuigte Bewegung \\
		y-Achse: gleichmässig beschleunigte Bewegung \\
		
		\begin{minipage}{0.45\linewidth}
			\textbf{x-Achse} \\
				\\
				$a_x = 0$ \\
				$v_x = v_0 \cdot \cos(\phi)$ \\
				$x = v_0 \cdot \cos(\phi) \cdot t + x_0$ \\
		\end{minipage}	
		\hfill	
		\begin{minipage}{0.45\linewidth}
			\textbf{y-Achse} \\
				\\
				$a_y = -g$ \\
				$v_y = -g  \cdot t + v_0 \cdot \sin(\phi)$ \\
				$y = - \frac{1}{2} \, g \cdot t^2 + v_0 \cdot \sin(\phi) \cdot t + y_0$ \\
		\end{minipage}	
		
		\textbf{Tipp:} Lege den Koordinatenursprung in den Abwurf-Ort

		\subsubsection{Beschreibung der Flugbahn (Eliminierung von $t$)}	
			Die y-Koordinate soll als Funktion der x-Koordinate ausgedrückt werden: $y = f(x)$ \\
			\\
			$x(t) = v_0 \cdot \cos(\phi) \cdot t$ \quad $\Rightarrow$ \quad $t = \frac{x}{v_0 \cdot \cos(\phi)}$ \\
			\\
			$\Rightarrow$ \quad $y = - \frac{g}{2 \, v_0^2 \cdot \cos^2(\phi)} \cdot x^2 + \tan(\phi) \cdot x = y(x)$

		\subsubsection{Ansätze zur Bestimmung von Extrema}		
			\begin{tabular}{ll}
				max. Wuftweite $s_{max}$ & $y \overset{!}{=} 0 $ \quad ($\phi \in \{ 45° ; 135° \}$)\\
				& $s_{max} = x_{max} \in \{ 0, \frac{2 \, v_0^2}{g} \cos(\phi) \cdot \sin(\phi) \}$ \\
				\\
				Elevationswinkel & $\phi = \frac{1}{2} \arcsin \big( \frac{g \cdot d}{v_0^2} \big) = \frac{1}{2} \arcsin \big( \frac{g \cdot x_{max}}{v_0^2} \big)$ \\
				\\
				max. Wurfhöhe & $v_y \overset{!}{=} 0 $ \\
				& $x_{maxH"ohe} =  h_{max} = \frac{s_{max}}{2} = \frac{x_{max}}{2}$ \\
				& $y(x_{maxH"ohe}) = \frac{v_0^2 \cdot sin^2(\phi)}{2 \, g}$
			\end{tabular}


		\section{Dynamik}
		
	\subsection{Newtonsche Gesetze}
		Gesetze, welche Bewegungen beschreiben.

		\subsubsection{Erstes Newtonsches Gesetz: Trägheitsgesetz}
			Ein Körper verharrt in seine Zustand (Ruhe, gleichförmige geradlinige Bewegung), wenn er nicht durch eine Kraft gezwungen wird, seinen Zustand zu ändern. \\
			\\
			Die \textbf{Trägheit} eines Körpers hängt von seiner (Trägheits-) Masse ab.
			
		\subsubsection{Zweites Newtonsches Gesetz: Aktionsgesetz}	
			\begin{minipage}{0.2\linewidth}
				$\boxed {\vec{F} = m \cdot \vec{a} }$
			\end{minipage}
			\hfill
			\begin{minipage}{0.75\linewidth}
				\begin{tabular}{c l c}
					$\vec{F}$ & Kraft & $[F] = \mathrm{\frac{kg \cdot m}{s^2}  = N}$ \\
					$m$ & (Trägheits-) Masse & $[m] = \mathrm{kg}$ \\	
					$\vec{a}$ &  Beschleunigung & $[a] = \mathrm{\frac{m}{s^2}}$ \\
					\\
				\end{tabular}
			\end{minipage}
				
			$\Rightarrow$ Anwendung erfolgt meist komponentenweise!
				
		\subsubsection{Drittes Newtonsches Gesetz: Wechselwirkungsgesetz}
			
		Wirkt ein Körper A auf einen Körper B mit der Kraft $\vec{F}_{AB}$, so wirkt der Körper B auf A mit der Kraft $\vec{F}_{BA} = - \vec{F}_{AB}$ \\

\vfill\null
\columnbreak	

	\subsection{Reibungskräfte}

		$$ \boxed{ \text{Haftreibung:} \quad  \vec{F}_{R,max} = \mu_H \cdot \vec{F}_N \quad \Rightarrow \text{treibende Kraft} } $$
		
		$$ \boxed{ \text{Gleitreibung:} \quad \vec{F}_{Gleit} \approx \mu_G \cdot \vec{F}_N } $$ 
		
		$$ \boxed{ \text{Rollreibung:} \quad \vec{F}_{Roll} \approx \mu_R \cdot \vec{F}_N \quad \Rightarrow \text{bremsende Kraft} } $$
		
		\begin{tabular}{c l c}
			$\vec{F}_R$ & Reibungskraft & $[\vec{F}_R] = \mathrm{N}$ \\
			$\vec{F}_{R,max}$ & Haftreibungskraft & $[\vec{F}_{R,max}] = \mathrm{N}$ \\
			$\vec{F}_{Gleit}$ & Gleitreibungskraft & $[\vec{F}_{Gleit}] = \mathrm{N}$ \\
		\end{tabular}

	\subsection{Rollreibungslänge $e$ (Drehmoment)}

		\begin{minipage}{0.48\linewidth}
			\includegraphics[width=0.8\linewidth]{Bilder/rollreibung_1} \\
		\end{minipage}
		\hfill
		\begin{minipage}{0.48\linewidth}
			\includegraphics[width=0.8\linewidth]{Bilder/rollreibung_2} \\
		\end{minipage}
		
		$$ \boxed{ e = \frac{r \cdot F}{F_N} = \frac{r \cdot F_R}{F_N} =  \frac{r \cdot \mu_R \cdot F_R}{F_N} = \mu_R \cdot r }$$ 
		
		$$ \boxed{ M_R = e \cdot F_N = \mu_R \cdot r \cdot F_N = r \cdot F_R = r \cdot F} $$ \\

		\begin{tabular}{c l c}
			$e$ & Rollreibungslänge & $[e] = \mathrm{m}$ \\
			$r$ & Radius des Rades & $[r] = \mathrm{m}$ \\
			$F_R$ & Rollreibungskraft & $[F_R] = \mathrm{N}$ \\
			$F_N$ & Normalkraft & $[F_N] = \mathrm{N}$ \\
			$\mu_R$ & Rollreibungskoeffizient & $[\mu_R] = 1$ \\
			$M_R$ & Rollreibungsmoment & $[M_R] = \mathrm{Nm}$ \\
		\end{tabular}

	\subsection{Angetriebenes Rad}
	
		\includegraphics[width=0.8\linewidth]{Bilder/antriebsrad} \\
		\\

		\begin{tabular}{c l c}
			$\vec{F_Z}$ & Zugkraft & $[F_Z] = \mathrm{N}$ \\
			$\vec{F_N}$ & Normalkraft & $[F_N] = \mathrm{N}$ \\
			$\vec{F_R}$ & Rollreibungskraft & $[F_R] = \mathrm{N}$ \\
			$\vec{F_A}$ & Haftreibungskraft & $[F_A] = \mathrm{N}$ \\
		\end{tabular}

		\subsubsection{Hinweise zu Reibung an Rädern}
		
			\begin{tabular}{ll}
				$\bullet$ & Jedes Rad weist Rollreibung auf \\
				$\bullet$ & Zusätzlich zur Rollreibung weist ein angetriebenes Rad \\
				          & eine Haftreibung auf \\
			\end{tabular}

	\subsection{Arbeit und Energie}
	
		\subsubsection{Arbeit}
		
			Wird der Angriffspunkt einer Kraft $\vec{F}$ um die Strecke $d \vec{s}$ verschoben so leistet die Kraft die Arbeit $W$ 
			
			$$ \boxed{ W_{AB} =  \int \limits_A^B d \, W =  \int \limits_A^B \vec{F} \bullet d \, \vec{s} \qquad \text{(Skalarprodukt)} }$$ \\
			
			Wenn die projizierte Kraft konstant ist: $\boxed{ W = F \bullet s_{AB} }$ \\
			\\
			\begin{tabular}{c l c}
				$W$ & Arbeit & $[W] = \mathrm{N \cdot m = J}$ \\
				$F$ & Kraft & $[F] = \mathrm{N}$ \\
				$s$ & Weg & $[s] = \mathrm{m}$ \\
			\end{tabular}

		\subsubsection{Potentielle Energie $W_{pot}$}
			Beim Anheben eines Körpers gewinnt der Körper an potentieller Energie (Lageenergie) 
			
			$$ \boxed{ W_{pot} = m \cdot g \cdot h}$$
			\\
			\begin{tabular}{c l c}
				$W_{pot}$ & Potentielle Energie & $[W] = \mathrm{N \cdot m = J}$ \\
				$m$ & Masse des Körpers & $[m] = \mathrm{kg}$ \\
				$g$ & Erdbeschleunigung & $[g] = \mathrm{\frac{m}{s^2}}$ \\
				$h$ & Höhe der Körpers & $[h] = \mathrm{m}$ \\
				\\
			\end{tabular}
			
			\textbf{Beispiel: Spannen einer Feder} \\
				\\  
				Federkraft als Funktion der Auslenkung x \qquad $F = -k \cdot x$ \\
				\\
				$$ \boxed{ W_{pot} = \int \limits_0^{x_0}  - \vec{F} \bullet d \vec{x} = \int \limits_0^{x_0}  k \cdot x \, dx = \frac{1}{2} \, k \cdot \Delta x^2} $$ \\
				
			\begin{tabular}{c l c}
				$W_{pot}$ & Potentielle Energie & $[W] = \mathrm{N \cdot m = J}$ \\
				$F$ & Federkraft & $[F] = \mathrm{N} $ \\
				$k$ & Federkonstante & $[k] = \mathrm{\frac{N}{m}}$ \\
				$\Delta x$ & Auslenkung der Feder & $[\Delta x] = \mathrm{m}$ \\
			\end{tabular}

			\vfill\null
			\columnbreak

		\subsubsection{Kinetische Energie $W_{kin}$}
		
			$$ \boxed{ \normalsize{ W_{kin} = \int \limits_A^B \vec{F} \bullet d \, \vec{s} =  F \bullet s_{AB} = m \, a \cdot \frac{a}{2} t^2 = m \frac{a^2 \cdot t^2}{2} = \frac{1}{2} m \cdot v^2} }$$
			
			\begin{tabular}{c l c}
				$W_{kin}$ & Kinetische Energie & $[W] = \mathrm{N \cdot m = J}$ \\
				$F$ & Kraft & $[F] = \mathrm{N} $ \\
				$s$ & Wegstück (Kinematik) & $[s] = \mathrm{m}$ \\
				$m$ & Masse des Körpers & $[m] = \mathrm{kg}$ \\
				$a$ & Beschleunigung (Kinematik) & $[a] = \mathrm{\frac{m}{s^2}}$ \\
				$v$ & Geschwindigkeit (Kinematik) & $[v] = \mathrm{\frac{m}{s}}$ \\
			\end{tabular}

	\subsection{Energieerhaltung (in abgeschlossenen Systemen)}
	
		Die Gesamtenergie eines \underline{abgeschlossenen Systems} ist unveränderlich! \\
		\\
		\textbf{abgeschlossen: Es wird keine Masse hinzugefügt/entfernt und es wirken keine äusseren Kräfte!} \\
			\\	
			$$ \boxed{ W = \underbrace{m \cdot g \cdot h}_{\substack{\text{pot. Energie}}} =  m \cdot g \cdot \underbrace{ \frac{1}{2} \, g \cdot t^2}_{\substack{\text{h(t)}}}  = \underbrace{ \frac{1}{2} m \cdot v^2 }_{\substack{\text{kin. Energie}}} } $$ \\
			\\
			Für \underline{nicht abgeschlossene Systeme} kann eine Bilanzrechnung aufgestellt werden: \\
			Die Energiezunahme im Gesamtsystem entspricht der von aussen zugeführten Energie. \\
			Die Energieabnahme im Gesamtsystem entspricht der von aussen entzogenen Energie. \\

		\subsubsection{Energiesatz der Mechanik} 
			$$ \boxed{ E_{pot} + E_{kin} = E_{tot} = \text{const} } \qquad \text{(gilt zu jedem Zeitpunkt)} $$

	\subsection{Leistung und Wirkungsgrad}
	
		\subsubsection{Leistung}
		
			$$ \boxed{ P = \frac{\Delta W}{\Delta t} = \frac{\vec{F} \bullet \Delta \vec{s}}{\Delta t} = \vec{F} \frac{\Delta \vec{s}}{\Delta t} = \vec{F} \bullet \vec{v} } $$ 
			
			
			\begin{tabular}{c l c}
				$P$ & Leistung & $[P] = \mathrm{W = \frac{J}{s}}$ \\
				$\Delta W$ & geleistete Arbeit & $[W] = \mathrm{J}$ \\
				$\Delta t$ & verstrichene Zeit & $[t] = \mathrm{s}$ \\
				$F$ & Kraft & $[F] = \mathrm{N}$ \\
				$\Delta s$ & Wegstück & $[s] = \mathrm{m}$ \\
				\\
			\end{tabular}
			
			\textbf{Pferdestärken} \\
				\\
				$1 \; \mathrm{PS} = 75 \, \mathrm{kg} \cdot 9.81 \mathrm{\frac{m}{s^2}} \cdot 1 \mathrm{\frac{m}{s}} = 735.5 \, \mathrm{W}$ \\

		\subsubsection{Wirkungsgrad $\eta$}	
			Faustregel: Je grösser eine Maschine, desto besser ihr Wirkungsgrad \\
			
			$$ \boxed{ \eta = \frac{P_{ab}}{P_{zu}} } \qquad \textcolor{red}{\eta < 1} \qquad [\eta] = 1 $$

	\subsection{Impuls $\vec{p}$}
	
		$$ \boxed{ \vec{p} = m \cdot \vec{v} }$$ 
		
		2. Newton'sches Gesetz allgemeingültiger (relativistisch): \\
		
		$$ \boxed{ \vec{F} = m \cdot \vec{a} = m \cdot \frac{d \, \vec{v}}{dt} = \frac{d}{dt} (m \cdot \vec{v}) = \frac{d \, \vec{p}}{dt} } $$ 
		
		\begin{tabular}{c l c}
			$\vec{p}$ & Impuls & $[\vec{p}] = \mathrm{\frac{kg  \, m}{s}}$ \\
			$m$       & Masse & $[m] = \mathrm{kg}$ \\
			$\vec{v}$ & Geschwindigkeit & $[v] = \mathrm{\frac{m}{s}}$ \\
			$F$       & Kraft & $[F] = \mathrm{N}$ \\
			$\vec{a}$ & Beschleunigung & $[\vec{a}] = \mathrm{\frac{m}{s^2}}$ \\
		\end{tabular}

		\subsubsection{Kraftstoss $\Delta p$}
			Ein Kraftstoss entspricht einer Impulsänderung und kann über die mittlere Kraft beschrieben werden. \\
			\\
			\begin{minipage}{0.55\linewidth}
				$$ \boxed{ \int \limits_{t_a}^{t_a + \Delta t}  F(t) \, dt = \overline{F} \cdot \Delta t = \Delta p = p' - p } $$ 	
			\end{minipage}
			\hfill
			\begin{minipage}{0.42\linewidth}
				\includegraphics[width=0.8\linewidth]{Bilder/impuls} \\
			\end{minipage}
			
			\begin{tabular}{c l c}
				$F(t)$ & Kraftverlauf & $[F] = \mathrm{N}$ \\
				$\overline{F}$ & mittlere Kraft & $[\overline{F}] = \mathrm{N}$ \\
				$\Delta t$ & Zeitdauer des Kraftstosses & $[\Delta t] = \mathrm{s}$ \\
				$\Delta p$ & Impulsänderung & $[\Delta p] = \mathrm{Ns}$ \\
				$p$ & Impuls vor dem Stoss & $[p] = \mathrm{Ns}$ \\
				$p'$ & Impuls nach dem Stoss & $[p'] = \mathrm{Ns}$ \\
				$\vec{a}$ & Beschleunigung & $[\vec{a}] = \mathrm{\frac{m}{s^2}}$ \\
			\end{tabular}

	\subsection{Impulserhaltungssatz (Impulssatz)}
		In einem \textbf{abgschlossenen System} bleibt der Gesamtimpuls \\
		konstant \\
		abgeschlossenes System: es wirken keine externen Kräfte \\
		
		$$ \boxed{ \vec{p} =  \int \underbrace{  \frac{d \, \vec{p}}{dt} }_{\substack{F_{aussen} = 0}}   \, dt = c = \text{const} }  $$

	\subsection{Stösse}
	
		\begin{tabular}{ll}
			Elastizitätszahl: & $k = \frac{v_2' - v_1'}{v_1 - v_2} = - \frac{v'_{rel}}{v_{rel}} \geq 0$ \\
			\\
			Deformtionsarbeit: & $Q = (E_1 + E_2) - (E_1' + E_2') \geq 0$  \\
			\\
		\end{tabular}

		\begin{tikzpicture}
			[
			x=1cm, y=1cm, scale=0.5, font=\footnotesize, >=latex 
			%Voreinstellung für Pfeilspitzen
			]
			%Raster im Hintergrund
			%\draw[step=1, gray, very thin] (0,0) grid (5.5,5.5);
			
			%m1
			\begin{scope}[xshift=0cm, yshift=0cm, rotate=0, scale=1]
				%Kräfte
				\fill[gray!50!white] (0, 0) circle(1.25);
				\draw[thick] (0, 0) circle(1.25);
				\fill[black] (0, 0) circle(5pt) node [midway, above, yshift=2pt, scale=1.5] {$m_1$};
				\draw [-latex, very thick, purple] (1.25,0) -- ++(1.5,0) node[midway, above, scale=1.5] {$\vec{v}_1$};
			\end{scope}		
			
			%m2
			\begin{scope}[xshift=4cm, yshift=0cm, rotate=0, scale=1]
				%Kräfte
				\fill[gray!50!white] (0, 0) circle(0.75);
				\draw[thick] (0, 0) circle(0.75);
				\fill[black] (0, 0) circle(5pt) node [midway, above, yshift=2pt, scale=1.5] {$m_2$};
				\draw [-latex, very thick, purple] (0.75,0) -- ++(1.5,0) node[midway, above, scale=1.5] {$\vec{v}_2$};
			\end{scope}	
		\end{tikzpicture}

		\subsubsection{Gerader, zentraler, total elastischer Stoss}
			Die beiden Stosspartner verformen sich nicht!\\
			$\Rightarrow$ Für die Deformationsarbeit gilt: $Q = 0$ \\
			\boxed{
				\begin{tabular}{ll}
					Impulssatz: &  $p  \overset{!}{=} p'$ \\
					& $m_1 \, v_1 + m_2 \, v_2 \overset{!}{=} m_1 \, v_1' + m_2 \, v_2'$ \\
					\\
					Energiesatz: & $E_{kin} \overset{!}{=} E_{kin}'$ \\
					& $\frac{1}{2} m_1 \, v_1^2 + \frac{1}{2} m_2 \, v_2^2 \overset{!}{=} \frac{1}{2} m_1 \, v_1'^2 + \frac{1}{2} m_2 \, v_2'^2 $ \\
					\\
					& $v_1' = \frac{m_1 - m_2}{m_1 + m_2} \cdot v_1 + \frac{2 \, m}{m_1 + m_2} \cdot v_2$ \\
					\\
					& $v_2' = \frac{2 \, m_1}{m_1 + m_2} \cdot v_1 + \frac{m_2 - m_1}{m_1 + m_2} \cdot v_2$ \\
				\end{tabular}
			}

		\subsubsection{Gerader, zentraler, total inelastischer Stoss}
			Die beiden Stosspartner haften nach dem Stoss aneinander und haben die gleiche Geschwindigkeit. \\
			$\Rightarrow$ Für die Deformationsarbeit gilt: $Q \neq 0$ \\
			\boxed{
				\begin{tabular}{ll}
					Impulssatz: & $p  \overset{!}{=} p'$ \\
					& $m_1 \, v_1 + m_2 \, v_2 \overset{!}{=} (m_1 + m_2) \, v'$ \\
					\\
					Energiesatz: & $E_{kin} \overset{!}{=} E_{kin}'$ \\
					& $\frac{1}{2} m_1 \, v_1^2 + \frac{1}{2} m_2 \, v_2^2 \overset{!}{=} \frac{1}{2} (m_1 + m_2) \, v'^2 + Q$ \\
					\\
					Deformationsarbeit: & $Q = \frac{m_1 \, m_2}{2 (m_1 + m_2)} (v_1 - v_2)^2 =  \frac{1}{2} \mu \cdot v_{rel}^2 $ \\
					\\
					Relativgeschw.: & $v_{rel} := \vert v_1 - v_2  \vert$ \\
					\\
					Reduzierte Masse: & $\mu = \frac{m_1 \, m_2}{m_1 + m_2}$\\
				\end{tabular}
			}

			\begin{tabular}{c l c}
				$k$ & Elastizitätszahl & $[k] = 1$ \\
				$E_1, \,E_2$ & Energien vor Stoss & $[E] = \mathrm{J}$ \\	
				$E_1 ', \, E_2 '$ & Energien nach Stoss & $[E'] = \mathrm{J}$ \\	
				$m_1, \, m_2$ & stossende Massen & $[m] = \mathrm{kg}$ \\
				$v_1, \, v_2$ & Geschwindigkeit vor Stoss & $[v] = \mathrm{\frac{m}{s}}$ \\
				$v_1', \, v_2'$ & Geschwindigkeit nach Stoss & $[v'] = \mathrm{\frac{m}{s}}$ \\
				$Q$ & Deformationsarbeit & $[Q] = \mathrm{J}$ \\
				$v_{rel}$ & Relativgeschwindigkeit &  $[v_{rel}] = \mathrm{\frac{m}{s}}$ \\
				$\mu$ & reduzierte Masse & $[\mu] = \mathrm{kg}$ \\
			\end{tabular}

	\subsection{Rakete}
		\subsubsection{Rakete im Flug}
			$\Rightarrow$ Masse ist hier veränderbar! \qquad $m(t) = m = m_{Start} - \mu \cdot t$ \\
			\\
			Die Rakete verliert an Treibstoff, wodurch die Masse der Rakete abnimmt ($dm < 0$)\\
			\\	
			\includegraphics[width=0.73\linewidth]{Bilder/rakete} \\
			\\
			Impulssatz: \quad $ m \cdot v(t) = (m + dm)(v(t) + dv) + dm \,(u-v) $ \qquad $dm < 0$ \\
			\\
			Raketengleichung: $v(t) = - u \cdot \ln(m) + v_0 + u \cdot \ln(m_0) = v_0 + u \cdot \ln(\frac{m_0}{m})$ \\
			\\
			Massenverhältnis: $\frac{Startmasse}{Endmasse}$ \\
			\\
			max. Geschwindigkeitsänderung: $\Delta v = v - v_0 = u \cdot \ln(\frac{m_0}{m})$ \\
			\\
			Schubkraft: $F_{Schub} = \frac{dp}{dt} = - \frac{u \cdot dm}{dt} =  \underbrace{ \frac{dm}{dt} }_{\substack{\mu}} (-u) = \mu \cdot u$ \\ 
			\\
			$\Rightarrow$ Hier wurde noch keine Erdbeschleunigung (Anziehung) berücksichtigt! \\
			\\
			\\
			\begin{tabular}{c l c}
				$u$ &  Strahlgeschwindigkeit der Rakete & $[u] =  \mathrm{\frac{m}{s}}$ \\	
				$m$ & Zeitlich veränderbare Masse $m(t)$ & $[m] = \mathrm{kg}$ \\
				$m_0$ & Masse zum Startzeitpunkt & $[m] = \mathrm{kg}$ \\
				$v_0$ & Startgeschwindigkeit & $[v_0] = \mathrm{\frac{m}{s}}$ \\
				$F_{Schub}$ & Schubkraft der Rakete & $[F_{Schub}] = \mathrm{N}$ \\
				$\mu$ & Treibstoffverbrauch pro Zeit & $[\mu] = \mathrm{\frac{kg}{s}}$
			\end{tabular}

		\subsubsection{Aufstieg der Rakete im Schwerefeld}
			Konstante Erdbeschleunigung g wird berücksichtigt \\
			\\
			Veränderbare Masse: $m(t) = m = m_{Start} - \mu \cdot t$ \\
			\\
			Gesamtkraft: $m(t) \frac{dv}{dt} = m(t) \cdot a = F_{Schub} - F_G = \mu \cdot u - m \cdot g$ \\
			\\
			Beschleunigung: $a(t) = \frac{dv}{dt} = \frac{\mu \cdot u}{m_0 - \mu \cdot t} - g$ \\
			\\
			Raketengleichung: $v(t) = u \cdot \ln( \frac{m_{Start}}{m(t)} ) -  g \cdot t  $ \\
			\\
			Spezifischer Impuls: $T = \frac{m(t)}{\mu} = \frac{u}{g}$ \\
			\\
			Steighöhe: $h_t = u \cdot t - \frac{1}{2}gt^2 - \frac{u}{\mu} \cdot ln(\frac{m_0}{m_t}) \cdot m_t $
			\\
			\begin{tabular}{c l c}
				$u$ &  Strahlgeschwindigkeit der Rakete & $[u] =  \mathrm{\frac{m}{s}}$ \\	
				$m$ & Zeitlich veränderbare Masse $m(t)$ & $[m] = \mathrm{kg}$ \\
				$m_0$ & Masse zum Startzeitpunkt & $[m] = \mathrm{kg}$ \\
				$v_0$ & Startgeschwindigkeit & $[v_0] = \mathrm{\frac{m}{s}}$ \\
				$g$ & Erdbeschleuigung & $[g] = \mathrm{\frac{m}{s^2}}$ \\
				$\mu$ & Treibstoffverbrauch pro Zeit & $[\mu] = \mathrm{\frac{kg}{s}}$ \\
				$T$ & spezifischer Impuls (Zeit von konstantem Schub) & $[T] = \mathrm{s}$  \\
			\end{tabular}

	\subsection{Gravitation}
		\subsubsection{Erstes Kepler'sches Gesetz}
			Die Planeten bewegen sich auf Ellipsen, in deren Brennpunkt sich die Sonne befindet. \\
			\\
			\begin{minipage}{0.45\linewidth}
				\includegraphics[width=\linewidth]{Bilder/ellipse}
			\end{minipage}
			\hfill
			\begin{minipage}{0.53\linewidth}
				\begin{tabular}{ll}
					$a$ & grosse Halbachse \\
					$b$ & kleine Halbachse \\
					$F_1, F_2$ & Brennpunkte \\
					$e$ & Exzentrizität \\
					$\epsilon$ & num. Exzentrizität $\epsilon = \frac{e}{a} $\\	
					$r_{min}$ & minimaler Radius \\	
					& $r_{min} = a \,(1 - \epsilon)$ \\
					$r_{max}$ & maximaler Radius \\
					& $r_{max} = a \,(1 + \epsilon)$ \\
				\end{tabular}
			\end{minipage}

		\subsubsection{Zweites Kepler'sches Gesetz}
			Der Fahrstrahl der Planeten überstreicht in der gleichen Zeit die gleiche Fläche. \\
			$\Rightarrow$ Bei kleinerem Abstand zur Sonne ist die Geschwindigeit schneller! \\
		
		\subsubsection{Drittes Kepler'sches Gesetz}
			Die Quadrate der Umlaufzeiten verhalten sich wie die Kuben der grossen Halbachsen. \\
			\\
			$a = \big(  \frac{T}{T_{ref}}^{\frac{2}{3}} \cdot a_{ref} \big)$ \qquad $\Leftrightarrow$ \qquad $\big( \frac{a}{a_{ref}}  \big)^3 =  \big( \frac{T}{T_ref}  \big)^2 $ \\
			\\
			Als Referenz wird die Erde verwendet! \\
			\\
			Astronomische Einheit: $a_{ref} = 1 \, \mathrm{AE} = 149.6 \cdot 10^6 \, \mathrm{km}$ \\
			\\
			Referenzzeit: $T_{ref} = 1 \, a = 1 \; \mathrm{Jahr}$ \\
			\\
			\begin{tabular}{c l c}
				$a$ & grosse Halbachse gesuchtet Planet & $[a] = \mathrm{AE}$ \\
				$a_{ref}$ &  grosse Halbachse Erde & $[a_{ref}] = \mathrm{AE}$ \\	
				$T$ & Umlaufzeit Planet & $[T] = \mathrm{Jahre}$ \\
				$T_{ref}$ & Umlaufzeit Erde & $[T] = \mathrm{Jahre}$ \\
			\end{tabular}

		\subsubsection{Gravitationsgesetz}
		
			$$ \boxed{ \text{Gravitationskraft:}  \quad F_G = G \, \frac{m_1 \cdot m_2}{r^2} \quad \text{mit }G = 6.67 \cdot 10^{-11} \mathrm{\frac{m^3}{kg \, s^2}} }$$ 

		\subsubsection{Gravitationswirkung innerhalb einer Kugel}
		
			$$ \boxed{ F_G = G \, \frac{m_{Kern} (r) \, m}{r^2} =  G \, \frac{4 \, \pi \, r^3 \, \rho \, m}{3 \, r^2} = \frac{4 \, \pi}{3} \, G \, \rho \, m \, r } $$ \\
		
			\begin{tabular}{c l c}
				$F_G$ & Gravitationskraft & $[F_G] = \mathrm{N}$ \\
				$G$ & Gravitationskonstante & $[G] = \mathrm{\frac{m^3}{kg \, s^2}}$ \\	
				$r$ & Radius (Abstand vom Zentrum) & $[r] = \mathrm{m}$ \\
				$\rho$ & homogene Dichte der Kugel & $[\rho] = \mathrm{\frac{kg}{m^3}}$ \\
				$m$ & Masse vom Massepunkt & $[m] = \mathrm{kg}$ \\
				$m_{Kern}$ & Masse des Kugelkerns & $[m_{Kern}] = \mathrm{kg}$ \\
			\end{tabular}

		\subsubsection{Gravitationswirkung ausserhalb einer Kugel}
		
			$$ \boxed{ F_G = G \, \frac{M \cdot m}{r^2}}  $$ \\
			
			\begin{tabular}{c l c}
				$F_G$ & Gravitationskraft & $[F_G] = \mathrm{N}$ \\
				$G$ & Gravitationskonstante & $[G] = \mathrm{\frac{m^3}{kg \, s^2}}$ \\	
				$r$ & Radius (Abstand vom Zentrum) & $[r] = \mathrm{m}$ \\
				$m$ & Masse vom Massepunkt & $[m] = \mathrm{kg}$ \\
				$M$ & Gesamtmasse der Kugel & $[M] = \mathrm{kg}$ \\
			\end{tabular}

		\subsubsection{Gravitationspotential  $\phi$}
		
			Wenn eine Masse in einem Gravitationsfeld bewegt wird, so wird Arbeit verrrichtet. \\
			\\
			$$ \boxed{ W_{12} = \int \limits_{r_1}^{r_2} \vec{F}_G \bullet d \vec{s}  =  \int \limits_{r_1}^{r_2} G \, \frac{M \cdot m}{r^2} \, dr = G \cdot M \cdot m \big( \frac{1}{r_2} - \frac{1}{r_1}  \big) }  $$  
			
			$$ \boxed{ \text{potentielle Energie:} \quad E_{pot}(r) = -G \, \frac{M \, m}{r} }$$ 
			
			$$ \boxed{ \text{Gravitationspotential:} \quad \phi = \frac{E_{pot}}{m} = - \frac{G \cdot M}{r}}  $$ \\

			\textbf{Im Inneren eines homogenen Zentralkörpers gilt} \\
			
				$$ \boxed{ F_G = \frac{4 \, \pi \cdot G \cdot \rho \cdot m \cdot r}{3} }$$ 
				
				$$ \boxed{ E_{pot} = - \frac{2 \,  \pi \cdot G \cdot \rho \cdot m}{3} \, r^2 + c' } $$
				
				$$ \boxed{ \phi = - \frac{2 \,  \pi \cdot G \cdot \rho}{3} \, r^2 + c = - \frac{G \cdot M(r)}{2 \, r}  + c = - \frac{G \cdot M(r)}{2 \, r} - \frac{G \cdot M}{2 \, R} } $$ \\
				
			\begin{tabular}{c l c}
				$W$ & Arbeit & $[W] = \mathrm{J} $ \\
				$F_G$ & Gravitationskraft & $[F_G] = \mathrm{N}$ \\
				$E_{pot}$ & potentielle Energie & $E_{pot} = \mathrm{J}$ \\
				$G$ & Gravitationskonstante & $[G] = \mathrm{\frac{m^3}{kg \, s^2}}$ \\	
				$r$ & Radius (Abstand vom Zentrum) & $[r] = \mathrm{m}$ \\
				$\rho$ & homogene Dichte der Kugel & $[\rho] = \mathrm{\frac{kg}{m^3}}$ \\
				$m$ & Masse vom Massepunnkt & $[m] = \mathrm{kg}$ \\
				$M$ & Gesamtmasse der Kugel & $[M] = \mathrm{kg}$ \\
				$R$ & Radius der Kugeloberfläche & $[R] = \mathrm{m}$ \\
			\end{tabular}

	\subsection{Bezugssysteme: Inertialsystem}
		Inertialsystem: \textbf{unbeschleuigtes} Bezugssystem \\
		\\
		Wenn die Newton'schen Gesetze im Bezugssystem S gelten, so gelten sie auch im Bezugssystem S', solange dieses nicht beschleunigt ist und nicht rotiert. \\
		$\Rightarrow$ \textbf{In sämtlichen Inertialsystemen sind die mechanischen Gesetze identisch!} \\

		\subsubsection{Galilei-Transformation}
			\begin{minipage}{0.48\linewidth}
				Bezugssystem S' bewegt sich mit konstanter Geschwindigkeit $v_0$: \\
				\\
				\\
				$v_0 = \begin{pmatrix}v_x \\ v_y \\ v_z\end{pmatrix}$
			\end{minipage}
			\hfill
			\begin{minipage}{0.42\linewidth}
				Transformation zwischen \\
				S und S' \\
				\\
				$x = x' + v_x t$ \\
				$y = y' + v_y t$ \\
				$z  = z' + v_z t$ \\
				$t = t'$
			\end{minipage}

	\subsection{Beschleunigte Bezugssysteme}
		In beschleunigten Bezugssystemen müssen \textbf{Trägheitskräfte} berücksichtigt werden!

		\subsubsection{Translatorisch beschleunigtes Bezugssystem}
			Beispiel: Zug beschleunigt auf gerader Schiene \\
			\\
			Für einen Beobachter \textbf{im beschleunigten System} S' wirkt \\
			eine Trägheitskraft: 
			
			$$ \boxed{ \text{Gesamtkraft:} \quad \vec{F}' = \vec{F} - m \cdot \vec{a}_0 = \vec{F} + \vec{F}_{Tr"agheit} }$$ \\
			
			\begin{tabular}{c l c}
				$\vec{F}'$ & Gesamte im System wirkende Kraft & $[\vec{F}'] = \mathrm{N}$ \\
				$\vec{F}$ & Statisch wirkende Kräfte & $[\vec{F}] = \mathrm{N}$ \\
				$\vec{F}_{Tr"agheit}$ & Trägheitskraft & $[\vec{F}_{Tr"agheit}] = \mathrm{N}$ \\
				$m$ & Masse im System & $[m] = \mathrm{kg}$ \\
				$\vec{a}_0$ & Beschleunigung des Systems & $[\vec{a}_0] = \mathrm{\frac{m}{s^2}}$ \\
			\end{tabular}

		\subsubsection{Gleichförmig rotierendes Bezugssystem (Scheinkräfte)}

			\textbf{Fest verbundene Masse} $\Rightarrow$ \textbf{Scheinkraft: Zentrifugalkraft} \\
				\\
				\begin{minipage}{0.48\linewidth}
					$$ \boxed{ \vec{F}_z = - m \, \vec{a}_z = - m \cdot \omega^2 \cdot \vec{r} }$$ 
					
					$$ \boxed{ \vec{F}_{Zentrifugal}  \overset{!}{=} - \vec{F}_{Zentripetal} }$$ 
				\end{minipage}
				\hfill
				\begin{minipage}{0.48\linewidth}
					\includegraphics[width=0.75\linewidth]{Bilder/zentrifugalkraft} \\
				\end{minipage}
				\\
				\begin{tabular}{c l c}
					$\vec{F}_z$ & Zentrifugalkraft (Trägheitskraft; Scheinkraft) & $[\vec{F}_z] = \mathrm{N}$ \\
					$m$ & Masse im System & $[m] = \mathrm{kg}$ \\
					$\vec{a}_z$ & Beschleunigung des Systems ($a_{radial}$) & $[\vec{a}_z] = \mathrm{\frac{m}{s^2}}$ \\
					$\omega$ & Winkelgeschwindigkeit & $[\omega] = \mathrm{\frac{rad}{s}}$ \\
					$\vec{r}$ & Radius des Systems (nach innen zeigend) & $[\vec{r}] = \mathrm{m}$ \\ 
					\\
					\\
				\end{tabular}

			\textbf{lose Masse} $\Rightarrow$ \textbf{Scheinkraft: Corioliskraft} \\	
				\\
				\\
				\begin{minipage}{0.48\linewidth}
					$$ \boxed{\vec{F}_c = - m \cdot \vec{a}_c = - m \cdot 2 \, (\vec{\omega} \times \vec{v}_R)} $$ 
				\end{minipage}
				\hfill
				\begin{minipage}{0.48\linewidth}
					\includegraphics[width=0.75\linewidth]{Bilder/corioliskraft} \\
				\end{minipage}
				\\
				\begin{tabular}{c l c}
					$\vec{F}_c$ & Corioliskraft (Trägheitskraft; Scheinkraft) & $[\vec{F}_c] = \mathrm{N}$ \\
					$m$ & Masse im System & $[m] = \mathrm{kg}$ \\
					$\vec{a}_c$ & Coriolisbeschleunigung & $[\vec{a}_c] = \mathrm{\frac{m}{s^2}}$ \\
					$\omega$ & Winkelgeschwindigkeit & $[\omega] = \mathrm{\frac{rad}{s}}$ \\
					$\vec{v}_R$ & Relativgeschwindigkeit & $[\vec{v}_R] = \mathrm{\frac{m}{s}}$ \\ 
				\end{tabular}

		\subsubsection{D'Alembert'sches Prinzip}
			Wird ein Körper in einem mitbewegten Koordinatensystem \\
			betrachtet, so bleibt er in Ruhe: \quad $\vec{v}_R = 0$ und $\vec{a}_R = 0$ \\
			
			$$ \boxed{ \vec{F} + \underbrace{ \vec{F}_z + \vec{F}_c }_{\substack{\text{Scheinkräfte}}} = \vec{0} }$$ 
			
			$\Rightarrow$ Statisches Gleichgewichtsproblem

	\subsection{Rotation starrer Körper}
	
		\begin{tabular}{ll}
			Rotation: & Drehung um feste Achse \\
			Kreisel: & Drehung um starren Punkt \\
			Kreiselbewegung & Drehung eines völlig freien, \\
			&  starren Körpers um seinen Schwerpunkt \\
		\end{tabular}

		\subsubsection{Dynamisches Grundgesetz der Rotation}
			Es ist \textbf{nur die tangentiale Komponente} der Kraft \\
			(des Drehmoments) eines rotierenden Körpers relevant! \\
				
			$$dM_t = r \cdot dF_t = r \cdot dm \cdot a_t = dm \cdot r^2 \cdot \alpha$$ \\
			
			$$ \boxed{ M = \int dM = \int r^2 \, \alpha \cdot dm = \alpha  \underbrace{  \int r^2 \cdot dm }_{\substack{J_{Scheibe} = m \cdot r^2}} }$$ \
			
			$$ \boxed{ \Rightarrow \; M = J \cdot \alpha = r \cdot F} $$ \\
				
			\begin{tabular}{c l c}
				$dM_t$ & kleine Tan.-Komponente des Drehmoments & $[dM_t] = \mathrm{Nm}$ \\
				$M$ & (gesamtes) Drehmoment & $[M] = \mathrm{Nm}$ \\
				$dF_t$ & kleine Tangentialkomponente der Kraft & $[dF_t] = \mathrm{N}$ \\
				$r$ & Abstand Drehachse zu Massepunkt (Rand) & $[r] = \mathrm{m}$ \\
				$dm$ & kleines Massestück des Körpers & $dm = \mathrm{kg}$ \\
				$a_t$ & Tangentialbeschleunigung ($a_t = r \cdot \alpha$) & $[a_t] = \mathrm{\frac{m}{s^2}}$ \\
				$\alpha$ & Winkelbescheunigung & $[\alpha] = \mathrm{\frac{rad}{s^2}}$ \\
				$J$ & (Massen-) Trägheitsmoment & $[J] = \mathrm{kg \, m^2}$ \\
			\end{tabular}

		\subsubsection{Massenträgheitsmomente} %TODO: Irgendwann mal Bilder gegen ganze Tabelle tauschen
		
			\includegraphics[width=0.8\linewidth]{Bilder/massentraegheitsmomente}

			Ring: $m \cdot r^2$

	\subsection{Trägheitsellipsoid}
		Trägheitsradius $r_0$: als ob ganze Masse eines Körpers nur einen \\
		Radius hätte \\
		\\
		\begin{minipage}{0.48\linewidth}
			$$ \boxed{ r_0 = \sqrt{\frac{J}{m}}} $$
		\end{minipage}
		\hfill
		\begin{minipage}{0.48\linewidth}
			$$ \boxed{s_0 = \frac{1}{r_0} }$$
		\end{minipage}
		
		\begin{tabular}{c l c}
			$r_0$ & Trägheitsradius & $[r_0] = \mathrm{m}$ \\
			$m$ & Masse des Körpers & $[m] = \mathrm{kg}$ \\
			$J$ & (Massen-) Trägheitsmoment & $[J] = \mathrm{kg \, m^2}$ \\
			$s_0$ & reziproker Trägheitsradius & $[s_0] = \mathrm{m}$ \\
			\\
		\end{tabular}
		
		\includegraphics[width=0.8\linewidth]{Bilder/traegheits_ellipsoid} \\	
		\\
		\textcolor{red}{Hauptträgheits-Achsen} (entsprechen immer Symmetrie-Achsen, falls vorhanden) \\
		\textcolor{green}{beliebige Achse $J_A$} \quad $J_A = J_x \cdot \cos^2(\alpha) + J_y \cdot \cos^2(\beta) + J_z \cdot \cos^2(\gamma)$

	\subsection{Satz von Steiner}
		Beschreibt, wie man das Trägheitsmoment $J$ berechnet, wenn die Drehachse nicht durch den Schwerpunkt des rotierenden Körpers geht, sonden \textbf{parallel} dazu verläuft. \\
		
		\includegraphics[width=0.6\linewidth]{Bilder/steiner} \\	
		\\
		\begin{tabular}{c l c}
			$J_S$ & Trägheitsmoment (Rot. um Schwerp.)  & $[J_S] = \mathrm{kg \, m^2}$ \\
			$J_A$ & Trägheitsmoment (Rot. um bel. Punkt)  & $[J_A] = \mathrm{kg \, m^2}$ \\
			$m$ & Masse des Körpers & $[m] = \mathrm{kg}$ \\
			$d$ & Abstand zum Schwerpunkt & $[d] = \mathrm{m}$ \\
		\end{tabular}

	\subsection{Arbeit und Leistung (Rotation)}
		$dW = \vec{F} \bullet d\vec{s} = F_t \cdot ds = F_t \cdot r \cdot d \phi = M \cdot d \phi $ \\
		\\
		$P = \frac{dW}{dt} = M \frac{d \phi}{dt} = M \cdot \omega$ \\
		\\
		\begin{tabular}{c l c}
			$F_t$ & \textbf{Tantentialer} Kraftanteil der Rotation & $[F_t] = N$ \\
			$d \phi$ & zurückgelegter Kreiswinkel & $[d \phi] = rad$ \\	
			$P$ & Leistung  & $[P] = W$ \\
			$W$ & Energie  & $[W] = J$ \\
			$\omega$ & Winkelgeschwindigkeit & $[\omega] = \frac{rad}{s}$ \\
			$M$ & Drehmoment & $[M] = Nm$ \\
		\end{tabular}

	\subsection{Rotationsenergie}
		\textbf{Folgendes gilt nur für die Rotation um den Schwerpunkt eines Körpers!} \\
		
		Die totale kinetische Energie ist die Summe aller kinetischer Energien eines Körpers \\
		
		$$ \boxed{ E_{kin} = \int \frac{1}{2} \, v^2 \, dm  = E_{trans} + E_{rot} } $$ 
		
		\begin{minipage}{0.48\linewidth}
			$$ \boxed{ E_{trans} = \frac{1}{2} \, m \cdot v_s^2 } $$ 
		\end{minipage}
		\hfill
		\begin{minipage}{0.48\linewidth}
			$$ \boxed{ E_{rot} = \frac{1}{2} \, J_s \cdot \omega^2 } $$ 
		\end{minipage}

		\begin{tabular}{c l c}
			$E_{trans}$ & Translationsenergie des Schwerpunkts & $[E_{trans}] = \mathrm{J}$ \\
			$m$ & Masse des Körpers & $[m] = \mathrm{kg}$ \\
			$v_s$ & Geschwindigkeit des Schwerpunkts & $[v_s] = \mathrm{\frac{m}{s}}$ \\
			$E_{rot}$ & Rotationsenergie & $[E_{rot}] = \mathrm{J}$ \\
			$J_S$ & Trägheitsmoment (Rot. um Schwerp.)  & $[J_S] = \mathrm{kg \, m^2}$ \\
			$\omega$ & Winkelgeschwindigkeit & $[\omega] = \mathrm{\frac{rad}{s}}$ \\
		\end{tabular}
	 
	\subsection{Drehimpuls $\vec{L}$ / Impulserhaltung (Rotation)}
		\begin{minipage}{0.42\linewidth}
			\includegraphics[width=\linewidth]{Bilder/drehimpuls} \\
			\\
		\end{minipage}
		\hfill
		\begin{minipage}{0.56\linewidth}
			$$ \boxed{ \vec{L} = \int d \vec{L} = \int \vec{r} \times \vec{v} \cdot dm = \vec{r} \times \vec{p} }$$ \\
		\end{minipage}

		\begin{tabular}{c l c}
			$\vec{L}$ & Drehimpuls & $[\vec{L}] = \mathrm{\frac{kg \, m^2}{s}}$ \\
			$\vec{r}$ & Abstand Massepunkt zu Rot-Achse & $[\vec{r}] = \mathrm{m}$ \\
			$\vec{v}$ & Rotationsgeschwindigkeit & $[\vec{v}] = \mathrm{\frac{m}{s}}$ \\
			$dm$ & kleines Masse-Stück & $[dm] = \mathrm{kg}$ \\
			$\vec{p}$ & Impuls & $[\vec{p}] = \mathrm{\frac{kg \, m}{s}}$ \\
		\end{tabular}

		\subsubsection{Energie beim Runterrollen}

			$$ \boxed{E_{pot} = E_{kin} + E_{rot} , \quad \quad m \cdot g \cdot h = \frac{1}{2}m \cdot v^2 + \frac{1}{2}J \cdot \omega^2} $$

		\subsubsection{Drehmoment $\vec{M}$ vs. Drehimpuls $\vec{L}$}
			$$ \boxed{ \vec{M} = \vec{r} \times \vec{F} = \frac{d}{dt} (\vec{r} \times \vec{p}) =  \frac{d}{dt} \vec{L} = \dot{\vec{L}} }$$ \\
			
			\textbf{In einem abgschlossenen System ($\vec{M} = 0$) bleibt der \\
			Gesamtdrehimpuls erhalten} \\
			$\Rightarrow \vec{L} = \text{const}$ \\
			\\
			\boxed{
				\begin{tabular}{ll}
					Impulserhaltung: &  $L  \overset{!}{=} L'$ \\
					& $J_1 \cdot \omega + J_2 \cdot \omega \overset{!}{=} J_1 \cdot \omega_1' + J_2 \cdot \omega_2'$ \\
					\\
					Energiesatz: & $E_{rot} \overset{!}{=} E_{rot}' + Q$ \\
					& $\frac{1}{2} J_1 \cdot \omega_1^2 + \frac{1}{2} J_2 \cdot \omega_2^2 \overset{!}{=} \frac{1}{2} J_1 \cdot \omega_1'^2 + \frac{1}{2} J_2 \cdot \omega_2'^2 + Q $ \\
				\end{tabular}
			}
			\\
			\\
			\begin{tabular}{c l c}
				$\vec{M}$ & Drehmoment & $[\vec{M}] = \mathrm{Nm}$ \\
				$\vec{r}$ & Abstand Massepunkt zu Rot-Achse & $[\vec{r}] = \mathrm{m}$ \\
				$\vec{F}$ & Kraft, welche Drehmoment bewirkt & $[\vec{F}] = \mathrm{N}$ \\
				$\vec{p}$ & Impuls & $[\vec{p}] = \mathrm{\frac{kg \, m}{s}}$ \\
				$\vec{L}$ & Drehimpuls & $[\vec{L}] = \mathrm{\frac{kg \, m^2}{s}}$ \\
				$J$  & Massenträgheitsmoment & $[J] = \mathrm{kg \, m^2}$ \\
				$\omega$ & Winkelgeschwindigkeit & $[\omega] = \mathrm{\frac{1}{s}}$ \\
				$Q$ & Deformationsarbeit & $[Q] = \mathrm{J}$ \\
			\end{tabular}

		\subsubsection{Drehimpuls $\vec{L}$ vs. Winkelgeschwindigkeit $\omega$}

		$$ \boxed{ L = \int dL = \int r^2 \, \omega \, dm = \omega \int r^2 \, dm = J \, \omega }$$ \\
		
			\begin{tabular}{c l c}
				$L$ & Drehimpuls & $[L] = \mathrm{\frac{kg \, m^2}{s}}$ \\
				$r$ & Abstand Massepunkt zu Rot-Achse & $[r] = \mathrm{m}$ \\
				$dm$ & kleines Masse-Stück & $[dm] = \mathrm{kg}$ \\
				$\omega$ & Winkelgeschwindigkeit & $[\omega] = \mathrm{\frac{rad}{s}}$ \\
				$J$ & (Massen-) Trägheitsmoment (hier Tensor) & $[J] = \mathrm{kg \, m^2}$ \\
			\end{tabular}

	\subsection{Rotation vs. Translation}
	
		\includegraphics[width=0.85\linewidth]{Bilder/rotation_translation}
		

		\section{Hydrostatik}

\subsection{Festkörper, Flüssigkeit, Gas}

\subsubsection{Festkörper}

\begin{tabular}{ll}
	$\bullet$ & kein Fluid \\
	$\bullet$ & festes Volumen; feste Gestalt \\
	$\bullet$ & Moleküle / Atome befinden sich in regelmässiger \\
			  & Gitter-Anordnung \\
	$\bullet$ & inkompressibel (sehr schlecht komprimierbar) \\
	$\bullet$ & Kraft: Weiterleitung (längs ihrer Wirkungslinie) \\
	$\bullet$ & Druck: Verstärkung \\
\end{tabular}

\subsubsection{ideale Flüssigkeit}

\begin{tabular}{ll}
	$\bullet$ & Fluid \\
	$\bullet$ & festes Volumen; keine feste Gestalt \\
	$\bullet$ & Moleküle / Atome bewegen sich chaotisch aneinander vorbei \\
	$\bullet$ & Moleküle / Atome füllen den Raum aus / berühren sich \\
	$\bullet$ & inkompressibel (schlecht komprimierbar) \\
	$\bullet$ & reibungsfrei (keine Scherkräfte)\\
	$\bullet$ & Kraft: Verstärkung \\
	$\bullet$ & Druck: Weiterleitung (gleichmässig) \\
\end{tabular}



\subsubsection{Gas}

\begin{tabular}{ll}
	$\bullet$ & Fluid \\
	$\bullet$ & kein festes Volumen; keine feste Gestalt \\
	$\bullet$ & Moleküle / Atome fliegen mit hoher Geschwindigkeit durch\\
			  & den Raum \\
	$\bullet$ & Es gibt sehr viel Zwischenraum \\
	$\bullet$ &  Moleküle / Atome führen bei Zusammenstoss unter sich oder\\
			  & mit Gefässwand elestische Stösse aus \\
	$\bullet$ & kompressibel (gut komprimierbar)  \\
	$\bullet$ & reibungsfrei (keine Scherkräfte)\\

\end{tabular}


\vfill\null
\columnbreak


\subsection{Druck $p$ / Schubspannung $\tau$}
\textbf{Druck ist eine skalare Grösse (hat keine Richtung)} 


$$\boxed{ p = \frac{F_{\perp}}{A} } \qquad \boxed{ \tau = \frac{F_{\parallel}}{A} } $$

	\begin{tabular}{c l c}
		$p$ & Druck & $[p] = \mathrm{Pa = \frac{N}{m^2}}$ \\
		$\tau$ & Schubspannung (Scherkraft) & $[\tau] = \mathrm{N} $ \\
		$F_{\perp}$ & Kraft senkrecht zu A & $[F_{\perp}] = \mathrm{N}$ \\
		$F_{\parallel}$ & Kraft parallel zu A & $[F_{\parallel}] = \mathrm{N}$ \\
		$A$ & Fläche & $[A] = \mathrm{m^2}$ \\
		\\
	\end{tabular}
	
	\textbf{In abgeschlossenen, miteinander verbundenen Systemen herrscht ein Druck-Gleichgewicht!} 
	
	$$ \boxed{ p_1 = p_2  \qquad \Rightarrow \frac{F_1}{A_1} = \frac{F_2}{A_2} }$$
	
	
	
	\subsubsection{Weitere Einheiten von Druck}
	\textbf{1 bar = $10^5$ Pa} \qquad (Absulutdruck: Vergleich zu Vakuum)\\
	$ 1 \, \mathrm{hPa} = 100 \, \mathrm{Pa} = 1 \, \mathrm{mbar}$ \\
	$1 \, \mathrm{at} = 1 \, \mathrm{kp \cdot cm^{-2}} = 9.81 \cdot 10^4 \, \mathrm{Pa}$  \\
	$1 \, \mathrm{at"u} = 1 \, \mathrm{at}$ ("Uberdruck; Vergleich zu normalem Luftdruck) \\
	$1 \, \mathrm{Torr} = \frac{1}{760} \mathrm{at}$ (1mm-Hg-Säule) \\
	$1 \, \mathrm{psi} = 6894.76 \, \mathrm{Pa}$ (Britisch) \\


	
	


\subsection{Kompression}
	
	
	$$ \boxed{ \text{Flüssigkeiten:} \qquad \Delta p = \frac{1}{\kappa} \cdot - \frac{\Delta V}{V} = K \cdot - \frac{\Delta V}{V} } $$  
	
	$$ \boxed{  \text{Gase:} \qquad \Delta p = p(h) - p_0 = \frac{1}{\kappa_T} \cdot - \frac{\Delta V}{V} } $$ \\
	
	
	
	\begin{tabular}{c l c}
		$\Delta p$ & Druckerhöhung & $[\Delta p] = \mathrm{Pa = \frac{N}{m^2}}$ \\
		$\kappa$ & Kompressibilität (Flüssigkeit) & $[\kappa] = \mathrm{\frac{1}{Pa}}$ \\
		$K = \frac{1}{\kappa}$ & Kompressionsmodul & $[K] = \mathrm{Pa}$ \\
		$\kappa_T$ & Kompressibilität (Gas) & $[\kappa_T] = \mathrm{\frac{1}{Pa}}$ \\
		$- \frac{\Delta V}{V}$ & realtive Volumen-Abnahme & $[\frac{\Delta V}{V}] = 1$ 
	\end{tabular}
	
	
	
\subsection{Dichte $\rho$}

$$ \boxed{ \rho = \frac{m}{V} } \qquad \Leftrightarrow \qquad \boxed{ m = \rho \cdot V }$$	 


	\begin{tabular}{c l c}
		$\rho$ & Dichte & $[\rho] = \mathrm{\frac{kg}{m^3}}$ \\
		$m$ & Masse & $[m] = \mathrm{kg} $ \\
		$V$ & Volumen & $[V] = \mathrm{m^3}$ \\
	\end{tabular}
	
	
\subsubsection{Wichtige Dichten}	
	
	\begin{tabular}{l}
		$\rho_{Wasser} = 1000 \, \mathrm{\frac{kg}{m^3}}  $ \\
		$\rho_{Luft} = 1.2 \, \mathrm{\frac{kg}{m^3}}  $ \\
	\end{tabular}
	
	
	
\vfill\null
\columnbreak
	
\subsection{Boyle-Mariotte}	
\textbf{Das Gesetz von Boyle-Mariotte beschreibt die \\
Kompressibilität von Gasen.} \\
\textbf{$\Rightarrow$ Das Gesetz gilt nur bei konstanter Temperatur!} \\

$$ \boxed{ p_1 \cdot V_1 = p_2 \cdot V_2 = \, \const } \qquad  \Rightarrow  \boxed{ \frac{p_1}{p_2} = \frac{\rho_1}{\rho_2} }$$ 


	\begin{tabular}{c l c}
		\rule{0pt}{8pt}$\rho_x$ & Gas-Dichte & $[\rho_x] = \mathrm{\frac{kg}{m^3}}$ \\
		$p_x$ & Gas-Druck & $[p_x] = \mathrm{Pa} $ \\
		$V_x$ & Volumen & $[V_x] = \mathrm{m^3}$ \\
	\end{tabular}
	
	
	
\subsection{Hydrostatischer Druck (Schweredruck)}
\textbf{Fluid inkompressibel!} 

$$ \boxed{ p = \rho \cdot g \cdot h }$$	


	\begin{tabular}{c l c}
		\rule{0pt}{8pt}$\rho$ & Dichte der Flüssigkeit & $[\rho] = \mathrm{\frac{kg}{m^3}}$ \\
		\rule{0pt}{8pt}$g$ & Erdbeschleunigung $g = 9.81 \mathrm{\frac{m}{s^2}}$ & $[g] = \mathrm{\frac{m}{s^2}}$ \\
		$h$ & Höhe \textbf{unter} der Flüssigkeits-Oberfläche & $[h] = \mathrm{m}$ \\
		\\
	\end{tabular}
	
	\textbf{Der Druck ist nur von der Höhe der darüberliegenden Flüssigkeit abhängig, nicht von deren Volumen oder \\
	Gewicht.}
	
	

\subsection{Barometrische Höhenformel (Gase)}
\textbf{Fluid kompressibel!} 

$$ \boxed{ p(h) = p_0 \cdot e^ {- \frac{\rho_0}{p_0} \cdot g \cdot h} }$$	


	\begin{tabular}{c l c}
	$p(h)$ & Schweredruck des Gases bei Höhe $h$ & $[p(h)] = \mathrm{Pa}$ \\
	$p_0$ & Luftdruck auf Meereshöhe $p_0 = 10^5 \, \mathrm{Pa}$ & $[p_0] = \mathrm{Pa}$ \\ 
		\rule{0pt}{8pt}$\rho_0$ & Luft-Dichte auf Meereshöhe $\rho_0 = 1.2 \mathrm{\frac{kg}{m^3}}$ & $[\rho_0] = \mathrm{\frac{kg}{m^3}}$ \\
		\rule{0pt}{8pt}$g$ & Erdbeschleunigung $g = 9.81 \mathrm{\frac{m}{s^2}}$ & $[g] = \mathrm{\frac{m}{s^2}}$ \\
		$h$ & Höhe über Meer & $[h] = \mathrm{m}$ \\
	\end{tabular}
	
\subsection{Statischer Auftrieb (Fluid)}
Der Auftrieb eines Körpers entspricht dem Gewicht der von ihm \\
verdrängten Flüssigkeit (Archimedes). 


\begin{minipage}{0.6\linewidth}
$ \boxed{ F_A = \rho_{Fl} \cdot V_K \cdot  g} $ \\
\\
$\boxed{ F_A = F_{G,Fl} = m_{Fl} \cdot g = \rho_{Fl} \cdot V_K \cdot g } $ \\

\end{minipage}
\hfill
\begin{minipage}{0.35\linewidth}
\includegraphics[width=0.75\linewidth]{Bilder/auftrieb} \\
\end{minipage}




	\begin{tabular}{c l c}
		$F_A$ & Auftriebskraft & $[F_A] = \mathrm{N}$ \\
		\rule{0pt}{8pt}$\rho_{Fl}$ & Dichte \textbf{verdrängtes Fluid} & $[\rho_{Fl}] = \mathrm{\frac{kg}{m^3}}$ \\
		$V_K$ & verdrängtes Fluid-Volumen & $[V_K] = \mathrm{m^3}$  \\
		\rule{0pt}{8pt}$g$ & Erdbeschleunigung $g = 9.81 \mathrm{\frac{m}{s^2}}$ & $[g] = \mathrm{\frac{m}{s^2}}$ \\
		$m_{Fl}$ & Masse des \textbf{verdrängten Fluids} & $[m_{Fl}] = \mathrm{kg}$ \\
		$F_{G,Fl}$ & Gewichtskraft \textbf{verdrängtes Fluid} & $[F_{G,Fl}] = \mathrm{N}$ \\
	\end{tabular}
	


\subsection{Oberflächenspannung $\sigma$}

$$ \boxed{ \sigma := \frac{F}{l} } $$ 


	\begin{tabular}{c l c}
		\rule{0pt}{10pt}$\sigma$ & Oberflächenspannung & $[\sigma] = \mathrm{\frac{N}{m}} = \mathrm{\frac{J}{m^2}}$ \\
		$F$ & Kraft & $[F] = \mathrm{N} $ \\
		$l$ & Länge & $[l] = \mathrm{m}$  \\
		\\
	\end{tabular}
	
	\textbf{Die Länge $l$ entspricht der gesamten Berührungslänge  \\
	zwischen Flüssigkeit und Festkörper / Gas} \\
	
	\begin{tabular}{ll}
	Zylinder & $l = 2 \, \pi \, r$ \\
	Lamellen & $l = 2 \, b$  (beidseitig!) \\
	\end{tabular}


\subsection{Grenzflächenspannung}

$$ \boxed{ \sigma_{sl} + \sigma_{lg} \cdot cos \varphi = \sigma_{sg }} $$

\begin{minipage}{0.48\linewidth}
	\includegraphics[width=\linewidth]{Bilder/benetzung.png} \\
	$ \varphi < 90 $ %TODO: Winkel einfügen
\end{minipage}
\hfill
\begin{minipage}{0.48\linewidth}
	\includegraphics[width=\linewidth]{Bilder/nichtbenetzung.png} \\
	$ \varphi > 90° $
\end{minipage}





\subsection{Kapillarität $h$}

$$\boxed{  h = \frac{2 \cdot \sigma}{\rho \cdot g \cdot r} = \frac{\sigma}{\rho \cdot g \cdot d} }$$ 


	\begin{tabular}{c l c}
		\rule{0pt}{10pt}$\sigma$ & Totale Grenzflächenspannung & $[\sigma] = \mathrm{\frac{N}{m}}$ \\
		\rule{0pt}{10pt}$\rho$ & Dichte der Flüssigkeit & $[\rho] = \mathrm{\frac{kg}{m^3}} $ \\
		\rule{0pt}{10pt}$r$ & Radius der Kapillare & $[r] = \mathrm{m}$  \\
		$d$ & Durchmesser der Kapillare & $[r] = \mathrm{m}$  \\
		\\
	\end{tabular}

\begin{minipage}{0.48\linewidth}
\includegraphics[width=0.3\linewidth]{Bilder/kapillaritaet_benetzend} \\

benetzend
\end{minipage}
\hfill
\begin{minipage}{0.48\linewidth}
\includegraphics[width=0.3\linewidth]{Bilder/kapillaritaet_nicht_benetzend} \\

nicht benetzend

\end{minipage}




\subsection{Druck in Seifenblase $p$}

$$ \boxed{ p = \frac{2 \cdot \sigma}{r} } $$ 


	\begin{tabular}{c l c}
		\rule{0pt}{8pt}$\sigma$ & Oberflächenspannung & $[\sigma] = \mathrm{\frac{N}{m}}$ \\
		$r$ & Radius der Seifenblase & $[r] = \mathrm{m}$  \\
	\end{tabular}



\vfill\null
\columnbreak




\section{Hydrodynamik - Ideale Fluide}

\textbf{Ideale Fluide nehmen keine Scherkräfte auf (keine Reibung) und sind inkompressibel.}

\subsection{Stromlinien-Modell}

\begin{tabular}{ll}
$\bullet$ & Stromlinien zeigen Geschwindigkeit des Fluids \\
$\bullet$ & \textbf{Dichte} Stromlinien bedeutet \textbf{hohe} Geschwindigkeit \\
$\bullet$ & \textbf{Dünne} Stromlinien bedeutet \textbf{niedrige} Geschwindigkeit \\
$\bullet$ & Stationär: Stromlinien = Bahnlinien $\Rightarrow$ schneiden sich nicht 
\end{tabular}




\subsection{Kontinuitätsgleichung}


\includegraphics[width=0.7\linewidth]{Bilder/Kontinuitaet.png}


$$ \boxed{ \frac{\Delta V}{\Delta t} = \dot{V} = A \cdot v = \const } \quad \Leftrightarrow \quad \boxed{  A_1 \cdot v_1 = A_2 \cdot v_2 = \frac{\Delta V}{\Delta t} = \dot{V}} $$ 

\begin{tabular}{c l c}
		$\Delta V$ & Volumenänderung & $[\Delta V] = \mathrm{m^3}$ \\
		$\Delta t$ & Zeitänderung & $[\Delta t] = \mathrm{s}$  \\
		\rule{0pt}{8pt}$\dot{V}$ & Volumenstrom (Volumen pro Zeit) & $[\dot{V}] = \mathrm{\frac{m^3}{s}}$ \\
		$A_x$ & Querschnittsfläche & $[A_x] = \mathrm{m^2}$ \\
		\rule{0pt}{8pt}$v_x$ & Geschwindigkeit der Flüssigkeit & $[v_x] = \mathrm{\frac{m}{s}}$ \\
		\\
	\end{tabular}

$\Rightarrow$ Gilt auch für Gase, wenn $v << v_{Schall}$

\vfill\null
\columnbreak


\subsection{Bernoulli-Gleichung}

\begin{minipage}{0.38\linewidth}
Die Bernoulli-Gleichung beschreibt ein \\
\underline{bewegtes} Fluid \\
\end{minipage}
\hfill
\begin{minipage}{0.6\linewidth}
\includegraphics[width=0.99\linewidth]{Bilder/Bernoulli} \\
\end{minipage}








$$ \underbrace{ p + \rho \cdot g \cdot h }_{\substack{\mathrm{statisch}}} + \underbrace{ \frac{1}{2} \, \rho \cdot v^2 }_{\substack{\mathrm{dynamisch}}} = \const $$
	
	$$ \boxed{ p_1 +  \rho \cdot g \cdot h_1 + \frac{1}{2} \, \rho \cdot v_1^2 = p_2 +  \rho \cdot g \cdot h_2 + \frac{1}{2} \, \rho \cdot v_2^2 } $$


\subsubsection{Spezialfall: Horizontal}

$$ \boxed{ p + \frac{1}{2} \, \rho \cdot v^2 = \const } $$


\subsubsection{Spezialfall: Statik}

$$  \boxed{ p + \rho \, \cdot g \cdot h =  \const} $$




\subsubsection{Hydrodynamisches Paradoxon}
\textbf{Je grösser die Strömungsgeschwindigkeit, desto kleiner der Druck} \\


%\vfill\null
%\columnbreak



\subsection{Bernoulli-Gleichung und Energieerhaltung} % eventuell weglassen
Die in der Bernoulli-Gleichung vorkommenden Terme können als \underline{Energie pro Volumen} betrachtet werden \\
\\
\begin{tabular}{l c l}
$  \mathrm{E_{Mech}}$ & $=$ & $\mathrm{elast. \; Energie +  pot. \; Energie + kin. \; Energie}$ \\
\\
& $=$ & $ p \cdot V + m \cdot g \cdot h + \frac{1}{2} \, m \cdot v^2 = \const$ \\
\\
\end{tabular}


Wenn durch das Volumen dividiert wird erhält man: \\
\\
\begin{tabular}{l c l}
$  \mathrm{\frac{E_{Mech}}{Volumen}}$ & $=$ & $\mathrm{\frac{elatische Energie}{Volumen}   + \frac{pot. \; Energie}{Volumen} + \frac{kin. \; Energie}{Volumen}}$ \\
\\
& $=$ & $ p + \rho \cdot g \cdot h + \frac{1}{2} \,  \rho \cdot v^2 =\const$ \\
\\
\end{tabular}


Bei einer horizontalen Strömung entfällt die pot. Energie\\
(pro Volumen) \\
\\

\begin{tabular}{l c l}
$ \mathrm{ \frac{E_{Mech}}{Volumen}}$ & $=$ & $\mathrm{ \frac{elatische Energie}{Volumen} + \frac{kin. \; Energie}{Volumen} }$ \\
\\
& $=$ & $ p + \frac{1}{2} \, \rho \cdot v^2 = \const$ \\
\end{tabular}





\section{Hydrodynamik - Reale Fluide}

\textbf{Reale Fluide nehmen Scherkräfte auf (Reibung)}


\subsection{Newton'sches Reibungs-Gesetz}
Ein \underline{reales Fluid} erfährt \underline{Reibung} 

$$ \boxed{ \tau = \eta \cdot \frac{v}{d} }  \qquad  \boxed{ \tau = \eta \cdot \frac{d \, v}{d \, z} } $$

\begin{tabular}{c l c}
		$\tau$ & Schubspannung & $[\tau] = \mathrm{N}$ \\
		$\eta$ & dynmaische Zähigkeit (Viskosität) & $[\eta] = \mathrm{Pa \cdot s}$ \\
		\rule{0pt}{8pt}$v$ & Geschwindigkeitsdifferenz zw. Auflagen & $[v] = \frac{m}{s}$ \\
		$z$ & Richtung senkrecht zur Verschiebung & $[z] = \mathrm{m}$ \\
		$d$ & Distand zwischen den Auflagen & $[d] = \mathrm{m}$ \\
		\rule{0pt}{8pt}$\frac{d \, v}{d \, z}$ & Geschwindigkeits-Gradient in z-Richtung & $[\frac{d \, v}{d \, z}] = \mathrm{\frac{1}{s}}$ \\
		\\
\end{tabular}
	
\textbf{Beispiele: Werte für $\eta$} \\ %TODO: Gradzeichen fixen!
\\
\begin{tabular}{l c l}
		$\eta_{Luft}$ & $:=$ & $17 \cdot 10^{-6} \; \mathrm{Pa \cdot s} $ \\
		$\eta_{Wasser} (20°C)$ & $:=$ & $10^{-2} \; \mathrm{Pa \cdot s}$ \\
		$\eta_{Oel}$ & $:=$ & $0.1 \; \mathrm{Pa \cdot s}$ bis $1 \; \mathrm{Pa \cdot s}$ \\
\end{tabular}


\subsubsection{Kinematische Zähigkeit $\nu$}

$$ \boxed{ \nu = \frac{\eta}{\rho}  } $$

\begin{tabular}{c l c}
		\rule{0pt}{8pt}$\nu$ & kinematische Zähigkeit & $[\nu] = \mathrm{\frac{m^2}{s}}$ \\
		\rule{0pt}{8pt}$\rho$ & Dichte & $[\rho] = \mathrm{\frac{kg}{m^3}}$  \\	
\end{tabular}
	

	
\subsection{Stokes'sche Reibung $F_R$}
Z.B. für Kugel in Öl oder fallende Wassertropfen

$$ \boxed{ F_R = 6 \cdot \pi \cdot \eta \cdot R \cdot v } $$

\begin{tabular}{c l c}
		$F_R$ & Reibungskraft & $[F_R] = \mathrm{N}$ \\
		$\eta$ & Dynamische Zähigkeit (Viskosität) & $[\eta] = \mathrm{Pa \cdot s}$  \\
		$R$ & Kugelradius & $[R] = \mathrm{m}$ \\
		\rule{0pt}{8pt}$v$ & Geschwindigkeit & $[v] = \mathrm{\frac{m}{s}}$		
\end{tabular}


\subsubsection{Kugelfall-Viskosimeter}
Auf eine Kugel, welche in einer Flüssigkeit hinabgleitet wirken \\
folgende Kräfte: \\
\\
\begin{minipage}{0.4\linewidth}
\includegraphics[width=0.8\linewidth]{Bilder/kugelfall-viskosimeter} \\
\end{minipage}
\hfill
\begin{minipage}{0.53\linewidth}
\begin{tabular}{ll}
$F_G$ & Gewichtskraft \\
$F_A$ & statischer Auftrieb \\
$F_R$ & Stokes'sche Reibung \\
\\
\end{tabular}

Ansatz zum Lösen von Aufgaben: \textbf{Kräftegleichgewicht}
\end{minipage}






\vfill\null
\columnbreak


\subsection{Hagen-Poiseuille}
Beschreibung von \underline{laminaren} Strömungen in einem \underline{runden Rohr} \\
$\Rightarrow$ Schichtströmung

\subsubsection{Gesetz von Hagen-Poiseuille}

$$ \boxed{ \dot{V} = \frac{\pi \cdot \Delta \, p \cdot R^4}{8 \cdot \eta \cdot l} } $$
 
\subsubsection{Geschwindigkeitsverteilung von $r=0$ bis R}

\includegraphics[width=0.8\linewidth]{Bilder/laminare_geschwindigkeit.png}

$$ \boxed{ v(r) = \frac{1}{4 \cdot \eta} \cdot \frac{\Delta \, p}{l} \, (R^2 - r^2) } $$



\begin{tabular}{c l c}
		\rule{0pt}{8pt}$v(r)$ & Fliessgeschwindigkeit beim Radius $r$ & $[v(r)] = \mathrm{\frac{m}{s}}$ \\
		$r$ & betrachteter Radius & $[r] = \mathrm{m}$ \\
		$\eta$ & Dynamische Zähigkeit (Viskosität) & $[\eta] = \mathrm{Pa \cdot s}$  \\
		$R$ & Rohr-(Innen)Radius & $[R] = \mathrm{m}$ \\
		$\Delta \, p$ & Druckdifferenz & $[\Delta \, p] = \mathrm{Pa}$ \\
		\rule{0pt}{8pt}$\dot{V} = \frac{d \, V}{d \, t}$ &  Volumenstrom & $[\dot{V}] = \mathrm{\frac{m^3}{s}}$	 \\
		$l$ & Länge des Rohrs & $[l] = \mathrm{m}$
\end{tabular}



\subsection{Reynolds-Zahl $Re$}
Gibt ein Richtmass für die Wirbelbildung  \\
\\
$\bullet$ Druck-Differenz (Bernoulli) begünstigt Wirbelbildung \\
$\bullet$ Innere Reibung (Schubspannung) verhindert Wirbelbildung 

$$\boxed{  Re = \frac{\Delta \, p}{\tau} = \frac{\rho \cdot \overline{v} \cdot d}{\eta} \qquad \qquad \mathrm{mit} \; \; \overline{v} = \frac{\dot{V}}{A} }  $$

	
\begin{tabular}{c l c}
		$Re$ & Reynolds-Zahl & $[Re] = 1$ \\
		$\eta$ & Dynamische Zähigkeit (Viskosität) & $[\eta] = \mathrm{Pa \cdot s}$  \\
		$\overline{v}$ & Mittlere Geschwindigkeit & $[\overline{v}] = \mathrm{\frac{m}{s}}$ \\
		$d$ & Typische Dimension (Rohrdurchmesser) & $[d] = \mathrm{m}$ \\
		$\Delta \, p$ & Druckdifferenz & $[\Delta p] = \mathrm{Pa}$ \\
		$\tau$ & Schubspannung & $[\tau] = \mathrm{N}$ \\
		\\
\end{tabular}

\textbf{Sobald die Reynolds-Zahl $Re$ grösser ist als ein kritischer Wert bilden sich Wirbel} \\
\\
$\Rightarrow$ Rohr:  $Re_{kritisch} \approx 2320$


\subsubsection{Ähnlichkeitsgesetz}
Reynolds-Zahl dient auch richtigem Vergleich von Modellversuchen. \\
\\
$\Rightarrow$ Gleiche Reynolds-Zahl bedeutet gleiches Verhalten \\
\\
$\Rightarrow$ Gleiche Reynolds-Zahl bedeutet auch gleiche \\
 Relative Grenzschicht-Dicke $D$ (siehe \ref{Grenzschichtdicke})



% \vfill\null
% \columnbreak


\subsection{Turbulente / Laminare Rohrströmung}

\subsubsection{Hilfe, um Reynoldszahl zu bestimmen (laminar)}

$$ \boxed{ \Delta p = 32 \cdot \eta \cdot l \cdot \frac{v}{d^2} }  $$


\subsubsection{Druckunterschied in laminare / turbulente Strömung}

$$ \lambda_{turbulent} = \frac{0.316}{\sqrt[4]{Re}}  \qquad \qquad \lambda_{laminar} = \frac{64}{Re}  $$

$$ \boxed{ \Rightarrow \Delta p_x = \lambda_x \frac{l}{d} \cdot \frac{\rho}{2} \cdot v^2 } $$



\begin{tabular}{c l c}
		$\Delta \, p_x$ & Druckdifferenz (laminar/turbulent) & $[\Delta p] = \mathrm{Pa}$ \\
		$\eta$ & Dynamische Zähigkeit (Viskosität) & $[\eta] = \mathrm{Pa \cdot s}$  \\
		$l$ & Rohr-Länge & $[l] = \mathrm{m}$ \\
		\rule{0pt}{8pt}$v$ & Fliess-Geschwindigkeit & $[v] = \mathrm{\frac{m}{s}}$ \\
		$d$ & Rohr-Durchmesser & $[d] = \mathrm{m}$ \\		
		\rule{0pt}{8pt}$\rho$ & Dichte des Fluids & $[\rho] = \mathrm{\frac{kg}{m^3}}$ \\
		$Re$ & Reynolds-Zahl & $[Re] = 1$ \\
\end{tabular}




\subsubsection{Unbekannt / Gemischt (Pratische Anwendung)}
Vorgehen, wenn man nicht weiss, ob sich Wirbel bilden oder nicht \\
\\
\begin{tabular}{ll}
1. & Laminar rechnen (um fehlenden Parameter $\rho, \; v, \; d, \; \mathrm{oder} \; \eta$ \\
   &  zu bestimmen) \\
2. & Aus Resultat Reynolds-Zahl berechnen \\
3. & Mit kritischer Reynolds-Zahl vergleichen \\
4. & Beim \textbf{Überschreiten} $\Rightarrow$ Turbulent rechnen! \\
\end{tabular}



\subsection{Prandl'sche Grenzschicht-Dicke $D$}\label{Grenzschichtdicke}
Prandl'sche Grenzschicht-Dicke $D$ beschreibt, in welcher \textbf{Distanz} die \textbf{Geschwindigkeit} eines laminar bewegten Teils (z.B. ein \\
Flugzeugflügel) \textbf{Null} ist. 

$$\boxed{ D = \sqrt{\frac{\eta}{\rho} \cdot \frac{l}{v}} }$$


\begin{tabular}{c l c}
		$D$ & Prandl'sche Grenzschicht-Dicke & $[D] = \mathrm{m}$ \\
		$\eta$ & Dynamische Zähigkeit (Viskosität) & $[\eta] = \mathrm{Pa \cdot s}$  \\
		\rule{0pt}{8pt}$\rho$ & Dichte des Fluids & $[\rho] = \mathrm{\frac{kg}{m^3}}$ \\
		$l$ & Länge des bewegten Teils (in Richtung von $v$) & $[l] = \mathrm{m}$ \\
		\rule{0pt}{8pt}$v$ & Geschwindigkeit & $[v] = \mathrm{\frac{m}{s}}$ \\
			\\
\end{tabular}

\begin{minipage}{0.48\linewidth}
\includegraphics[width=0.8\linewidth]{Bilder/prandl}
\end{minipage}
\hfill
\begin{minipage}{0.48\linewidth}
Die Geschwindigkeit innerhalb der Grenzschicht $D$ nimmt vom Teil bis hin zum äussersten Rand \textbf{linear} ab.
\end{minipage}


% \vfill\null
% \columnbreak




\subsection{Bernoulli-Gleichung mit innerer Reibung}

$$ \boxed{  p_1 +  \rho \cdot g \cdot h_1 + \frac{1}{2} \, \textcolor{red}{\alpha_1} \cdot \rho \cdot v_1^2 = p_2 +  \rho \cdot g \cdot h_2 + \frac{1}{2} \, \textcolor{red}{\alpha_2} \cdot \rho \cdot v_2^2 \textcolor{red}{+ \Delta \, p_v}  }$$

\begin{tabular}{c| c |c}
 & turbulent & laminar \\ 
\hline 
Korrekturfaktoren & $\alpha_1 \approx \alpha_2 \approx 2$  & $\alpha_1 \approx \alpha_2 \approx 1$ \\ 
\hline 
\rule{0pt}{11pt} Druckverlust $\Delta \, p_v$ & \multicolumn{2}{c}{$\Delta p_v = \lambda_x \frac{l}{d} \cdot \frac{\rho}{2} \cdot v^2$} \\ 
\hline 
\rule{0pt}{11pt}  & $\lambda_{turbulent} = \frac{0.316}{\sqrt[4]{Re}}  $ & $\lambda_{laminar} = \frac{64}{Re} $  \\ 
\end{tabular} 



\subsection{Druckwiderstand $F_D$}
Bezeichnet die turbulente Luftreibungskraft $F_R$ und wird meist als Luftwiderstand bezeichnet \\

\includegraphics[width=0.8\linewidth]{Bilder/widerstandsbeiwert.png}

$$ \boxed{ F_D = \Delta \, p \cdot A_s = \frac{1}{2} \, \cdot \rho \cdot v^2 \cdot A_s \cdot c_W } $$


\begin{tabular}{c l c}
		$F_D$ & Druckwiderstand & $[F_D] = \mathrm{N}$ \\
		$\Delta \, p$ & Druckdifferenz & $[\Delta \, p] = \mathrm{Pa}$  \\
		\rule{0pt}{8pt}$\rho$ & Luft-Dichte & $[\rho] = \mathrm{\frac{kg}{m^3}}$ \\
		\rule{0pt}{8pt}$v$ & Strömungs-Geschwindigkeit & $[v] = \mathrm{\frac{m}{s}}$ \\
		$c_W$ & Widerstandsbeiwert / Widerstandszahl & $[c_W] = 1$\\
		$A_s$ & projizierte Fläche senkrecht zur Strömung & $[A_s] = \mathrm{m^2}$ \\
		\\
\end{tabular}

Der Widerstandsbeiwert $c_W$ ist \textbf{geometrieabhängig}!






\subsection{Auftriebskraft $F_A$ nach Kutta-Jukowski}
Beschreibt Proportionalität zwischen dynamischem Auftrieb \\
und Zirkulation 

$$ \boxed{ F_A = \rho \cdot v \cdot l \cdot \Gamma } $$

\begin{tabular}{c l c}
		$F_A$ & dynamischer Auftrieb & $[F_A] = \mathrm{N}$ \\
		\rule{0pt}{8pt}$\rho$ & Dichte des Fluids & $[\rho] = \mathrm{\frac{kg}{m^3}}$ \\
		\rule{0pt}{8pt}$v$ & Geschwindigkeit & $[v] = \mathrm{\frac{m}{s}}$ \\
		$l$ & Länge quer zur Strömung & $[l] = \mathrm{m}$ \\
		\rule{0pt}{8pt}$\Gamma$ & Zirkulation & $[\Gamma] = \mathrm{\frac{m^2}{s}}$ \\
\end{tabular}






\subsubsection{Zirkulation $\Gamma$}
Die Zirkulation ist ein Mass für die \textbf{Rotation} im Strömungsfeld \\

$$ \boxed{ \Gamma = 	\oint \vec{v} \bullet d\vec{s} } $$


\begin{tabular}{c l c}
		\rule{0pt}{8pt}$\Gamma$ & Zirkulation & $[\Gamma] = \mathrm{\frac{m^2}{s}}$ \\
		\rule{0pt}{8pt}$\vec{v} \bullet d\vec{s}$ & Geschwindigkeit entlang dem Weg & $[\vec{v}] = \mathrm{\frac{m}{s}}$  \\
		& (Skalarprodukt: $\vec{v} \bullet d\vec{s} = a \cdot b \cdot \cos(\varphi)$ & \\
\end{tabular} \\
\\
\textbf{Rotierender Zylinder:} $$ \boxed{ \Gamma = 2\pi r v_{Zyl} = 4\pi^2r^2f } $$


% \vfill\null
% \columnbreak



\subsection{Dynamischer Auftrieb $F_A$}

$$ \boxed{ F_A = c_A \cdot  \underbrace{\frac{1}{2} \cdot \rho \cdot v^2 }_{\substack{\Delta \, p}} \cdot A_{\|}	} $$


\begin{tabular}{c l c}
		$F_A$ & dynamischer Auftrieb & $[F_A] = \mathrm{N}$ \\
		$c_A$ & Auftriebskoeffizient & $[c_A] = 1$ \\
		\rule{0pt}{8pt}$\rho$ & Luft-Dichte & $[\rho] = \mathrm{\frac{kg}{m^3}}$ \\
		\rule{0pt}{8pt}$v$ & Strömungsgeschwindigkeit & $[v] = \mathrm{\frac{m}{s}}$ \\
		$A_{\|}$ & Projizierte Fläche \textbf{parallel} zur Strömung & $[A_{\|}] = \mathrm{m^2}$ \\
\end{tabular}




\subsubsection{Wissenswertes zum dynamischen Auftrieb}
Ein gerade ausgerichtetes, symmetrisches Stromlinienprofil erzeugt \textbf{keinen} dynamischen Auftrieb \\
\\
An einem asymmetrischen Flügelprofil entsteht dynamischer\\
Auftrieb 


\subsection{Induzierter Widerstand $F_W$}
Kommt durch Energieverlust (Wirbelbildung) zu Stande, welcher entsteht, wenn die Umgebungsluft in Bewegung gesetzt wird

$$ \boxed{ F_W = c^*_W \cdot \frac{1}{2} \cdot \rho \cdot v^2 \cdot A_{\|} } $$


\begin{tabular}{c l c}
		$F_W$ & Induzierter Widerstand & $[F_W] = \mathrm{N}$ \\
		$c^*_W$ & Widerstands-Koeffizient & $[c*_W] = 1$ \\
		\rule{0pt}{8pt}$\rho$ & Luft-Dichte & $[\rho] = \mathrm{\frac{kg}{m^3}}$ \\
		\rule{0pt}{8pt}$v$ & Strömungsgeschwindigkeit & $[v] = \mathrm{\frac{m}{s}}$ \\
		$A_{\|}$ & Projizierte Fläche \textbf{parallel} zur Strömung & $[A_{\|}] = \mathrm{m^2}$ \\
\end{tabular}


\subsection{Gleitwinkel $\varphi$}
Gibt die zurückgelegte Stecke pro verbrauchte Höhe an \\
Im Luft-Kanal ist dies der Anstell-Winkel \\

\begin{minipage}{0.5\linewidth}
	\includegraphics[width=\linewidth]{Bilder/gleitwinkel.png}
\end{minipage}
\hfill
\begin{minipage}{0.45\linewidth}
	$$ \boxed{ tan(\varphi) = \frac{F_W}{F_A} = \frac{c^*_W}{c_A}= \frac{v_V}{v_H} }	$$
\end{minipage}






\begin{tabular}{c l c}
		$\varphi$ & Gleitwinkel & $[\varphi] = \text{°}$ \\
		$F_W$ & Widerstandskraft & $[F_W] = \mathrm{N}$ \\
		$F_A$ & Auftriebskraft & $[F_A] = \mathrm{N}$ \\ 
		$c^*_W$ & Widerstands-Koeffizient & $[c^*_W] = 1$ \\
		$c_A$ & Auftriebs-Koeffizient & $[c_A] = 1$ \\
		\rule{0pt}{8pt}$v_V$ & Vertikal-Geschwindigkeit & $[v_V] = \mathrm{\frac{m}{s}}$ \\
		\rule{0pt}{8pt}$v_H$ & Horizontal-Geschwindigkeit & $[v_H] = \mathrm{\frac{m}{s}}$ \\
\end{tabular}

\subsubsection{Gängige Gleitzahlen}
\begin{center}
    \begin{tabular}{lc}
        \textbf{Flugobjekt} & \textbf{Gleitzahl}\\ \hline
		Hängegleiter  & 10 bis 15 \\
		Boeing 747    & 15 \\
		Airbus A380   & 20 \\
		Segelflugzeug &	40 (Rekord 70) \\
	\end{tabular}
\end{center}


\subsection{Helmholz'sche Wirbelsätze}
\begin{tabular}{ll}
1. & Wirbel hat kein Anfang und kein Ende \\
2. & Wirbel besteht immer aus denselben Fluidteilchen \\
3. & Zirkulation zeitlich konstant \\
\end{tabular}

% \vfill\null
% \columnbreak

		\section{Thermodynamik}

\subsection{Terminologie}

\begin{center}
    \begin{tabular}{l||l|lll}
        \textbf{System ist $\downarrow$} & \textbf{Materie-} & & \multicolumn{2}{l}{\textbf{Energietausch}}  \\
			& \textbf{tausch} &	& Arbeit & Wärme \\ \hline
		offen & erlaubt & - & erlaubt  & erlaubt \\
		      &         & adiabatisch  & erlaubt & Nein\\
			  &         & arbeitsdicht & Nein    & erlaubt\\
			  &         & beides       & Nein    & Nein\\ \hline
		geschlossen & Nein & - & möglich  & möglich \\
		      &         & adiabatisch  & möglich & Nein\\
			  &         & arbeitsdicht & Nein    & möglich\\
			  &         & energiedicht & Nein    & Nein\\
	\end{tabular}
\end{center}

\includegraphics[width=\linewidth]{Bilder/thermodynamisches_system.png}

\subsection{Absolute Temperatur $T$}
$$ \boxed{ T = \theta + 273.15 \, K = \theta - \theta_0 }$$

\begin{tabular}{c l c}
		$T$ & Absolute Temperatur gemessen in Kelvin & $[T] =\mathrm{K}$  \\
		$\theta$ & Temperatur gemessen in °C & $[\theta] = \text{°C}$ \\
		$\theta_0$ & Absoluter Nullpunkt: $= -273.15 \,  \text{°C} = 0 \, \mathrm{K}$  &  \\
\end{tabular}




\subsection{Thermische Ausdehnung}

\subsubsection{Längenausdehnung $\Delta \, l $}

$$ \boxed{  l' = l + \Delta \,  l = l + \alpha \cdot l \cdot \Delta \, T = l\, (1 + \alpha \cdot \Delta \, T ) }$$


\begin{tabular}{c l c}
		$l'$ & Länge nach Ausdehnung & $[l'] = \mathrm{m}$ \\
		$l$ & Anfangslänge & $[l] = \mathrm{m}$ \\
		$\Delta \, l$ & Längenänderung & $[\Delta \, l] = \mathrm{m}$ \\
		\rule{0pt}{8pt}$\alpha$ & Längenausdehnungskoeffizient & $[\alpha] = \mathrm{\frac{1}{K}}$ \\ 
		$\Delta \, T $ & Temperaturänderung & $[\Delta \, T ] = \mathrm{K}$ \\
\end{tabular}


\subsubsection{Flächenausdehnung $\Delta \, A $}

$$ \boxed{ A' = A + \Delta \,  A = A + \underbrace{  \beta }_{\substack{\approx 2 \, \alpha}}  \cdot A \cdot \Delta \, T = A\, (1 + \beta \cdot \Delta \, T ) } $$


\begin{tabular}{c l c}
		$A'$ & Länge nach Ausdehnung & $[A'] = \mathrm{m^2}$ \\
		$A$ & Anfangslänge & $[A] = \mathrm{m^2}$ \\
		$\Delta \, A$ & Längenänderung & $[\Delta \, A] = \mathrm{m^2}$ \\
		\rule{0pt}{8pt}$\beta$ & Flächenausdehnungskoeffizient & $[\beta] = \mathrm{\frac{1}{K}}$ \\ 
		$\Delta \, T $ & Temperaturänderung & $[\Delta \, T ] = \mathrm{K}$ \\
\end{tabular}



\subsubsection{Volumenausdehnung $\Delta \, V $}

$$ \boxed{ V' = V + \Delta \,  V = V + \underbrace{  \gamma }_{\substack{\approx 3 \, \alpha}}  \cdot V \cdot \Delta \, T = V \, (1 + \gamma \cdot \Delta \, T ) }$$


\begin{tabular}{c l c}
		$V'$ & Volumen nach Ausdehnung & $[V'] = \mathrm{m^3}$ \\
		$V$ & Anfangsvolumen & $[V] = \mathrm{m^3}$ \\
		$\Delta \, V$ & Volumenänderung & $[\Delta \, V] = \mathrm{m^3}$ \\
		\rule{0pt}{8pt}$\gamma$ & Volumenausdehnungskoeffizient & $[\gamma] = \mathrm{\frac{1}{K}}$ \\ 
		$\Delta \, T $ & Temperaturänderung & $[\Delta \, T ] = \mathrm{K}$ \\
\end{tabular}

% \vfill\null
% \columnbreak

\begin{center}
    \begin{tabular}{lc}
        \textbf{Material} & \textbf{Koeffizient ($10^{-6} K^{-1} $)}\\ \hline
		Aluminium       & 23 \\
		Eisen           & 12 \\
		Stahl, unlegiert& 11 ... 13 \\
		Diamant         & 1.3 \\
		Silizium        & 2 \\
		Gummi           & 220 \\
		Beton           & 12 \\
		Polysterol      & 70 \\
		Zerodur         & 0 $\pm$ 0.007 \\
	\end{tabular}
\end{center}




\subsection{Thermische Spannung $\sigma$}


$$\boxed{  p = \sigma = \varepsilon \cdot E = E \cdot \frac{\Delta l}{l} =  E \cdot \alpha \cdot \Delta \, T } $$

\begin{tabular}{c l c}
		$\sigma$ & Thermische Spannung & $[\sigma] = \mathrm{Pa}$ \\
		$\varepsilon$ & Dehnung & $[\varepsilon] = 1$ \\
		\rule{0pt}{8pt}$E$ & Elastizitätsmodul & $[E] = \mathrm{\frac{N}{m^2}}$ \\
		\rule{0pt}{8pt}$\alpha$ & Längenausdehnungskoeffizient & $[\alpha] = \mathrm{\frac{1}{K}}$ \\ 
		$\Delta \, T $ & Temperaturänderung & $[\Delta \, T ] = \mathrm{K}$ \\
		$p$ & Druck & $[p] = \mathrm{Pa} $ \\
\end{tabular}






\section{Ideales Gas}

\subsection{Modell des idealen Gases}
\textbf{Jedes Gas ist gleich!} \\


\begin{tabular}{ll}
$1.$ & Moleküle sind Massepunkte (keine Ausdehnung) \\
$2.$ & Stösse sind elastisch (keine zwischenmolekularen Kräfte) \\
& Kein Volumen bei $T = 0$ \\
& Kein Druck bei $T = 0$ \\
\end{tabular}





\subsubsection{Thermische Ausdehnung von Gasen}
\begin{tabular}{ll}
$\bullet$ & Ausdehnung von Gasen ist sehr gross \\
$\bullet$ & Bei \textbf{allen} Gasen ist die Ausdehnung \textbf{gleich} \\
$\bullet$ & Volumen beim Nullpunkt ist \textbf{Null} \\
\end{tabular}


\vfill\null
\columnbreak

\subsection{Universelle Gasgleichung}
Alle Gase verhalten sich gleich, insbesondere bei gleicher Anzahl Moleküle \\


$$ \boxed{ \frac{p \cdot V}{T} = \, \const } \qquad  \Rightarrow \boxed{ \frac{p_1 \cdot V_1}{T_1} = \frac{p_2 \cdot V_2}{T_2} } $$ 


\begin{tabular}{c l c}
	$p_x$ & \textbf{Absolut-}Druck & $[p_x] = \mathrm{Pa}$ \\
	& Absolut-Druck: $p_0 + p$ \\
	$V_x$ & Volumen & $[V_x] = \mathrm{m^3}$ \\
	$T_x$ & \textbf{Absolut-}Temperatur (in K) & $[T] = \mathrm{K}$ \\
\end{tabular}
	
% \vfill\null
% \columnbreak

	
\subsubsection{Boyle-Mariotte}	
\textbf{Das Gesetz gilt nur bei konstanter Temperatur!} \\
$\Rightarrow$ \textbf{Isotherme} Zustandsänderung

$$  \boxed{ p \cdot V = \, \const } \qquad  \Rightarrow  \boxed{ p_1 \cdot V_1 = p_2 \cdot V_2 } $$ 





\subsubsection{Gay-Lussac}

\textbf{ Das Gesetz gilt nur bei konstantem Druck!} \\
$\Rightarrow$ \textbf{Isobare} Zustandsänderung

$$  \boxed{ \frac{V}{T} = \; \const } \qquad  \Rightarrow  \boxed{ \frac{V_1}{T_1} = \frac{V_2}{T_2} } $$


	
	
\subsubsection{Gay-Lussac und Amontons}

\textbf{Das Gesetz gilt nur bei konstantem Volumen!} \\
$\Rightarrow$ \textbf{Isochore} Zustandsänderung

$$ \boxed{ \frac{p}{T} = \; \const } \qquad  \Rightarrow \boxed{ \frac{p_1}{T_1} = \frac{p_2}{T_2} }$$


	
\subsection{Universelle Gasgleichung für ideale Gase}\label{Gasgleichung ideal}

$$ \boxed{ p \cdot V = n \cdot R \cdot T = N \cdot k \cdot T }$$
	
	
\begin{tabular}{c l c}
	$p$ & \textbf{Absolut-}Druck & $[p] = \mathrm{Pa}$ \\
	    & Absolut-Druck: $p_0 + p$ & \\
	$V$ & Volumen & $[V] = \mathrm{m^3}$ \\
	$n$ & Mol-Zahl & $[n] = \mathrm{mol}$ \\
	\rule{0pt}{8pt}$R$ & Universelle Gaskonstante: $R = 8.314 \mathrm{\frac{J}{mol \cdot K}}$ & $[R] = \mathrm{\frac{J}{mol \cdot K}} $ \\
	$T$ & \textbf{Absolut-}Temperatur (in K) & $[T] = \mathrm{K}$ \\
	$N$ & Anzahl Moleküle & $[N] = 1$ \\
	\rule{0pt}{8pt}$k$ & Boltzmann-Konstante $k = 1.381 \cdot 10^{-23} \mathrm{\frac{J}{K}}$ & $[k] = \mathrm{\frac{J}{K}}$ \\
\end{tabular}
	
	
	
	
\subsubsection{Zusammenhänge zwischen den Konstanten}
	
$$  \boxed{ R = k \cdot N_A = \frac{N \cdot k}{n} } $$

$$\boxed{ n = \frac{N}{N_A} = \frac{m}{M} = \frac{N \cdot k}{R} } $$
	


\begin{tabular}{c l c}
	\rule{0pt}{8pt}$R$ & Universelle Gaskonstante: $R = 8.314 \mathrm{\frac{J}{mol \cdot K}}$ & $[R] = \mathrm{\frac{J}{mol \cdot K}} $ \\
	\rule{0pt}{8pt}$k$ & Boltzmann-Konstante $k = 1.381 \cdot 10^{-23} \mathrm{\frac{J}{K}}$ & $[k] = \mathrm{\frac{J}{K}}$ \\
	$N$ & Anzahl Moleküle & $[N] = 1$ \\
	\rule{0pt}{8pt}$N_A$ & 	Avogadrokonstante: $N_A = 6.022 \cdot 10^{23} \, \mathrm{\frac{1}{mol}} $ & $[N_A] =  \mathrm{\frac{1}{mol}}$  \\	
	$n$ & Mol-Zahl & $[n] = \mathrm{mol}$ \\
	$m$ & Masse & $[m] = \mathrm{kg}$ \\
	\rule{0pt}{8pt}$M$ & Mol-Masse & $[M] = \mathrm{\frac{kg}{mol}}$ \\
\end{tabular}

% \vfill\null
% \columnbreak


\subsection{Mechanische Arbeit $\Delta W$ von Gasen}
\label{MechArbeit}

Folgende Formel ist für Flüssigkeiten \textbf{nicht} gültig, da diese \\
inkompressibel sind ($\Delta V = 0$)


$$ \boxed{ \Delta W = F \cdot \Delta s = p \cdot A \cdot \Delta s = p \cdot \Delta V } $$
\\

\begin{tabular}{c l c}
	$\Delta W$ & Mechanische Arbeit von Gas & $[\Delta W] = \mathrm{J}$ \\
	$F$ & Kraft & $[F] = \mathrm{N}$ \\
	$\Delta s$ & Wegänderung & $[\Delta s] = \mathrm{m}$ \\
	$p$ & Druck & $[p] = \mathrm{Pa}$ \\
	$A$ & Fläche & $[A] = \mathrm{m^2}$ \\
	$\Delta V$ & Volumenänderung & $[\Delta V] = \mathrm{m^3}$ \\
\end{tabular}



\subsection{Gesetz von Avogadro}
Ein Mol eines Gases nimmt bei Normalbedingungen immer das \\
gleiche Volumen ein (=Molvolumen) \\
\\
Ideale Gase enthalten bei gleichem Druck p und gleicher \\
Temperatur T immer gleich viele Moleküle (im Molvolumen)





\subsection{Molmasse $M$, Molvolumen $V_m$}

Siehe auch \ref{Gasgleichung ideal}\\

Molmasse ist die \textbf{Ordnungszahl} im Periodensystem

$$  \boxed{ n = \frac{m}{M} = \frac{N}{N_A} } $$


Mol-Volumen:

$$ \boxed{ V_m = \frac{V}{n} }$$



\begin{tabular}{c l c}
	$p$ & \textbf{Absolut-}Druck & $[p] = \mathrm{Pa}$ \\
	    & Absolut-Druck: $p_0 + p$ & \\
	$V$ & Volumen & $[V] = \mathrm{m^3}$ \\
	\rule{0pt}{8pt}$R$ & Universelle Gaskonstante: $R = 8.314 \mathrm{\frac{J}{mol \cdot K}}$ & $[R] = \mathrm{\frac{J}{mol \cdot K}} $ \\
	$T$ & \textbf{Absolut-}Temperatur (in K) & $[T] = \mathrm{K}$ \\
	\rule{0pt}{8pt}$N_A$ & 	Avogadrokonstante: $N_A = 6.022 \cdot 10^{23} \, \mathrm{\frac{1}{mol}} $ & $[N_A] =  \mathrm{\frac{1}{mol}}$  \\	
	\rule{0pt}{8pt}$k$ & Boltzmann-Konstante $k = 1.381 \cdot 10^{-23} \mathrm{\frac{J}{K}}$ & $[k] = \mathrm{\frac{J}{K}}$ \\	
	$n$ & Mol-Zahl & $[n] = \mathrm{mol}$ \\
	$m$ & Masse & $[m] = \mathrm{kg}$ \\
	\rule{0pt}{8pt}$M$ & Mol-Masse & $[M] = \mathrm{\frac{kg}{mol}}$ \\
	$N$ & Anzahl Moleküle & $[N] = 1$ \\
	\rule{0pt}{8pt}$V_m$ & Mol-Volumen & $[V_m] = \mathrm{\frac{m^3}{mol}}$ \\
\end{tabular}

% \vfill\null
% \columnbreak



\subsection{Dichte eines Gases $\rho$}

$$ \boxed{ \rho = \frac{m}{V} = \frac{M}{V_m} = \frac{p \cdot M}{R \cdot T} }$$


\begin{tabular}{c l c}
	\rule{0pt}{8pt}$\rho$ & Gas-Dichte & $[\rho] = \mathrm{\frac{kg}{m^3}}$ \\
	$m$ & Masse & $[m] = \mathrm{kg}$ \\
	$V$ & Volumen & $[V] = \mathrm{m^3}$ \\
	\rule{0pt}{8pt}$M$ & Mol-Masse & $[M] = \mathrm{\frac{kg}{mol}}$ \\
	\rule{0pt}{8pt}$V_m$ & Mol-Volumen (22.4 L bei 0 °C und 1000 hPa) & $[V_m] = \mathrm{\frac{m^3}{mol}}$ \\
	\rule{0pt}{8pt}$p$ & \textbf{Absolut-}Druck & $[p] = \mathrm{Pa}$ \\
	    & Absolut-Druck: $p_0 + p$ & \\
	\rule{0pt}{8pt}$R$ & Universelle Gaskonstante: $R = 8.314 \mathrm{\frac{J}{mol \cdot K}}$ & $[R] = \mathrm{\frac{J}{mol \cdot K}} $ \\
	$T$ & \textbf{Absolut-}Temperatur (in K) & $[T] = \mathrm{K}$ \\
\end{tabular}



\subsection{Phänomene von idealen Gasen}

\subsubsection{Annomalie des Wassers}
Die feste Form (Eis) ist leichter als die flüssige Form (Wasser) \\
Die \textbf{grösste Dichte weist Wasser bei 4 °C} auf, nicht beim \\
 Gefrierpunkt von 0 °C \\

$\Rightarrow$ Ein See gefriert somit nur an der Oberfläche. Am Grund des Sees beträgt die Wassertemperatur 4 °C 


\subsubsection{Osmotischer Druck (Zelldruck)}

Grosse Moleküle innerhalb von vielen kleinen Molekülen in einer Flüssigkeit verhalten sich ähnlich wie die Moleküle eines idealen \\ Gases, wenn die Flüssigkeit von einer für die Müleküle \\
halb-durchlässigen (semi-permeabel) Membran umgeben ist.\\
\\
$ \mathrm{Osmotischer \; Druck:} \; p = \frac{n}{V} \cdot R \cdot T  \qquad \mathrm{(ideale \; Gasgleichung)}$

\vfill\null
\columnbreak

\subsection{Partialdruck $p_i$}
\textbf{Ausgangslage: Gasgemisch (z.B. Luft: Sauerstoff-Stickstoff)} \\
\\

\begin{minipage}{0.48\linewidth}
\includegraphics[width=\linewidth]{Bilder/partialdruck}
\end{minipage}
\hfill
\begin{minipage}{0.5\linewidth}
Der Partialdruck $p_i$ ist der Druck, welcher die i-te Gaskomponete erzeugen würde, wenn ihr das gesamte Volumen zur Verfügung stehen würde. \\
\end{minipage}

% \vfill\null
% \columnbreak

\subsection{Gesetz von Dalton}
In einem Gas ist die Summe der Partialdrücke $p_i$ gleich dem \\
Gesamtdruck 

$$ \boxed{ \sum_{i=1}^n  p_i = p } $$ 



\begin{tabular}{c l c}
	$p_i$ & Partialdruck & $[p_i] = \mathrm{Pa}$ \\
	$p$ & (Gesamt-) Druck & $[p] = \mathrm{Pa}$ \\
\end{tabular}


\subsection{Volumen- und Massenkonzentration (Gasgemisch)}


\subsubsection{Volumen-Konzentrationen (Volumen-Anteile)}


$$ \boxed{  q_i = \frac{V_i}{V} = \frac{n_i}{n} = \frac{p_i}{p} } $$


\begin{tabular}{c l c}
	$q_i$ & Volumen-Konzentration& $[q_i] = 1$ \\
	$V_i$ & Volumen der i-ten Gas-Komponente & $[V_i] = \mathrm{m^3}$ \\
	$V$ & Gesamt-Volumen & $[V] = \mathrm{m^3}$ \\
	$n_i$ & Molzahl der i-ten Gas-Komponente & $[n_i] = \mathrm{mol}$ \\
	$n$ & Gesamt-Molzahl des Gemischs & $[n] = \mathrm{mol}$ \\
	$p_i$ & Partialdruck der i-ten Gaskomponente & $[p_i] = \mathrm{Pa}$ \\
	$p$ & Druck des Gemischs & $[p] = \mathrm{Pa}$ \\
\end{tabular}






\subsubsection{Massen-Konzentration (Massen-Anteile)}

$$ \boxed{ \mu_i = \frac{m_i}{m} = \frac{M_i}{M} \cdot q_i } $$


\begin{tabular}{c l c}
	$\mu_i$ & Volumen-Konzentrationen & $[\mu_i] = 1$ \\
	$m_i$ & Masse der i-ten Gas-Komponente & $[m_i] = \mathrm{kg}$ \\
	$m$ & Masse der Gemischs & $[m] = \mathrm{kg}$ \\
	\rule{0pt}{8pt}$M_i$ & Mol-Masse der i-ten Gas-Komponete & $[M_i] = \mathrm{\frac{kg}{mol}}$ \\
	\rule{0pt}{8pt}$M$ & Mol-Masse des Gemischs & $[M] = \mathrm{\frac{kg}{mol}}$ \\
	$q_i$ & Volumen-Konzentration& $[q_i] = 1$ \\
\end{tabular}



\subsection{Mol-Masse Gasgemisch}
Die Mol-Masse des Gas-Gemischs kann als gewichteter Mittelwert \\
berechnet werden, gewichtet mit den jeweiligen Volumen-Anteilen  

%\textbf{Die Summe aller $q_i$ muss 1 (bzw. 100%) sein!}


$$ \boxed{ M = \sum_{i=1}^n  q_i \cdot M_i } $$

\begin{tabular}{c l c}
	\rule{0pt}{8pt}$M$ & Mol-Masse Gasgemisch & $[M] = \mathrm{\frac{kg}{mol}} $ \\
	$q_i$ & Volumen-Konzentration& $[q_i] = 1$ \\
	\rule{0pt}{8pt}$M_i$ & Mol-Masse der i-ten Gas-Komponete & $[M_i] = \mathrm{\frac{kg}{mol}}$ \\
\end{tabular}

% \vfill\null
% \columnbreak

\section{Reales Gas}
Im Vergleich zum idealen Gas müssen zwei Dinge berücksichtigt \\
werden: \\


Eigen-Volumen: \\
Ideales Gas hat \textbf{kleineres} Volumen als gemessen \\
(Ideal-Gas-Volumen um das Molekül-Eigenvolumen reduzieren) \\
\\
Binnen-Druck: \\Ideales Gas hat \textbf{grösseren} Druck als gemessen \\
(Ideal-Gas-Druck um Binnendruck erhöhen) 




\subsection{Van der Waals-Gleichung (1 Mol)}
\textbf{$\Rightarrow$ Für nicht-ideale Gase!} 

$$ \boxed{ p' \cdot V'_m = R \cdot T  }$$

$$ \boxed{ p' = p + \frac{a}{V_m^2} }  \qquad \qquad  \boxed{ V'_m = V_m - b }$$


\begin{tabular}{c l c}
	$p'$ & Korrigierter Druck & $[p'] = \mathrm{Pa}$ \\
	\rule{0pt}{10pt}$V'_m$ & Korrigiertes Mol-Volumen & $[V_m] = \mathrm{\frac{m^3}{mol}}$ \\
	\rule{0pt}{10pt}$R$ & Universelle Gaskonstante: $R = 8.314 \mathrm{\frac{J}{mol \cdot K}}$ & $[R] = \mathrm{\frac{J}{mol \cdot K}} $ \\
	$T$ & \textbf{Absolut-}Temperatur (in K) & $[T] = \mathrm{K}$ \\
	$p$ & Druck des Gemischs & $[p] = \mathrm{Pa}$ \\
	\rule{0pt}{10pt}$a$ & Eigenvolumen & $[a] = \mathrm{\frac{J \cdot m^3}{mol^2}}$ \\
	\rule{0pt}{10pt}$b$ & Binnendruck & $[b] = \mathrm{\frac{m^3}{mol}}$ \\
	\rule{0pt}{10pt}$V_m$ & Mol-Volumen & $[V_m] = \mathrm{\frac{m^3}{mol}}$ \\
\end{tabular}

% \vfill\null
% \columnbreak


\subsection{Van der Waals-Gleichung (n Mol)}

$$ \boxed{ \Big(  p + \frac{n^2 \cdot a}{V^2} \Big)  \cdot (V - n \cdot b) = n \cdot R \cdot T } $$



\begin{tabular}{c l c}
	$p$ & Druck des Gemischs & $[p] = \mathrm{Pa}$ \\
	$n$ & Mol-Zahl & $[n] = \mathrm{mol}$ \\
	\rule{0pt}{8pt}$a$ & Eigenvolumen & $[a] = \mathrm{\frac{J \cdot m^3}{mol^2}}$ \\
	$V$ & Volumen & $[V] = \mathrm{m^3}$ \\
	\rule{0pt}{8pt}$b$ & Binnendruck & $[b] = \mathrm{\frac{m^3}{mol}}$ \\
	\rule{0pt}{8pt}$R$ & Universelle Gaskonstante: $R = 8.314 \mathrm{\frac{J}{mol \cdot K}}$ & $[R] = \mathrm{\frac{J}{mol \cdot K}} $ \\
	$T$ & \textbf{Absolut-}Temperatur (in K) & $[T] = \mathrm{K}$ \\
\end{tabular}






\subsubsection{Van der Waals-Parameter}

$$ \boxed{ a = \frac{9}{8} \cdot R \cdot T_k \cdot V_{mk} = \frac{27 R^2 T_k^2}{64 \cdot p_k}}  \qquad \qquad  \boxed{ b = \frac{V_{mk}}{3} = \frac{R T_k}{8 \cdot p_k}}$$


$$ \boxed{ V_{mk} = 3 \cdot b } \quad \quad  \boxed{ T_k = \frac{8 \cdot a}{27 \cdot R \cdot b} } \quad \quad  \boxed{ p_k = \frac{a}{27 \cdot b^2} } $$



\begin{tabular}{c l c}
	\rule{0pt}{8pt}$a$ & Eigenvolumen & $[a] = \mathrm{\frac{J \cdot m^3}{mol^2}}$ \\
	\rule{0pt}{8pt}$R$ & Universelle Gaskonstante: $R = 8.314 \mathrm{\frac{J}{mol \cdot K}}$ & $[R] = \mathrm{\frac{J}{mol \cdot K}} $ \\
	$T_k$ & Kritische \textbf{Absolut-} Temperatur & $[T_k] = \mathrm{K}$ \\
	\rule{0pt}{8pt}$V_{mk}$ & Kritisches Mol-Volumen & $[V_{mk}] = \mathrm{\frac{m^3}{mol}}$ \\
	\rule{0pt}{8pt}$b$ & Binnendruck & $[b] = \mathrm{\frac{m^3}{mol}}$ \\
	$p_k$ & Kritischer Druck & $[p_k] = \mathrm{Pa}$ \\
\end{tabular}



\section{Wärmelehre}

\subsection{Wärme Q}
Wärme ist Energie, welche stets \textbf{(von allein)} von höherer zu \\
niederigerer Temperatur fliesst \\
\\

\begin{tabular}{l c l}
 & $\underleftarrow{1. HS \; 100\%}$ & \\
$\Delta U$ & $=$ & $\Delta W + \Delta Q$ \\
&  $\overrightarrow{2. HS  \; \xout{100\% }}$ & \\
\end{tabular}

\vfill\null
\columnbreak

\subsection{Erster Hauptsatz der Wärmelehre}
Nicht nur durch Wärmezufuhr, sondern auch durch mechanische \\
Arbeit lässt sich die Temperatur und damit die innere Energie $U$ \\
erhöhen


$$ \boxed{ \Delta U =\Delta W + \Delta Q } $$

\begin{tabular}{c l c}
	$\Delta U$ & Zu-/Abgeführte Innere Energie & $[\Delta U] = \mathrm{J}$ \\
	$\Delta W$ & Zu-/Abgeführte Arbeit & $[\Delta W] = \mathrm{J}$ \\
	& z.B. $\mathrm{E_{kin}, \; E_{pot}, \, W_{Gas}, \; W_{reib} }$ \\
	$\Delta Q$ & Zu-/Abgeführte Wärme & $[\Delta Q] = \mathrm{J}$ \\
\end{tabular}


\subsubsection{Ansätze für 1. HS}

$$ \Delta Q = E_{kin} = \frac{1}{2} \, m \cdot v^2 $$

$$ \Delta Q = E_{pot} = m \cdot g \cdot h $$

$$ \Delta \dot{Q} = \Delta P $$




\subsubsection{Mechanische Arbeit eines Gases}
Für mehr Details, siehe Abschnitt ~\ref{MechArbeit}

$$  \boxed{\Delta W = p \cdot \Delta V } $$





\subsection{Mechanische Wärmeäquivalente}

1 Kalorie = $4,1868 \, \mathrm{J \; (cal)}$ \\
 \quad $\Rightarrow$ Energie, um 1 Gramm Wasser um 1 Grad zu erwärmen \\
 \\
1 kcal = $4186,8 \, \mathrm{J} $ \\
 \quad $\Rightarrow$ Energie, um 1 Kilogramm Wasser um 1 Grad zu erwärmen \\




\subsubsection{Elektrisches Wärmeäquivalent $c$}
\textbf{Elektrische Energie = Wärme}

$$ U \cdot I \cdot t = c \cdot m \cdot \Delta T \quad \Leftrightarrow \quad c = \frac{U \cdot I \cdot t}{m \cdot \Delta T}$$

\begin{tabular}{c l c}
	\rule{0pt}{8pt}$c$ & Elektrisches Wärmeäquivalent & $[c] = \mathrm{\frac{J}{kg \cdot K}}$ \\
	$U$ & Spannung & $[U] = \mathrm{V}$ \\
	$I$ & Strom & $[I] = \mathrm{A}$ \\
	$t$ & Zeit & $[t] = \mathrm{s}$ \\
	$m$ & Masse & $[m] = \mathrm{kg}$ \\
	$\Delta T$ & Temperaturänderung & $[\Delta T] = \mathrm{K}$ \\
\end{tabular}


% \vfill\null
% \columnbreak

\subsection{Wärmekapazität}
Die Wärmekapazität drückt das Energiespeicher-Vermögen aus.

$$ \boxed{ Q = c \cdot m \cdot \Delta T = n \cdot c_M \cdot \Delta T = C \cdot \Delta T  }$$


\subsubsection{Absolute Wärmekapazität $C$}
Energiespeicher-Vermögen eines \textbf{Gegenstands}

$$ \boxed{ \Delta Q = C \cdot \Delta T } $$


\subsubsection{Spezifische Wärmekapazität $c$}
Energiespeicher-Vermögen einer \textbf{Substanz}

\begin{minipage}{0.4\linewidth}
	$$ \boxed{ \Delta Q = c \cdot m \cdot \Delta T  } $$	
\end{minipage}
\hfill
\begin{minipage}{0.5\linewidth}
	\begin{center}
		\begin{tabular}{lc}
			\textbf{Substanz} & \textbf{c bei 20°C}\\ \hline
			Wasser      & 4182 \\
			Ethanol     & 2430 \\
			Glyzerin    & 2390 \\
			Quecksilber & 139 \\
			Gold        & 129 \\
			Stahl       & 480 \\
		\end{tabular}
	\end{center}
\end{minipage}

\subsubsection{Molare Wärmekapazität $c_M$}
Energiespeicher-Vermögen einer \textbf{Anzahl Moleküle}

$$ \boxed{ c_M = \frac{c}{n} = M \cdot c } $$


\begin{tabular}{c l c}
	$\Delta Q$ & Zu-/Abgeführte Wärme & $[\Delta Q] = \mathrm{J}$ \\
	\rule{0pt}{8pt}$c$ & spezifische Wärmekapazität & $[c] = \mathrm{\frac{J}{kg \cdot K}}$ \\
	\rule{0pt}{8pt}$c_M$ & molare Wärmekapazität & $[c_M] = \mathrm{\frac{J}{mol \cdot K}}$ \\
	\rule{0pt}{8pt}$C$ & absolute Wärmekapazität & $[C] = \mathrm{\frac{J}{K}}$ \\
	$m$ & Masse & $[m] = \mathrm{kg}$ \\
	$\Delta T$ & Temperaturänderung & $[\Delta T] = \mathrm{K}$ \\
	$n$ & Mol-Zahl & $[n] = \mathrm{mol}$ \\
	\rule{0pt}{8pt}$M$ & Mol-Masse & $[M] = \mathrm{\frac{kg}{mol}}$ \\
\end{tabular}

% \vfill\null
% \columnbreak

\subsubsection{Molare Wärmekapazität von Gasen}

$$ \boxed{ C_{mp} - C_{mV} = R }$$


\begin{tabular}{c l c}
	\rule{0pt}{8pt}$C_{mp}$ & isobare Wärme-Kapazität ($p = \const)$ & $[C_{mp}] = \mathrm{\frac{J}{mol \cdot K}}$ \\
	\rule{0pt}{8pt}$C_{mV}$ & isochore Wärme-Kapazität ($V = \const)$ & $[C_{mV}] = \mathrm{\frac{J}{mol \cdot K}}$ \\
	\rule{0pt}{8pt}$R$ & Universelle Gaskonstante $R = 8.314 \mathrm{\frac{J}{mol \cdot K}}$ & $[R] = \mathrm{\frac{J}{mol \cdot K}} $ \\
\end{tabular}




\subsubsection{Molare Wärmekapazität von Festkörpern}


$$ T > \Theta_D: \quad C_m \approx 3 \, R \approx 25 \frac{J}{mol \cdot K}  \qquad \mathrm{(Dulong-Petit)}$$

$$ T \ll \Theta_D: \quad C_m = \frac{12 \cdot \pi^4}{5}  \cdot R \cdot  \Big( \frac{T}{\Theta_D}  \Big)^3 \qquad \mathrm{(Debye)}  $$


\begin{tabular}{c l c}
	$T$ & \textbf{Absolut-}Temperatur (in K) & $[T] = K$ \\
	$\Theta_D$ & Debye-Temperatur $\Theta_D \approx 200 \, \mathrm{K}$ & $[\Theta_D] = \mathrm{K}$ \\
	\rule{0pt}{8pt}$C_m$ & molare Wärmekapazität & $[C_m] = \mathrm{\frac{J}{mol \cdot K}}$ \\
	\rule{0pt}{8pt}$R$ & Universelle Gaskonstante: $R = 8.314 \mathrm{\frac{J}{mol \cdot K}}$ & $[R] = \mathrm{\frac{J}{mol \cdot K}} $ \\
\end{tabular}






\subsection{Latente Wärme, Enthalpie (Schmelz-/ Verdampfungswärme)}



\begin{minipage}{0.5\linewidth}
\includegraphics[width=0.99\linewidth]{Bilder/latente_waerme_2}\\
\\
\end{minipage}
\hfill
\begin{minipage}{0.5\linewidth}
Beim Schmelzen und Verdampfen findet \textbf{keine} Temperaturerhöhung statt \\
Beim Gefrieren und oder \\
Kondensieren wird diese \\
versteckte Wärme wieder frei, \textbf{ohne} Abnahme der Temperatur \\
\\
\end{minipage}

\textbf{Die Schmelz-/ Verdampfungswärme ist stark druckabhängig} \\


$$ \boxed{ Q_f = q_f \cdot m } \qquad \qquad q_{f_{Wasser}} := 334 \mathrm{\frac{kJ}{kg}} $$

$$ \boxed{ Q_s = q_s \cdot m } \qquad \qquad q_{s_{Wasser}} := 2256 \mathrm{\frac{kJ}{kg} } $$



\begin{tabular}{c l c}

	$Q_f$ & Schmelz-/Erstarrungs-Wärme & $[Q_f] = \mathrm{J}$ \\
	\rule{0pt}{8pt}$q_f$ & Spezifische Schmelzwärme & $[q_f] = \mathrm{\frac{J}{kg}}$ \\
	$Q_S$ & Verdampfungs-/Kondensations-Wärme & $[Q_S] = \mathrm{J}$ \\
	\rule{0pt}{8pt}$q_s$ & Spezifische Verdampfungs-Wärme& $[q_s] = \mathrm{\frac{J}{kg}}$ \\
	$m$ & Masse & $[m] = \mathrm{kg}$ \\
\end{tabular}




% \vfill\null
% \columnbreak



\subsection{Wärmebilanz}
Wärmeaustausch zwischen verschiedenen Materialien \\

In einem abgeschlossenen System (nach aussen isoliert) muss gelten: \\
\textbf{Zugeführte Wärme = Abgeführte Wärme}

$$  \boxed{ \sum_{i=1}^n  ( \Delta Q_i + \Delta Q_{f_i} + \Delta Q_{s_i} ) = 0 } $$


\begin{tabular}{c l c}
	$\Delta Q_i$ & i-te Wärme-Menge aus & $[\Delta Q_i] = \mathrm{J}$ \\
				& Temperatur-Zu-/Abnahme & \\
	$\Delta Q_{f_i}$ & i-te Wärme-Menge aus  &  $[\Delta Q_{f_i}] = \mathrm{J}$ \\
				& Schmelz-/Erstarrungs-Vorgang  & \\
	$\Delta Q_{s_i}$ & i-te Wärme-Menge aus & $[\Delta Q_{s_i}] = \mathrm{J}$\\
					 & Verdampfungs-/Kondensations-Vorgang &  \\
	& +  zugeführte Wärme-Menge &  \\
	& - abgeführter Wärme-Menge & \\
\end{tabular}




\section{Phasen und Phasenübergänge}


\subsection{Phasen}


\begin{tabular}{ll}
$\bullet$ & \textbf{Fest} \\
		  & feste Gestalt; festes Volumen \\
$\bullet$ & \textbf{Flüssig} \\
		  & keine feste Gestalt; festes Volumen \\
$\bullet$ & \textbf{Gasförmig} \\
		  & keine feste Gesalt; kein festes Volumen \\
$\bullet$ & \textbf{Plasma} \\
		  & Bei sehr hoher Temperatur ist Materie ionisiert (Elektronengas) \\
$\bullet$ & \textbf{Mischung / Dispersion:} \\
\end{tabular}

\begin{center}
	\begin{tabular}{l|cc}
		                   & \textbf{flüssig} & \textbf{gasförmig} \\ \hline
		\textbf{fest}      & Suspension (Sol) & Aerosol (Rauch) \\
		\textbf{flüssig}   & Emulsion & Aerosol (Nebel) \\
		\textbf{gasförmig} & Schaum & - \\
	\end{tabular}
\end{center}

%\subsection{Zweiphasengebiete (pT-Diagramm}



% \vfill\null
% \columnbreak


\subsection{Dampfdruck $p_s(T)$}
\textbf{Der Dampfdruck bedeutet das Gleichgewicht der Flüssigkeit mit ihrer Dampfphase} \\

Der Dampfdruck ist das Niveau des kontanten Drucks im\\
2-Phasengebiet eines realen Gases nach van der Waals. \\
\\
Der Dampfdruck ist nur \textbf{temperaturabhängig} \\
\\
Bei Kompression oder Expansion ändert sich der Dampfdruck nicht, sondern der Anteil Flüssigkeit zu Gas muss ändern \\
\\

\includegraphics[width=0.9\linewidth]{Bilder/dampfdruck} \\
\\
\textbf{Verdunsten} $\Rightarrow$ Schnellste Teilchen treten aus Flüssigkeit aus \\
\\
\textbf{Sieden/Verdampfen} Dampfdruck = Umgebungsdruck



% \vfill\null
% \columnbreak



\subsection{Dampfdruck-Kurve (Clausius-Clapeyron)}

\textbf{Kondensieren $\Leftrightarrow$ Verdampfen}  \qquad flüssig $\Leftrightarrow$ gasförmig  \\

$$ \boxed{ \frac{d \, p_s}{d\, T} = \frac{q_s}{T \cdot  \Big( \frac{1}{\rho_g} - \frac{1}{\rho_f} \Big)  }     } $$



\subsubsection{Dampfdruck $p_s(T)$ von Wasser (Clausius-Clapeyron)}

$$ \boxed{ p_s(T) = p_{s0} \cdot e^{\frac{q_s \cdot M_W}{R} \cdot ( \frac{1}{T_0} - \frac{1}{T}) } } $$

$$ p_{s0} = 610.7 \, \mathrm{Pa} \quad T_0 = 273 \, \mathrm{K} \quad q_s = 2420 \, \mathrm{\frac{kJ}{kg}} \quad M_W = 18.02 \, \mathrm{\frac{g}{mol}}  $$



\subsection{Schmelzdruck-Kurve (Clausius-Clapeyron)}

\textbf{Erstarren $\Leftrightarrow$ Schmelzen}  \qquad fest $\Leftrightarrow$ flüssig \\

$$ \boxed{ \frac{d \, p_f}{d\, T} = \frac{q_f}{T \cdot  \Big( \frac{1}{p_f} - \frac{1}{p_s} \Big)  }     } $$




\subsection{Gasdruck-Kurve (Clausius-Clapeyron)}

\textbf{Desublimieren $\Leftrightarrow$ Sublimieren} \qquad fest $\Leftrightarrow$ gasförmig \\

$$ \boxed{ \frac{d \, p_{sub}}{d\, T} = \frac{q_s + q_f}{T \cdot  \Big( \frac{1}{\rho_g} - \frac{1}{\rho_s} \Big)  }     } $$



\begin{tabular}{c l c}
	\rule{0pt}{10pt}$q_s$ & spezifische Verdampfungs-Wärme & $[q_s] = \mathrm{\frac{J}{kg}}$ \\
	\rule{0pt}{10pt}$q_f$ & spezifische Schmelz-Wärme & $[q_f] = \mathrm{\frac{J}{kg}}$ \\
	$q_s + q_f$ & spezifische Sublimations-Wärme & \\
	$p_s$ & Dampfdruck & $[p_s] = \mathrm{Pa}$ \\
	$p_f$ & Schmelzdruck & $[p_f] = \mathrm{Pa}$ \\
	$p_g$ & Schmelzdruck & $[p_g] = \mathrm{Pa}$ \\
	\rule{0pt}{10pt}$\rho_g$ & Dichte Gas & $[\rho_g] = \mathrm{\frac{kg}{m^3}}$ \\
	\rule{0pt}{10pt}$\rho_f$ & Dichte Flüssgkeit & $[\rho_f] = \mathrm{\frac{kg}{m^3}}$ \\
	\rule{0pt}{10pt}$\rho_s$ & Dichte Festkörper & $[\rho_s] = \mathrm{\frac{kg}{m^3}}$ \\
	$T$ & Temperatur   & $[T] = K$ \\
	$M$ & Molare Masse & $[M] = \frac{kg}{mol}$ \\
	\rule{0pt}{8pt}$R$ & Universelle Gaskonstante: $R = 8.314 \mathrm{\frac{J}{mol \cdot K}}$ & $[R] = \mathrm{\frac{J}{mol \cdot K}} $ \\
\end{tabular}


% \vfill\null
% \columnbreak



\subsection{Formeln von Magnus}
Die Formeln von Magnus dienen der vereinfachten Berechnung des Dampfdrucks von Wasser = Sättigungsdruck 

\subsubsection{Dampfdruck von Wasser $p_s(\theta)$ $(\theta \geq 0 ^{\circ}C)$}


$$ \boxed{ p_s(\theta) = p_{s0} \cdot 10^{ \frac{7.5 \cdot \theta}{\theta + 237}  } } $$



\subsubsection{Schmelzdruck von Wasser $p_s(\theta)$ $ (\theta \leq 0 ^{\circ}C)$}

$$ \boxed{ p_s(\theta) = p_{s0} \cdot 10^{ \frac{9.5 \cdot \theta}{\theta + 265.5} } } $$


\subsubsection{WMO erweiterte Lösung $p_s(\theta)$ $ (-40^{\circ}C < \theta < 50^{\circ}C) $}

$$ \boxed{p_s(\theta) = p_{s0} \cdot \e ^{\left( \frac{17.62 \cdot \theta}{243.04 + \theta} \right)} }$$



\begin{tabular}{c l c}
	$p_s$ & Dampfdruck / Schmelzdruck & $[p_s] = \mathrm{Pa}$ \\
	$p_{s0}$ & Dampfdruck bei $0^{\circ}C$ \quad $p_{s0} = 610.7 \mathrm{Pa} $ & $[p_{s0}] = \mathrm{Pa}$ \\
	$\theta$ & Temperatur & $[\theta] = \text{°C}$ \\
	
\end{tabular}




\subsection{Umkehrformeln von Magnus}

\subsubsection{$\theta(p_s)$ für $p_s \geq p_{s0}$}


$$ \boxed{ \theta(p_s) = \frac{237 \cdot log \big( \frac{p_s}{6.107} \big)  }{7.5 - log \big( \frac{p_s}{6.107} \big)} } $$



\subsubsection{$\theta(p_s)$ für $p_s \leq p_{s0}$}

$$ \boxed{ \theta(p_s) = \frac{265.5 \cdot log \big( \frac{p_s}{p_{s0}} \big)  }{9.5 - log \big( \frac{p_s}{p_{s0}} \big)} } $$




\subsection{Luftfeuchtigkeit}

\subsubsection{Absolute Luftfeuchtigkeit  $f$}

$$ \boxed{ f = \frac{m_W}{V}  } $$



\subsubsection{Relative Luftfeuchtigkeit $f_r$}

$$ \boxed{ f_r = \frac{m_W}{m_S} = \frac{p_D}{p_S} = \frac{p_D}{p_S(\theta)}  } $$



\begin{tabular}{c l c}
	$f$ & Absolute Luftfeuchtigkeit & $[f] = \frac{kg}{m^3}$ \\
	$f_r$ & Relative Luftfeuchtigkeit & $[f_r] = 1$ \\
	$m_W$ & Masse Wasserdampf & $[m_W] = \mathrm{kg}$ \\
	$m_S$ & Masse Wasserdampf bei Sättigung & $[m_S] = \mathrm{kg}$ \\
	$V$ & Volumen & $[V] = \mathrm{m^3}$ \\
	$p_D$ & Partialdruck Wasserdampf & $[p_D] = \mathrm{Pa}$ \\
	$p_S$ & Dampfdruck = Sättigungsdruck Wasserdampf & $[p_s] = \mathrm{Pa}$ \\
	$\theta$ & Temperatur & $[\theta] = \text{°C}$ \\
\end{tabular}


% \vfill\null
% \columnbreak



\subsubsection{Feuchte vs. trockene Luft}

\textbf{Feuchte Luft ist leichter als trockene Luft!}

% $$ \boxed{ \rho_f < \rho_t } \qquad \mathrm{(da} \, M_W < M_L \mathrm{)}$$

$$ \boxed{ \rho_f = \rho_t + \frac{p_D}{RT}(M_W - M_L)} $$

\begin{tabular}{c l c}
	\rule{0pt}{10pt}$\rho_f$ & Dichte feuchte Luft & $[\rho_f] = \mathrm{\frac{kg}{m^3}}$ \\
	\rule{0pt}{10pt}$\rho_t$ & Dichte trockene Luft & $[\rho_t] = \mathrm{\frac{kg}{m^3}}$ \\
	\rule{0pt}{10pt}$p_D$ & Partialdruck Wasserdampf & $[p_D] = \mathrm{Pa}$ \\
	\rule{0pt}{10pt}$T$   & Temperatur & $[T] = K$ \\
	\rule{0pt}{10pt}$M_W$ & Molmasse $H_2O$ & $[M_W] = \mathrm{\frac{kg}{mol}}$ \\
	\rule{0pt}{10pt}$M_S$ & Molmasse Luft g & $[M_W] = \mathrm{\frac{g}{mol}}$ \\
	\rule{0pt}{10pt}$R$ & Universelle Gaskonstante: $R = 8.314 \mathrm{\frac{J}{mol \cdot K}}$ & $[R] = \mathrm{\frac{J}{mol \cdot K}} $ \\	
\end{tabular}



\subsection{Taupunkts-Temperatur $\theta_d$}
Temperatur, bei welcher 100\% Luftfeuchtigkeit herrscht. \\
\\
Wenn die Taupunkt-Temperatur \textbf{unterschritten} wird, dann \\
kondensiert Wasser.

$$ \boxed{ \theta_d (\theta, f_r) = \frac{237 \cdot \Big( \log(f_r) + \frac{7.5 \cdot \theta}{\theta + 237}    \Big)}{7.5 - \Big( \log(f_r) + \frac{7.5 \cdot \theta }{\theta + 237} \Big) }  }$$


$$ \boxed{ \theta_d (x) = \frac{237 \cdot x}{7.5 - x}    \qquad  \text{mit } \quad    x(\theta, f_r) = \log(f_r) + \frac{7.5 \cdot \theta}{\theta + 237}  }   $$


\begin{tabular}{c l c}
	$\theta_d$ & Taupunkts-Temperatur & $[\theta_d] = \text{°C}$ \\
	$f_r$ & relative Luftfeuchtigkeit & $[f_r] = 1$ \\
	$\theta$ & Temperatur & $[\theta] = \text{°C}$ \\
\end{tabular}




\subsection{Relative Innen-Feuchte $f_{ri}$}

$$ \boxed{ f_{ri} = \frac{p_s(\theta_a)}{p_s(\theta_i)} \cdot f_{ra}  }$$

\begin{tabular}{c l c}
	$f_{ri}$ & relative Feuchte im Inneren & $[f_{ri}] = 1$ \\
	$f_{ra}$ & relative Feuchte der Aussenluft & $[f_{ra}] = \text{1}$ \\
	$p_s(\theta_i)$ & Dampfdruck bei Innentemperatur & $[p_s(\theta_i)] = \mathrm{Pa}$ \\
	$p_s(\theta_a)$ & Dampfdruck bei Aussentemperatur &  $[p_s(\theta_a)] = \mathrm{Pa}$  \\
\end{tabular}



% \vfill\null
% \columnbreak


\section{Kinetische Gas-Theorie} %TODO: Evtl. Folie 6, Seite 5 noch einf¨ügen 

\subsection{Aequipartitionsgesetz}

\textbf{Mittlere kinetische Energie} \\
\\
Idealisierte Annahmen: \\

\begin{tabular}{ll}
1. & Moleküle = Massenpunkte \\
2. & Keine (bzw.) elastische Zusammenstösse \\
3. & Keine Kräfte zwischen den Molekülen\\
4. & Elastischer Stoss gegen Wand \\
5. & Alle Moleküle haben gleiche Geschwindigkeit \\
6. & 1/6 aller Moleküle fliegen gegen eine einzelne Wand \\
\\
\end{tabular}


\begin{minipage}{0.48\linewidth}
$$ \boxed{ \overline{E} = f \cdot \frac{k \cdot T}{2} } $$
\end{minipage}
\hfill
\begin{minipage}{0.48\linewidth}
\begin{tabular}{ll}
f = 3 & 1-atomiges Gas \\
f = 5 & 2-atomiges Gas \\
f = 6 & 3-atomiges Gas \\
\\
\end{tabular}
\end{minipage}


\begin{tabular}{c l c}
	$\overline{E}$ & Mittlere kinetische Energie & $[\overline{E}] = \mathrm{J}$ \\
	$f$ & Freiheitsgrade & $[f] = \text{1}$ \\
	\rule{0pt}{8pt}$k$ & Boltzmann-Konstante $k = 1.381 \cdot 10^{-23} \mathrm{\frac{J}{K}}$ & $[k] = \mathrm{\frac{J}{K}}$ \\
	$T$ & \textbf{Absolute} Temperatur & $[T] = \mathrm{K}$ \\	
\end{tabular}


\subsection{Geschwindigkeiten}

\subsubsection{Mittlere quadratische Geschwindigkeit  $u$}

$$ \boxed{ u = \sqrt{\frac{3 \cdot k \cdot T}{m}} = \sqrt{\frac{3 \cdot R \cdot T}{M}} }  $$


\subsubsection{Mittlere Geschwindigkeit $\overline{v}$}

$$ \boxed{ \overline{v} = \sqrt{\frac{8 \cdot k \cdot T}{\pi m}} = \sqrt{\frac{8 \cdot R \cdot T}{\pi M}} }  $$


\subsubsection{Wahrscheinlichste Geschwindigkeit $v_0$}

$$ \boxed{ v_0 = \sqrt{\frac{2 \cdot k \cdot T}{m}} = \sqrt{\frac{2 \cdot R \cdot T}{M}}  }  $$




\begin{tabular}{c l c}
	\rule{0pt}{8pt}$k$ & Boltzmann-Konstante $k = 1.381 \cdot 10^{-23} \mathrm{\frac{J}{K}}$ & $[k] = \mathrm{\frac{J}{K}}$ \\
	$T$ & \textbf{absolute} Temperatur & $[T] = \mathrm{K}$ \\
	$m$ & Masse des Teilchens & $[m] = \mathrm{kg}$ \\
	$M$ & Molmasse & $[M] = \frac{kg}{mol}$ \\
	\rule{0pt}{10pt}$R$ & Universelle Gaskonstante: $R = 8.314 \mathrm{\frac{J}{mol \cdot K}}$ & $[R] = \mathrm{\frac{J}{mol \cdot K}} $ \\	
\end{tabular}




\subsection{Maxwell-Boltzmann-Verteilung}

$$ \boxed{ f(m, \, T, \, v) = \sqrt{\frac{2 \cdot m^3}{\pi \cdot k^3 \cdot T^3 }}  \cdot v^2 \cdot \e ^{- \frac{m \cdot v^2}{2 \cdot k \cdot T}}  }  $$


\begin{tabular}{c l c}
	$m$ & Masse des Teilchens & $[m] = \mathrm{kg}$ \\
	\rule{0pt}{8pt}$k$ & Boltzmann-Konstante $k = 1.381 \cdot 10^{-23} \mathrm{\frac{J}{K}}$ & $[k] = \mathrm{\frac{J}{K}}$ \\
	$T$ & \textbf{absolute} Temperatur & $[T] = \mathrm{K}$ \\
	\rule{0pt}{8pt}$v$ & Geschwindigkeit & $[v] = \mathrm{\frac{m}{s}}$ \\
\end{tabular}





\subsection{Mittlere freie Weglänge $\overline{\lambda}$}

Gibt an, um welche Strecke sich ein Molekül im Mittel bis zum nächsten Zusammenstoss fortbewegen kann.

$$ \boxed{ \overline{\lambda} = \frac{1}{\sqrt{2}} \cdot \frac{1}{n \cdot (\pi \cdot d^2 )} }  \qquad \text{mit Wirkungsquerschnitt } \sigma = \pi \cdot d^2$$

\begin{tabular}{c l c}
	\rule{0pt}{8pt}$n$ & \textcolor{red}{Molekül-Dichte} & $[n] = \mathrm{\frac{1}{m^3}}$\\	
	$d$ & Molekül-Durchmesser & $[d] = \mathrm{m}$ \\
\end{tabular}




\subsection{Dichtefunktion}
Verteilungsfunktion der mittleren, freien Weglänge 

$$ \boxed{ f(x) = \frac{1}{\overline{\lambda}} \cdot \e^{- \frac{x}{\overline{\lambda}}}  } $$


\subsection{Transportvorgänge}

\subsubsection{Wärmeleitung}
Transport von \textbf{kinetischer Energie} (als Wärme wahrgenommen)

$$ \boxed{ j_Q = - \lambda_Q \cdot \frac{\mathrm{dT}}{\mathrm{dx}} \qquad \quad \lambda_Q = \frac{1}{6} \cdot n \cdot \overline{v} \cdot \overline{\lambda} \cdot f \cdot k }  $$




\subsubsection{Diffusion}
Transport von \textbf{Masse} 


$$ \boxed{ j_D = -D \cdot \frac{\mathrm{dn}}{\mathrm{dx}} \qquad \quad  D = \frac{1}{3} \cdot \overline{v} \cdot \overline{\lambda} }  $$



\subsubsection{Viskosität ($v << v_{therm}$)}
Transport von \textbf{Impuls} 


$$ \boxed{ \tau = - \eta \cdot \frac{\mathrm{dv}}{\mathrm{dx}} \qquad \quad  \eta = \frac{1}{3} \cdot \overline{v} \cdot \overline{\lambda} \cdot \rho }  $$


\begin{tabular}{c l c}
	\rule{0pt}{10pt}$j_Q$ & Wärmestrom & $[j_Q] = \mathrm{\frac{W}{m^2}}$ \\
	\rule{0pt}{10pt}$\lambda_Q$ & Wärmeleitfähigkeit & $[\lambda_Q] = \mathrm{\frac{W}{m \cdot K}}$ \\
	$j_D$ & Diffusionsstrom & $[j_D] = ?$ \\
	\rule{0pt}{10pt}$D$ & Diffusionskonstante & $[D] = \mathrm{\frac{m^2}{s}}$ \\
	$\tau$ & Schubspannung & $[\tau] = \mathrm{N}$ \\
	$\eta$ & Viskosität & $[\eta] = \mathrm{Pa \cdot s}$ \\
	\rule{0pt}{10pt}$n$ & \textcolor{red}{Molekül-Dichte} & $[n] = \mathrm{\frac{1}{m^3}}$\\	
	\rule{0pt}{10pt}$\overline{v}$ & Mittlere Geschwindigkeit & $[\overline{v}] = \mathrm{\frac{m}{s}}$ \\
	\rule{0pt}{10pt}$\overline{\lambda}$ & Mittlere freie Weglänge & $[\overline{\lambda}] = \mathrm{m}$ \\
	$f$ & Anzahl Freiheitsgrade & $[f] = \mathrm{1}$ \\
	\rule{0pt}{10pt}$k$ & Boltzmann-Konstante $k = 1.381 \cdot 10^{-23} \mathrm{\frac{J}{K}}$ & $[k] = \mathrm{\frac{J}{K}}$ \\
	$T$ & \textbf{absolute} Temperatur & $[T] = \mathrm{K}$ \\
	\rule{0pt}{10pt}$\rho$ & Dichte & $[\rho] = \mathrm{\frac{kg}{m^3}}$ \\
\end{tabular}


\section{Temperaturstrahlung}

\begin{tabular}{ll}
$\bullet$ & Wärmestahlung = Berührungslose Übertragung von Wärme \\
$\bullet$ & In Form von elektromagnetischen Wellen ($\lambda$ @ IR) \\
$\bullet$ & Körper absorbiert elektromagn. Strahlung und erhöht \\
		  & seine Temperatur \\ 
          & Jeder Körper mit $T > 0 \, K$ straht Wärme ab (Temp-strahlung) \\
$\bullet$ & Für jede Wellenlänge muss ein Körper gleich viel  \\
		  & Energie abstahlen, wie er zuvor aufgenommen hat!\\
\end{tabular}




\subsection{Strahlungs-Gesetze}


\subsubsection{Stefan-Boltzmann-Gesetz}

\begin{tabular}{ll}
$\bullet$ & Ideal schwarzer Körper (Hohlraum) absoliert  \\
		  & \textbf{alle Wellenlängen zu 100 \%}   \\		
$\bullet$ & Je mehr ein Körper absorbiert, desto mehr muss er \\
		  & emmitieren \textbf{(Energie-Gleichgewicht)}   \\	
		  \\
\end{tabular}

Ein schwarzer Körper (=Hohlraumstrahler) der Temperatur $T$ hat eine totale Abstrahlungs-Leistung \textbf{pro Oberfläche} $K_S$ von: 

$$ \boxed{ K_S = \sigma \cdot T^4 }  $$



\begin{tabular}{c l c}
	\rule{0pt}{10pt}$K_S$ & Schwarzkörper-Emission & $[K_S] = \mathrm{\frac{W}{m^2}}$ \\
	\rule{0pt}{10pt}$\sigma$ & Stefan-Boltzmann-Konstante  & $[\sigma] = \mathrm{\frac{W}{m^2 \cdot K^4}}$\\
	\rule{0pt}{10pt}&  $\sigma = 5.671 \cdot 10^{-8} \, \mathrm{\frac{W}{m^2 \cdot K^4}}$ &  \\ \\
	$T$ & Temperatur & $[T] = \mathrm{K}$ \\
\end{tabular}




\subsubsection{Wien'sches Verschiebungsgesetz}

Verschiebung der maximalen Wellenlänge:

$$ \boxed{ \lambda_{max} \cdot T = \const = b  }  $$

\begin{tabular}{c l c}
	$\lambda_{max}$ & Wellenlängen-Maximum (Planck) & $[\lambda_{max}] = \mathrm{m}$ \\
	$T$ & Temperatur & $[T] = \mathrm{K}$ \\
	\rule{0pt}{8pt}$b$ & Konstante: $b = 2.898 \cdot 10^{-3} \mathrm{ m \cdot K}$ & $ [b] = \mathrm{ m \cdot K}$ \\
\end{tabular}





\subsubsection{Planck'sches Gesetz der Quantenmechanik}

Ein Oszillator, welcher auf ein anderes Energieniveau  \\
(=Elektronen-Kreisbahnen nach Bohr) wechselt, setzt die \\
Energiedifferenz $\Delta E$ in ein Lichtquant (Photon) mit \\
entsprechender Frequenz $f$ um. \\
Je nach Vorzeichen von $\Delta E$ wird das Photon emmitiert  \\
oder absorbiert . \\

$$ \boxed { \Delta E = h \cdot f }  $$

\begin{tabular}{c l c}
	$\Delta E$ & spektrale Abstrahlung (Energie) & $[\Delta E] = \mathrm{J}$ \\
	$h$ & Planck'sches Wirkungsquantum & $[h] = \mathrm{J \cdot s} $\\
	&  $h = 6.628 \cdot 10^{-34} \, \mathrm{J \cdot s}$ &  \\
	\rule{0pt}{8pt}$f$ & Frequenz des Photons & $ [f] = \mathrm{\frac{1}{s} = Hz}$ \\
\end{tabular}



% \vfill\null
% \columnbreak


\subsection{Wärmetransport (an Beispiel Hauswand)}


\includegraphics[width=0.9\linewidth]{Bilder/Waermetransport} \\


\subsubsection{Wärmeleitung}

$$ \boxed{ j = - \lambda \cdot \frac{\mathrm{d} T}{\mathrm{d} x}  } $$ 


\begin{tabular}{c l c}
	\rule{0pt}{10pt}$j$ & Wärmestromdiche & $[j] = \mathrm{\frac{W}{m^2}}$ \\
	\rule{0pt}{10pt}$\lambda$ & Wärmeleitfähigkeit & $[\lambda] = \mathrm{\frac{W}{m \cdot K}} $\\
	\rule{0pt}{10pt}$\frac{\mathrm{d} T}{\mathrm{d} x}$ & Wärmeabnahme / Gradient & $ [\frac{\mathrm{d} T}{\mathrm{d} x}] = \mathrm{\frac{T}{m}}$ \\
\end{tabular}


\subsubsection{Wärmeübergang}

$$ \boxed{\mathrm{innen:} \quad  j = \alpha_i \cdot (T_i - T_{wi} )  \qquad \mathrm{mit} \, \alpha_i = 8 \mathrm{\frac{W}{m^2 \cdot K}}  }  $$ 

$$ \boxed{\mathrm{aussen:} \quad  j = \alpha_a \cdot (T_{wa} - T_a )  \qquad \mathrm{mit} \, \alpha_a = 20 \mathrm{\frac{W}{m^2 \cdot K}}  }  $$ 





\subsubsection{Wärmedurchgang}
Material + Dicke zusammengefasst

$$ \boxed{ j = k \cdot (T_i - T_a) = k \cdot \Delta T  \qquad \mathrm{mit} \, k = \frac{\lambda}{d} }  $$ 



\begin{tabular}{c l c}
	\rule{0pt}{10pt}$j$ & Wärmestromdiche & $[j] = \mathrm{\frac{W}{m^2}}$ \\
	\rule{0pt}{10pt}$\lambda$ & Wärmeleitfähigkeit & $[\lambda] = \mathrm{\frac{W}{m \cdot K}} $\\
	\rule{0pt}{10pt}$\frac{\mathrm{d} T}{\mathrm{d} x}$ & Wärmeabnahme / Gradient & $ [\frac{\mathrm{d} T}{\mathrm{d} x}] = \mathrm{\frac{T}{m}}$ \\
	\rule{0pt}{10pt}$\alpha_i$ & Wärmeübergangszahl innen & $[\alpha_i] = \mathrm{\frac{W}{m^2 \cdot K}} $\\
	\rule{0pt}{10pt}$\alpha_a$ & Wärmeübergangszahl aussen & $[\alpha_a] = \mathrm{\frac{W}{m^2 \cdot K}} $\\
	$T_{wa}$ & Temperatur Wand aussen & $[T_{wa}] = \mathrm{K}$ \\
	$T_a$ & Aussentemperatur & $[T_a] = \mathrm{K}$ \\
	$T_{wi}$ & Temperatur Wand innen & $[T_{wi}] = \mathrm{K}$ \\
	$T_i$ & Innentemperatur & $[T_i] = \mathrm{K}$ \\
	\rule{0pt}{10pt}$k$ & Wärmedurchgangszahl & $[k] = \mathrm{\frac{W}{m^2 \cdot K}}$ \\
	$d$ & Dicke der Wand & $[d] = \mathrm{m}$ \\
	\\
\end{tabular}


$$ \textcolor{orange}{  \boxed{ P =  \dot{Q} = j \cdot A } } $$ 




% \vfill\null
% \columnbreak



\subsubsection{Wärmedurchgang komplett}

Der komplette Wärmedurchgang leitet sich her durch die \textbf{Erhaltung der Wärmestrondichte $j$} und errechnet sich mit:

% $$ \boxed{\mathrm{2 \; Schichten:} \quad \frac{1}{k_{12}} = \frac{1}{k_1} + \frac{1}{k_2} }  $$  

$$ \boxed{\mathrm{n \; Schichten:} \quad \frac{1}{k_{tot}} = \frac{1}{\alpha_i} + \sum_x  \frac{1}{k_x} + \frac{1}{\alpha_a}  }  $$

$$ \boxed{\mathrm{zylindrisch:} \quad \frac{1}{k_{tot}} = r_a \Big(  \frac{1}{\alpha_i \cdot r_i} + \sum_x \frac{1}{\lambda_x} \cdot \ln \big( \frac{r_{xa}}{r_{xi}} \big)  + \frac{1}{\alpha_a \cdot r_a} \Big) }  $$




\begin{tabular}{c l c}
	\rule{0pt}{10pt}$k_x$ & Wärmedurchgangszahl x-te Schicht & $[k_x] = \mathrm{\frac{W}{m^2 \cdot K}}$ \\
	\rule{0pt}{10pt}$\alpha_i$ & Wärmeübergangszahl innen & $[\alpha_i] = \mathrm{\frac{W}{m^2 \cdot K}} $\\
	\rule{0pt}{10pt}$\alpha_a$ & Wärmeübergangszahl aussen & $[\alpha_a] = \mathrm{\frac{W}{m^2 \cdot K}} $\\
	$r_i$ & Innenradius Rohr & $[r_i] = \mathrm{m}$	 \\
	$r_a$ & Aussenradius Rohr & $[r_a] = \mathrm{m}$	 \\
	\rule{0pt}{10pt}$\lambda_x$ & Wärmeleitfähigkeit & $[\lambda] = \mathrm{\frac{W}{m \cdot K}} $\\
\end{tabular}





\subsection{Wärme-Bedarf (Heizleistung)}
Der Wärme-Bedarf (=Heizleistung) setzt sich zusammen aus \textbf{Wärmeverlust durch Wärmeleitung} und durch \textbf{Wärmeverlust durch Luftaustausch}: 

$$\mathrm{\underbrace{W"armeverlust}_{\substack{\dot{Q}}}  = \underbrace{Heizleistung}_{\substack{P}}  }   $$

$$ \boxed{ P = \dot{Q}_{tot} = \dot{Q}_W + \dot{Q}_L }   $$

\begin{minipage}{0.46\linewidth}
$$ \boxed{ \dot{Q}_W = A \cdot j = A \cdot k \cdot \Delta T  }$$ \\
\end{minipage}
\hfill
\begin{minipage}{0.46\linewidth}
$$ \boxed{ \dot{Q}_L = c_L \cdot \rho_L \cdot \dot{V} \cdot \Delta T}   $$
\\
\end{minipage}



$$ \boxed{ \mathrm{allgemein: } \quad \dot{Q}_{tot} = \sum_{i=1}^n  \big[  (A_i \cdot k_i + c_L \cdot \rho_L \cdot \dot{V} ) \cdot \Delta T \big]  }   $$


\begin{tabular}{c l c}
	\rule{0pt}{10pt}$\dot{Q}_{tot}$ & Totaler Wärmeverlust & $[\dot{Q}_{tot}] = \mathrm{\frac{J}{s} = W}$ \\
	\rule{0pt}{10pt}$\dot{Q}_W$ & Wärmeleitung & $[\dot{Q}_W] = \mathrm{\frac{J}{s} = W}$ \\
	\rule{0pt}{10pt}$\dot{Q}_L$ & Luftaustausch & $[\dot{Q}_L] = \mathrm{\frac{J}{s} = W}$ \\
	\rule{0pt}{10pt}$k_i$ & Wärmedurchgangszahl i-te Schicht & $[k_i] = \mathrm{\frac{W}{m^2 \cdot K}}$ \\
	\rule{0pt}{10pt}$\dot{V}$ & Volumenstrom & $[\dot{V}] = \mathrm{\frac{m^3}{s}}$ \\
	\rule{0pt}{10pt}$\rho_L$ & Dichte der Luft: $\rho_L = 1.2 \, \mathrm{\frac{kg}{m^3}}$ & $[\rho_L] = \mathrm{\frac{kg}{m^3}}$ \\
	\rule{0pt}{10pt}$c_L$ & Wärmekapazität Luft: $c_L = 1000 \, \mathrm{\frac{J}{kg \cdot K}}$ & $[c_L] = \mathrm{\frac{J}{kg \cdot K}}$ \\
	$A$ & Fläche der Wärmeleitung & $[A] = \mathrm{m^2}$ \\
	$\Delta T$ & Temperaturdifferenz & $[\Delta T] = \mathrm{K}$ \\
\end{tabular}



% \vfill\null
% \columnbreak



\subsection{Wärmeverlust durch Abstrahlung}

Durch Strahlung kann auch Wärme übertragen werden.

$$ \boxed{ j_{12} = c_{12} \cdot (T_1^4 - T_2^4) = \sigma \cdot \varepsilon \cdot (T_1^4 - T_2^4) }   $$



\begin{tabular}{c l c}
	\rule{0pt}{10pt}$j_{12}$ & W-Transport durch Strahlungsaustausch & $[j_{12}] = \mathrm{\frac{W}{m^2}}$ \\
	\rule{0pt}{10pt}$c_{12}$ & Strahlungsaustauschzahl & $[c_{12}] = \mathrm{\frac{W}{m^2 \cdot K^4}}$ \\
	\rule{0pt}{10pt}$\sigma$ & Stefan-Boltzmann-Konstante  & $[\sigma] = \mathrm{\frac{W}{m^2 \cdot K^4}}$\\
	\rule{0pt}{10pt}&  $\sigma = 5.671 \cdot 10^{-8} \, \mathrm{\frac{W}{m^2 \cdot K^4}}$ &  \\
	\rule{0pt}{10pt}$\varepsilon$ & Emissionsverhältnis & $[\varepsilon] = 1$ \\
\end{tabular}








\subsection{Zustandsänderungen}

$$ \mathrm{Erinnerung \; 1. \; Hauptsatz}: \quad  \Delta U = \Delta W + \Delta Q $$


\subsubsection{Isotherm}
\textbf{bei konstanter Temperatur}

\begin{minipage}{0.48\linewidth}
$$ \boxed{ W_{ab} = n \cdot R \cdot T \cdot \ln \Big( \frac{V_1}{V_2}  \Big) }  $$
\end{minipage}
\hfill
\begin{minipage}{0.48\linewidth}
$$ \boxed{ \Delta Q_{zu} = W } \quad  (\Delta U = 0) $$
\end{minipage}



\subsubsection{Isobar}
\textbf{bei konstantem Druck} \\

\begin{minipage}{0.48\linewidth}
$$ \boxed{ W_{ab} = p \cdot (V_2 -V_1)  }  $$
\end{minipage}
\hfill
\begin{minipage}{0.48\linewidth}
$$ \boxed{ \Delta Q_{zu} = n \cdot C_{mp} \cdot \Delta T } $$
\end{minipage}


\subsubsection{Isochor}
\textbf{bei konstantem Volumen}

\begin{minipage}{0.4\linewidth}
$$ \boxed{ W = 0 }  $$
\end{minipage}
\hfill
\begin{minipage}{0.58\linewidth}
$$ \boxed{ \Delta Q_{zu} = n \cdot C_{mV} \cdot \Delta T }  \quad  (\Delta U = \Delta Q) $$
\end{minipage}


\subsubsection{Adiabatisch}
\textbf{ohne Wärme-Austausch} \\


\begin{minipage}{0.48\linewidth}
$$ \boxed{ W_{ab} = n \cdot C_{mV} \cdot \Delta T }  $$
\end{minipage}
\hfill
\begin{minipage}{0.48\linewidth}
$$ \boxed{ \Delta Q = 0} $$
\end{minipage}



% \vfill\null
% \columnbreak


\section{Rückwandlung innerer Energie}

\subsection{Zweiter Hauptsatz der Wärmelehre}
Innere Energie kann \textbf{nicht zu 100 \%} in Arbeit umgesetzt werden \\
$\Rightarrow$ Carnot-Wirkungsgrad ist der theoretisch höchstmögliche. \\
\\
Wärme kann niemals \underline{von selbst} von einem kälteren Ort zu einem wärmeren Ort fliessen (Clausius)\\
\\
Es gibt keine \underline{periodisch wirkende} Maschine, die nichts anderes bewirkt als Erzeugung mechanischer Arbeit und Abkühlung eines Wärme-Reservoirs (Kelvin) \\
$\Rightarrow$ Es gibt kein Perpetuum mobile 2. Art


\subsection{Kreisprozess (reversibler Prozess)}
$$ \text{Anfangszustand} \; = \; \text{Endzustand} $$ 


\begin{minipage}{0.48\linewidth}
\textbf{Rechts}laufender Kreisprozess \\
\\
\begin{tabular}{ll}
$\bullet$ & Gibt Arbeit ab\\
$\bullet$ & \textcolor{blue}{Wärmekraftmaschine} \\
$\bullet$ & Bei hoher $T$ wird Wärme  \\
		  & aus Prozess \textbf{zu}geführt\\
$\bullet$ & Nur Bruchteil der Wärme \\
		  & in Arbeit verwandelbar \\
$\bullet$ & Obergrenze: \\
		  & Carnot-Wirkungsgrad\\

\end{tabular}
\end{minipage}
\hfill
\begin{minipage}{0.48\linewidth}
\textbf{Links}laufender Kreisprozess \\
\\
\begin{tabular}{ll}
$\bullet$ & Verbraucht Arbeit \\
$\bullet$ & \textcolor{violet}{Wärmepumpe} \\
$\bullet$ & Bei hoher $T$ wird dem \\
		  & Prozess Wärme\textbf{ab}geführt\\
$\bullet$ & Erzeugt mehrfaches an \\
		  & Wärme \\
$\bullet$ & Obergrenze: \\
		  & Inv. Carnot-Wirkungsgrad \\
\end{tabular}
\end{minipage}








\subsection{Carnot-Wirkungsgrad}


$$ \boxed{\text{\textcolor{blue}{Wärmekraftmaschine: } }  \quad n_C = \frac{W_{ab}}{Q_{zu}} = \frac{T_{hoch}-T_{tief}}{T_{hoch}} }  $$


$$ \boxed{\text{ \textcolor{violet}{Wärmepumpe:}} \quad  n_{iC} = \frac{Q_{zu}}{W_{ab}} = \frac{T_{hoch}}{T_{hoch}-T_{tief}} }  $$



\begin{tabular}{c l c}
	$n_C$ & Carnot-Wirkungsgrad & $[n_C] = \mathrm{1}$ \\
	$n_{iC}$ & Inverset Carnot-Wirkungsgrad & $[n_{iC}] = \mathrm{1}$ \\
	$T_{tief}$ & Temperatur des Warm-Reservoirs & $[T_{tief}] = \mathrm{K}$ \\
	$T_{hoch}$ & Temperatur des Kalt-Reservoirs & $[T_{hoch}] = \mathrm{K}$  \\
	$Q_{zu}$ & zugeführte Wärme & $[Q_{zu}] = \mathrm{J}$ \\ 
	$W_{ab}$ & abgeführte Energie & $[W_{ab}] = \mathrm{J}$ \\ 
\end{tabular}





% \vfill\null
% \columnbreak


\subsection{Adiabaten-Gleichung (Kreisprozess)}
Adiabate wird beschrieben im pV- / TV- / Tp-Diagramm
\\


\begin{minipage}{0.6\linewidth}
\includegraphics[width=\linewidth]{Bilder/kreisprozess}
\end{minipage}
\hfill
\begin{minipage}{0.38\linewidth}
$$ \boxed{ p \cdot V^\kappa  = \const } $$
$$ \boxed{ T \cdot V^{\kappa -1} = \const } $$
$$ \boxed{ T^\kappa \cdot p^{1-\kappa} = \const } $$
$$ \kappa = \frac{C_{mp}}{C_{mV}}$$
$$ \boxed{ C_{mp} - C_{mV} = R }$$
\\
\end{minipage}




\begin{tabular}{c l c}
	\rule{0pt}{10pt}$C_{mp}$ & Molare Wärmekapazität @ $p = \const$ & $[C_{mp}] = \mathrm{\frac{J}{mol \cdot K}}$ \\
	\rule{0pt}{10pt}$C_{mV}$ & Molare Wärmekapazität @ $V = \const$  & $[C_{mV}] = \mathrm{\frac{J}{mol \cdot K}}$ \\
	\rule{0pt}{10pt}$\kappa$ & Adiabaten-Exponent & $[\kappa] = 1$ \\
	\rule{0pt}{10pt}$R$ & Universelle Gaskonstante $R = 8.314 \mathrm{\frac{J}{mol \cdot K}}$ & $[R] = \mathrm{\frac{J}{mol \cdot K}} $ \\
\end{tabular}



\subsection{Kreisprozesse (Vorgänge)}


\begin{tabular}{lll}
isotherme Expansion & liefert Wärme & benötigt Energie \\
isotherme Kompression & benötigt Wärme & liefert Energie \\
\\
adiabatische Expansion & liefert Arbeit & ohne Wärme \\
adiabatische Kompression & benötigt Arbeit & ohne Wärme \\
\\
isochore Erwärmung & ohne Arbeit & benötigt Wärme \\
isochore Abkühlung & ohne Arbeit & liefert Wärme \\

\end{tabular}


\subsection{Beispiel Kreisprozess}
\begin{minipage}{0.48\linewidth}
\includegraphics[width=\linewidth]{Bilder/kreisprozess_2} \\
\end{minipage}
\hfill
\begin{minipage}{0.48\linewidth}
\includegraphics[width=\linewidth]{Bilder/kreisprozess_3} \\
\end{minipage}



% \vfill\null
% \columnbreak


\subsection{Entropie-Zunahme}

\subsubsection{Definition der Entropie-Zunahme}

$$ \Delta S = S_1 + S_2 = \int \frac{1}{T} \, \mathrm{d}Q  $$


\subsubsection{Boltzmann-Gleichung für Entropie-Zunahme}

$$ \boxed{ \Delta S = k \cdot \ln(W) }$$



\begin{tabular}{c l c}
	\rule{0pt}{10pt}$\Delta S$ & Entropie & $[\Delta S] = \mathrm{\frac{J}{K}}$ \\
	\rule{0pt}{10pt}$k$ & Boltzmann-Konstante $k = 1.381 \cdot 10^{-23} \mathrm{\frac{J}{K}}$ & $[k] = \mathrm{\frac{J}{K}}$ \\
	$W$ & Wahrscheinlichkeit eines Zustands & $[W] = 1$ \\
\end{tabular}


\subsubsection{Abgeschlossenes System}

\begin{tabular}{ll}
$ \Delta S \geq 0$ & Entropie kann nur zunehmen in abgeschl. System \\

$ \Delta S > 0$ & Irreversibler Prozess \\

$ \Delta S = 0$ & Reversibler Prozess \\
\end{tabular}



		\section{Molmassen wichtiger Atome}

\begin{tabular}{c | c | c }
\textbf{Symbol} & \textbf{Molekül} & \textbf{Molmasse} \\
\hline
\rule{0pt}{10pt} H & Wasserstoff & $1.008 \, \mathrm{\frac{g}{mol}}$ \\
\rule{0pt}{10pt} C & Kohlenstoff & $12.011 \, \mathrm{\frac{g}{mol}}$ \\
\rule{0pt}{10pt} N & Stickstoff & $14.007 \, \mathrm{\frac{g}{mol}}$ \\
\rule{0pt}{10pt} O & Sauerstoff & $15.999 \, \mathrm{\frac{g}{mol}}$ \\
\rule{0pt}{10pt} Al & Aluminium & $26.982 \, \mathrm{\frac{g}{mol}}$ \\
\rule{0pt}{10pt} Si & Silicium & $28.982 \, \mathrm{\frac{g}{mol}}$ \\
\end{tabular}

% \vfill\null
% \columnbreak



\section{Ansätze zu Aufgaben}

\subsection{Saugheber}
\includegraphics[width=.65\linewidth]{Bilder/saugheber.png}

\subsection{Barometer}
\begin{minipage}{0.4\linewidth}
\includegraphics[width=\linewidth]{Bilder/manometer} \\
\end{minipage}
\hfill
\begin{minipage}{0.5\linewidth}
$\boxed{ p_1 = p_0 + \underbrace{ \rho_{Fl} \cdot g \cdot h}_{\substack{p_s}} }$
 
 
\begin{tabular}{ll}
\\
$p_1$ & gemessener Druck \\
$p_0$ & Luftdruck \\
$p_s$ & Schweredruck \\
\\
\end{tabular}

$\Rightarrow$ Bernoulli \\
$\Rightarrow$ Kontinuität \\
\\
\\
\\
\\
\\
\end{minipage}




\subsection{Pitotrohr}
Prandtl'sches Staurohr; Staudruckmesser \\
Zur Messung von Strümungsgeschwindigkeiten \\
\\
\includegraphics[width=0.7\linewidth]{Bilder/pitotrohr} \\

$$ \mathrm{Bernoulli \; horizontal:} \quad \boxed{  \underbrace{p_1}_{\substack{p_L}} + \frac{1}{2} \, \rho_1 \cdot \underbrace{v_1^2}_{\substack{0}} =  \underbrace{p_2}_{\substack{p_L - \Delta p}} + \frac{1}{2} \, \underbrace{\rho_2}_{\substack{\rho_L}} \cdot v_2^2} $$

$$ 0 = - \Delta p + \frac{1}{2} \, \rho_L \cdot v_2^2 \qquad \Rightarrow \Delta p =\frac{1}{2} \, \rho_L \cdot v_2^2 $$

$$ \mathrm{Gleichsetzen: } \quad \Delta p = \rho_{Fl} \cdot g \cdot h \quad \Rightarrow \quad v_2 = \sqrt{\frac{2 \cdot \rho_{Fl} \cdot g \cdot \Delta h}{\rho_L}} $$


\subsection{Venturirohr}
$$  \boxed{ Q = A_1 \sqrt{\frac{2 \Delta p}{\rho \left( \frac{A_1^2}{A_2^2} - 1\right)}}  \quad \quad [Q] = \frac{kg}{m^3} } $$


\subsection{Pumpe}
$$ \boxed{ W = P \cdot t = F \cdot \Delta s = p \cdot A \cdot \Delta s = p \cdot \Delta V }  $$

$$ \boxed{ P =  \frac{W}{t} = \frac{p \cdot V}{t} = p  \cdot  \dot{V} }  \quad  \boxed{ F = p \cdot A } $$



% \vfill\null
% \columnbreak


\subsection{Bewegungen}

$$ \boxed{ P = F \cdot v } \quad \boxed{ E_{kin} = \frac{1}{2} \, m \cdot v^2 } $$


\subsection{U-Rohr}

\begin{minipage}{0.4\linewidth}
\includegraphics[width=\linewidth]{Bilder/u-rohr} \\
\end{minipage}
\hfill
\begin{minipage}{0.55\linewidth}
Ansatz: Druckgleichgewicht 

$$ p_1 = p_2 $$

$$ \rho_1 \cdot g \cdot h_1 = \rho_2 \cdot g \cdot h_2  $$
\end{minipage}


\subsection{Wasser mit Dampf erhitzen}

Ein Tasse mit $m_W = 200 \, \mathrm{g}$ Wasser  mit einer Temperatur von $T_K = 20 \, \text{°C}$ wird an
der Wasserdampfdüse einer Kaffeemaschine mittels Wasserdampf erhitzt. Der aus der
Kaffeemaschine ausströmende Wasserdampf ist $T_H = 96 \, \text{°C}$ heiss. Am Schluss haben sie 10 \%
mehr Wasser in der Tasse. (entspricht $m_D$)\\
Wie warm ist das Wasser nun? \\
\\
Ansatz: 1. Hauptsatz \quad $Q_{zu} = Q_{ab}$ \\
\\
$$ m_W \cdot c_W \, (T_M - T_K) = q_s \cdot m_D + m_D \cdot c_W \, (T_H - T_M) $$



\subsection{Eis in Wasser schmelzen}

In einem Gefäss beifinden sich $m_W = 1 \, \mathrm{kg}$ Wasser. Dazu wird ein Eiswürfel von $m_E = 20 \, \mathrm{g}$ gegeben. Das Eis hat eine Temperatur von $T_E = -5 \, \text{°C}$ und das Wasser hat eine Temperatur $T_W$. Die Temperatur $T_0$ steht für $0 \, \text{°C}$ bzw. 275.15 K \\
Gesucht ist die Mischtemperatur $T_M$ 

$$ \Delta Q_{ab} = \Delta Q_{zu}$$
$$ m_W \cdot c_W \cdot (T_W - T_M) = m_E \cdot c_E \cdot (T_E - T_0) + q_f \cdot m_E + c_W \cdot (T_M - T_0) $$


		\section{Optik}

\subsection{Licht}

Licht kann auf mehrere Arten beschrieben werden:


\subsubsection{Lichtstrahlen}

\begin{minipage}{0.48\linewidth}
\includegraphics[width=\linewidth]{Bilder/Wellen-Optik/lichtstrahlen}
\end{minipage}
\hfill
\begin{minipage}{0.48\linewidth}
Die \textbf{geometrische Optik} oder \textbf{Strahlenoptik} geht davon aus, dass sich das Licht im Vakuum als \textbf{geradliniger Strahl} ausbreitet.\\
\\
Mit der geometrischen Optik können die Phänomene Reflexion und Brechung erklärt werden.
\end{minipage}



\subsubsection{Lichtwellen}


\begin{minipage}{0.48\linewidth}
\includegraphics[width=\linewidth]{Bilder/Wellen-Optik/lichtwelle} \\
\\
\end{minipage}
\hfill
\begin{minipage}{0.48\linewidth}
Licht wird als elektromagnetische Welle modelliert.  \\
\\
Bild: linear polarisierte \\
Welle \\
\end{minipage}

\textbf{Lichtfarben und ihre Frequenzen / Wellenlängen} \\
\begin{minipage}{0.5\linewidth}
	\begin{tabular}{| l | c |}
		\hline
		\textbf{Farbe} & \textbf{Wellenlänge in $\mathrm{nm}$} \\
		\hline
		\color{red!50!blue!80!white}violett & 380 ... 435 \\
		\hline
		\color{blue}blau & 435 ... 465 \\
		\hline
		\color[HTML]{35c2bd}blaugrün & 465 ... 485 \\
		\hline
		\color{green!80!black}grün & 485 ... 565 \\
		\hline
		\color{yellow!75!red}gelb & 565 ... 590 \\
		\hline
		\color{orange}orange & 590 ... 630 \\
		\hline
		\color{red}rot & 630 ... 780 \\
		\hline
	\end{tabular}
\end{minipage}
	\hfill
\begin{minipage}{0.5\linewidth}
	\begin{center}
		$$ \boxed{ \lambda = \frac{c}{f} } $$

		$$ \boxed{ \frac{\Delta f}{\Delta \lambda} = -\frac{c}{\lambda^2} } $$
	\end{center}
\end{minipage}

\begin{tabular}{c l c}
	$\lambda$ & Wellenlänge & $[\lambda] = \m$ \\
	$c$ & Lichtgeschw. $c = 299'792'458 \mathrm{\frac{m}{s}}$ & $[c] = \mathrm{\frac{m}{s}}$ \\
	$f$ & Frequenz & $[f] = \mathrm{\frac{1}{s} = Hz}$ \\
\end{tabular}




\subsubsection{Lichtteilchen}

Modellvorstellung des Lichts als ein Fluss von Lichtteilchen (\textbf{Photonen}) \\


$$ \boxed{ E = h \cdot f } $$

\begin{tabular}{c l c}
	$E$ & Energie \textbf{eines} Photons & $[E] = \J$ \\
	$h$ & Planck'sche Konstante $6.626 \cdot 10^{-34} \frac{\J}{\Hz}$ & $[h] = \frac{\J}{\Hz}$ \\
	$f$ & Frequenz & $[f] = \mathrm{\frac{1}{s} = Hz}$ \\
\end{tabular}



\subsection{Lichtquellen}

\subsubsection{Thermische Strahler}

\textbf{Schwarzkörper-Modell}: Modell eines Körpers, der in alle \\
Richtungen abstrahlt (und energetisch im Gleichgewicht ist) \\
\\
Ein Schwarzkörper strahlt \textbf{alle} Lichtfarben ab. (auch die für den Mensch nicht sichtbaren.) \\
\\

\begin{minipage}{0.48\linewidth}
\textbf{Glühbirnen} \\
\\
Muss auf allen Wellenlängen angeregt werden, um schliesslich sichtbares Licht abzustrahlen. \\
$\Rightarrow$ Es wird viel Energie nicht nutzbar 'verheizt'
\end{minipage}
\hfill
\begin{minipage}{0.48\linewidth}
\textbf{LEDs} \\
\\
Können mit einer bestimmten Frequenz angeregt werden und strahlen nur gewünschtes Licht ab. \\
$\Rightarrow$ energieeffizient
\end{minipage}



\subsubsection{Lumineszenz}

\begin{minipage}{0.3\linewidth}
\includegraphics[width=\linewidth]{Bilder/Wellen-Optik/lumineszenz}
\end{minipage}
\hfill
\begin{minipage}{0.58\linewidth}
Elektronen werden angeregt und steigen in energetisch höheren Zustand. \\
Sobald die Elektronen wieder in ihren Grundzustand zurückkehren wird ein Lichtquant (Photon) abgestrahlt. \\
Die Leuchtfarbe wird durch die \\
Frequenz der Anregung bestimmt. \\
\\
\textbf{Fluoreszenz}: kein Nachleuchten \\
\textbf{Phosphoreszenz}: mit Nachleuchten \\
\end{minipage}

% \vfill\null
% \columnbreak



\subsection{Messgrössen}

\subsubsection{Radiometrie}
Physikalische Messgrössen der elekromagnetischen Strahlung


\subsubsection{Photometrie}
Radiometrische Grössen, gewichtet mit dem photometrischen Strahlungsäquivalent $K$, welches die \textbf{Empfindlichkeit des menschlichen Auges} angibt. \\
\\
\textbf{Photometrischen Strahlungsäquivalent $K$} \\
Gibt die Empfindlichkeit des menschlichen Auges wieder und ist eine empirisch genormte Kurve \\
$\Rightarrow$ Das menschliche Auge ist bei einer Wellenlänge von $555 \, \mathrm{nm}$ \\ (grüne Farbe) am empfindlichsten. (638 $\frac{lm}{W}$)




\subsection{Gegenüberstellung}

\includegraphics[width=\linewidth]{Bilder/Wellen-Optik/radiometrie_photometrie}




\subsection{Raumwinkel}

\subsubsection{Winkel in der Ebene (Radiant)}

\begin{minipage}{0.4\linewidth}
\includegraphics[width=\linewidth]{Bilder/Wellen-Optik/winkel_ebene}
\end{minipage}
\hfill
\begin{minipage}{0.48\linewidth}
Länge des Bogens auf einem Kreis mit $r = 1 \, \m$ 

$$ \boxed{ \alpha = \frac{L}{r}  } $$ 

Vollkreis: $2 \, \pi$ \qquad $[\alpha] = \mathrm{rad}$ 
\end{minipage}


\subsubsection{Winkel im Raum (Steradiant)}

\begin{minipage}{0.35\linewidth}
\includegraphics[width=\linewidth]{Bilder/Wellen-Optik/winkel_raum}
\end{minipage}
\hfill
\begin{minipage}{0.62\linewidth}
Aufgespannte Fläche, projiziert auf eine Kugel mit $r = 1 \, \m$

$$ \boxed{ \Omega = \frac{A}{r^2} = \frac{A}{1 \m^2} } $$

Kugel: $4 \, \pi$ \qquad $[\Omega] = \mathrm{sr}$ 
\end{minipage}

% \vfill\null
% \columnbreak




%\subsection{Lichtgeschwindigkeit}

%\subsubsection{Messung nach Romer}

%\subsubsection{Messung nach Fizeau}



\subsection{Reflexionsgesetz}

\begin{minipage}{0.44\linewidth}
\includegraphics[width=\linewidth]{Bilder/Wellen-Optik/reflexionsgesetz}
\end{minipage}
\hfill
\begin{minipage}{0.48\linewidth}
Einfallswinkel = Ausfallswinkel \\

$$ \boxed{ \varepsilon = \varepsilon' } $$
\end{minipage}



\subsubsection{Grenzflächen von Reflexionen}

\begin{minipage}{0.4\linewidth}
\includegraphics[width=\linewidth]{Bilder/Wellen-Optik/reflexion} \\
\\
\end{minipage}
\hfill
\begin{minipage}{0.58\linewidth}
	\begin{tabular}{ll}
		$L$ & \textbf{reelles Bild} \\
			& Bild, welches auf Schirm \\
			& abgebildet werden kann \\ 
			\\
		$L'$& \textbf{virtuelles Bild} \scriptsize(Spiegelbild von $L$)\normalsize\\
			& Bild, welches nicht auf Schirm \\
			& abgebildet werden kann \\
	\end{tabular}
\end{minipage}

\includegraphics[width=\linewidth]{Bilder/Wellen-Optik/rau-glatt.jpg}

\subsection{Reflexionen - Spezialfälle}

\begin{tabular}{ll}
Brennpunkt $F$ & Brennpunkt (Fokus) ist der Punkt, in dem parallel \\
 		       & zur optischen Achse auf einen Spiegel oder eine \\
 		       & Linse einfallende Stahlen sich schneiden \\
 		   \\
Brennweite $f$ & Abstand des Brennpunktes von der Linse bzw. \\
		       & dem Spiegel
\end{tabular}


\subsubsection{Parabolspiegel}
Parallel einfallende Strahlen werden in einem Punkt fokussiert \\


\includegraphics[width=0.9\linewidth]{Bilder/Wellen-Optik/parabolspiegel} \\

\begin{minipage}{0.48\linewidth}
$$ \boxed{ y(x) = a \, x^2 } $$ 
\end{minipage}
\hfill
\begin{minipage}{0.48\linewidth}
$$ \boxed{ f = \frac{1}{4 \, a} } $$
\end{minipage}


% \vfill\null
% \columnbreak



\subsubsection{Elliptischer Spiegel}

Konzentration von Energie in einem nicht zugänglichen Punkt \\

\begin{minipage}{0.48\linewidth}
\includegraphics[width=\linewidth]{Bilder/Wellen-Optik/elliptischer_spiegel} 
\end{minipage}
\hfill
\begin{minipage}{0.48\linewidth}
\includegraphics[width=\linewidth]{Bilder/Wellen-Optik/elliptischer_spiegel_2} 
\end{minipage}

$$ \boxed{ y(x) = \frac{x^2}{a^2} + \frac{y^2}{b^2} } $$




\subsubsection{Hyperbolischer Spiegel}

Objekt in Spiegel versetzen \\

\begin{minipage}{0.4\linewidth}
\includegraphics[width=\linewidth]{Bilder/Wellen-Optik/hyperbolischer_spiegel} 
\end{minipage}
\hfill
\begin{minipage}{0.48\linewidth}
\includegraphics[width=\linewidth]{Bilder/Wellen-Optik/hyperbolischer_spiegel_2} 
\end{minipage}

$$ \boxed{ \frac{x^2}{a^2} - \frac{y^2}{b^2} = 1} $$



\subsubsection{Sphärische Spiegel}

Paralell einfallende Stahlen werden nicht mehr in einem Punkt  \\
fokussiert (Katakaustik) \\
\\
Da die achsnahen Strahlen nach der Reflexion annähernd durch einen Punkt gehen, wird
dieser Punkt wieder Brennpunkt $F$ genannt. \\



\begin{minipage}{0.4\linewidth}
\includegraphics[width=0.8\linewidth]{Bilder/Wellen-Optik/sphaerischer_spiegel_2} 
\end{minipage}
\hfill
\begin{minipage}{0.4\linewidth}
\includegraphics[width=\linewidth]{Bilder/Wellen-Optik/sphaerischer_spiegel} 
\end{minipage}

$$ \boxed{ f = \frac{r}{2} } $$

\begin{tabular}{lll}
$f$ & Brennweite &  $[f] = \m$ \\
$r$ & Krümmungsradius des Spiegels & $[r] = \m$ \\
\end{tabular}


% \vfill\null
% \columnbreak



\subsection{Brechung}

Fällt ein Lichtstrahl auf die Grenzfläche zweier Medien, so dringt ein Teil des einfallenden Lichtes in das zweite Medium ein. \\
Die auftrentende Richtungsänderung wird als \textbf{Brechung} bezeichnet. \\
Der in das zweite Medium eindringende Strahl wird \textbf{gebrochener Strahl} genannt.




\subsubsection{Brechungsgesetz / Geschwindigkeit}

\begin{minipage}{0.48\linewidth}
\includegraphics[width=0.7\linewidth]{Bilder/Wellen-Optik/brechung} \\
$n_2 > n_1$  \\
\end{minipage}
\hfill
\begin{minipage}{0.48\linewidth}
$$ \boxed{ \frac{\sin(\varepsilon_1)}{\sin(\varepsilon_2)} = \frac{n_2}{n_1}  } $$

$$ \boxed{ v_i = \frac{c}{n_i}  } $$

Je grösser $n$, desto grösser die \\
Ablenkung und desto kleiner $\varepsilon$ \\

\end{minipage}


\begin{tabular}{c l c}
	$\varepsilon_i$ & Winkel zur Normalen & $[\varepsilon_i] =$° \\
	$n_i$ & Brechungsindex & $[n_i] = 1$ \\
	$v_i$ & Geschwindigkeit im Medium $n_i$ & $[v_i] = \mathrm{\frac{m}{s}}$ \\
	$c$ & Lichtgeschwindigkeit $c = 300 \cdot 10^6 \, \mathrm{\frac{m}{s}}$ & $[c] = \mathrm{\frac{m}{s}}$ \\
\end{tabular}

\subsubsection{Anwendung: Glasfaser}

Der Lichtstrahl bliebt in der Kernzone (Medium 1) gefangen, da diese einen grösseren Brechungsindex hat als die Mantelschicht (Medium 2) \\

\includegraphics[width=0.75\linewidth]{Bilder/Wellen-Optik/glasfaser}

\textbf{Extremfall} (\textit{Numerical Aperture} NA = $nsin\theta_{max}$):\\
\includegraphics[width=0.8\linewidth]{Bilder/Wellen-Optik/glasfaser-2.jpg}

\begin{center}
	$ \boxed{n_0sin\theta_{max} = \sqrt{n^2_{core} - n^2_clad}} $ 
\end{center}






\subsubsection{Totalreflexion}

Der Einfallswinkel $\varepsilon_1$ kann nicht grösser als 90° sein. \\
Für $\varepsilon_1 = 90$° berechnet sich $\varepsilon_2 = \varepsilon_g$ aus: 

%$$ \frac{sin(\varepsilon_1)}{sin(\varepsilon_1)} = \frac{1}{sin(\varepsilon_g)} = \frac{n_2}{n1} $$

$$ \varepsilon_g = \arcsin \Big( \frac{n_1}{n_2} \Big) $$


Für den Grenzfall von $\varepsilon_1 > 90$° wird der gesamte Stahl reflektiert. \\



\begin{minipage}{0.35\linewidth}
\includegraphics[width=\linewidth]{Bilder/Wellen-Optik/totalreflexion} 
\end{minipage}
\hfill
\begin{minipage}{0.35\linewidth}
\includegraphics[width=\linewidth]{Bilder/Wellen-Optik/totalreflexion_2} 
\end{minipage}

% \vfill\null
% \columnbreak


\subsubsection{Brechung an ebenen Grenzflächen}

Ein Lichtstrahl wird verschoben bzw. in eine beliebige Richtung geändert\\


\begin{minipage}{0.4\linewidth}
\includegraphics[width=\linewidth]{Bilder/Wellen-Optik/verschiebung_lichtstrahl} 
\end{minipage}
\hfill
\begin{minipage}{0.4\linewidth}
\includegraphics[width=\linewidth]{Bilder/Wellen-Optik/prisma} 
$$\boxed{n = \frac{\sin{\epsilon_1}}{\sin{\epsilon_2}} = \frac{\sin{\frac{\varphi + \delta}{2}}}{\sin{\frac{\varphi}{2}}}}$$
\end{minipage}



\subsubsection{Brechung an gekrümmten Flächen}

DIE Anwendung der Brechung ist eine Linse.


\subsubsection{Linsentypen}

\begin{minipage}{0.3\linewidth}
\includegraphics[height=2.0cm]{Bilder/Wellen-Optik/sammellinse} \\
\center{Sammellinse} \\
\end{minipage}
\hfill
\begin{minipage}{0.3\linewidth}
\includegraphics[height=2.0cm]{Bilder/Wellen-Optik/zerstreuungslinse}
\center{Zerstreuungslinse} \\
\end{minipage}
\hfill
\begin{minipage}{0.3\linewidth}
\includegraphics[height=2.0cm]{Bilder/Wellen-Optik/fresnellinse}
\center{Fresnellinse} \\
\end{minipage}

\vspace{0.2cm}

\textbf{Fresnellinse:} Es kann vermieden werden, dass die Linse eine übermässige Dicke aufweist. 



\subsubsection{Beispiel mit zwei Linsen}

\includegraphics[width=0.9\linewidth]{Bilder/Wellen-Optik/linsen}\\

Rechts gibt es ein kleineres Bild als links.

\subsubsection{Typische Brechungswerte}

\begin{tabular}{| l | c |}
	\hline
	\textbf{Material} & \textbf{Brechungsindex $n$ für $\lambda$ = 589 nm} \\ \hline
	Luft (Normalbed.)        & 1.000'292 \\ \hline
	Helium (Normalbed.)      & 1.000'034'911 \\ \hline
	Wasser (20°C)            & 1.33 \\ \hline
	Glycerin                 & 1.47 \\ \hline
	Quarzglas                & 1.54 \\ \hline
	Plexiglas                & 1.51 \\ \hline
	Kronglas                 & 1.52 \\ \hline
	Brillenglas (Kunststoff) & bis 1.76 \\ \hline
	Diamant                  & 2.42 \\ \hline
\end{tabular}



\subsection{Dispersion}
\label{Dispersion Optik}


Der \textbf{Brechungsindex} eines Mediums ist eine \textbf{Funktion \\
der Wellenlänge:} $n = n(\lambda)$ \\
Diese Wellenlängenabhängigkeit wird als \textbf{Dispersion} bezeichnet \\


\begin{minipage}{0.48\linewidth}
$$ \boxed{ n^2(\lambda) = 1 + \frac{A}{ \frac{1}{\lambda_0^2} -  \frac{1}{\lambda^2} }  }  $$
\end{minipage}
\hfill
\begin{minipage}{0.48\linewidth}
$$ \boxed{ n^2(f) = 1 + \frac{A'}{f_0^2 - f^2}  }  $$
\end{minipage}


\subsubsection{Abbe Zahl $V$}

Gibt an, wie stark dispersiv ein Material ist \\
Grosse Abbe-Zahl $\rightarrow$ wenig dispersives Material




\subsection{Abbildungen}

\subsubsection{Konstruktions-Anweisung}

Man benutzt zwei der drei Hauptstrahlen: \\

\begin{tabular}{l l c l}
1. & Paralleler Strahl & $\rightarrow$ & Brennpunkt \\
2. & Mittelpunkt-Strahl & $\rightarrow$ & mit gleichem Winkel zurück \\
3. & Brennpunkt-Strahl & $\rightarrow$ & Paralleler Strahl \\
\\
\end{tabular}



\begin{minipage}{0.48\linewidth}
\center{ \textbf{reelles Bild} }
\includegraphics[width= 0.9\linewidth]{Bilder/Wellen-Optik/reelles_bild} \\
\end{minipage}
\hfill
\begin{minipage}{0.48\linewidth}
\center{ \textbf{virtuelles Bild} }
\includegraphics[width= 0.9\linewidth]{Bilder/Wellen-Optik/virtuelles_bild} \\
\end{minipage}


$B$ wird als \textbf{reeller Bildpunkt} bezeichnet, wenn sich die austretenden Strahlen schneiden. \\
\\
$B$ wird als \textbf{virtueller Bildpunkt} bezeichnet, wenn sich nur die Verlängerungen der austretenden Strahlen schneiden. 


\subsubsection{Terminologie}

\includegraphics[width=0.9\linewidth]{Bilder/Wellen-Optik/terminologie} \\

Die \textbf{Öffnungsblende} oder \textbf{Aperturblende} begrenzt das in das System einfallende Lichtbündel \small (Irisblenden, Linsenfassungen)\normalsize \\
\\
Die \textbf{Feldblende} begrenzt das Bildfeld. Sie legt den Ausschnitt der Objektebene fest, der abgebildet wird. \small (Foto-, Filmkamera: Formatrahmen) \normalsize


\subsubsection{Beispiel: Abbildungen bei zwei Sammellinsen}

\begin{minipage}{0.48\linewidth}
\includegraphics[width=0.98\linewidth]{Bilder/Wellen-Optik/zwei_sammellinsen_1} 
\end{minipage}
\hfill
\begin{minipage}{0.48\linewidth}
\includegraphics[width=0.98\linewidth]{Bilder/Wellen-Optik/zwei_sammellinsen_2} 
\end{minipage}



% \vfill\null
% \columnbreak


\subsection{Abbildungsgleichungen}
\textbf{Ein Bild ist scharf dargestellt, wenn die Abbildungsgleichung erfüllt ist!} \\



\begin{minipage}{0.48\linewidth}
	\begin{center}
		\textbf{Abbildungsgleichung} \\
		$$ \boxed{ \frac{1}{b} + \frac{1}{g} = \frac{1}{f} } $$
	\end{center}
\end{minipage}
\hfill
\begin{minipage}{0.48\linewidth}
	\begin{center}
		\textbf{Newtonsche Abb.-Gleichung} \\
		$$ \boxed{ x_b \cdot x_g = f^2 } $$
	\end{center}
\end{minipage}


\includegraphics[width=\linewidth]{Bilder/Wellen-Optik/abbildungsgleichung}
\begin{minipage}{0.50\linewidth}
	\includegraphics[width=\linewidth]{Bilder/Wellen-Optik/abbildungsgleichung-1.png}
\end{minipage}
\hfill
\begin{minipage}{0.48\linewidth}
	\includegraphics[width=\linewidth]{Bilder/Wellen-Optik/abbildungsgleichung-2.png}
\end{minipage}

\textcolor{blue}{ \textbf{Hinweis:} Ein Spiegel hat eine Brennweite von $f = \infty$ \\
$\Rightarrow$ Vereinfachung der Abbildungsgleichung!}


\subsubsection{Vergrösserungsverhältnis}

\begin{minipage}{0.48\linewidth}
$$ \boxed{ V = \frac{B}{G} = \frac{b}{g} } $$
\end{minipage}
\hfill
\begin{minipage}{0.48\linewidth}
$$ \boxed{ V_{tot} = V_1 \cdot V_2 } $$ 
\end{minipage}

\vspace{0.2cm}



\begin{tabular}{c l c}
	$V$ & Vergrösserungsverhältnis & $[V] = 1$ \\
	$b$ & Bildweite & $[b] = \m$ \\
	$g$ & Gegenstandsweite & $[g] = \m$ \\
	$B$ & Bildgrösse & $[B] = \m$ \\
	$G$ & Gegenstandsgrösse & $[G] = \m$ \\
	$f$ & Brennweite & $[f] = \m$ 
\end{tabular}




\subsection{Brechkraft $D$}
Die Optiker benutzen nicht die Brennweite sondern die Brechkraft \\
in \textbf{Dioptrie}  \\
Es gilt: $1 \, \mathrm{dpt} = 1 \, \m^{-1}$

$$ \boxed{ D = \frac{1}{f} }$$

% \vfill\null
% \columnbreak



\subsection{Linsenschleifergleichung}

\begin{minipage}{0.52\linewidth}
\includegraphics[width=\linewidth]{Bilder/Wellen-Optik/Linsenschleifergleichung}
\end{minipage}
\hfill
\begin{minipage}{0.46\linewidth}
$$\boxed{D = (n-1)  \Big(  \frac{1}{R_1} - \frac{1}{R_2} \Big)  }$$

\textcolor{blue}{Vorzeichen von $R_i$ beachten!} \\
Beide haben \\
gleiches Vorzeichen, wenn Krümmungsmittelpunkte auf gleicher Seite der Linse liegen
\end{minipage}



\subsubsection{Symmetrische Linsen ($R_1 = R_2$)}

Für symmetrische Linsen gilt:

\begin{minipage}{0.48\linewidth}
$$\boxed{D = (n - 1)\frac{2}{R}}$$ \\
\end{minipage}
\hfill
\begin{minipage}{0.48\linewidth}
$$\boxed{f = \frac{1}{D} = \frac{R}{2}\frac{1}{(n - 1)}} $$ \\
\end{minipage}



\begin{tabular}{c l c}
	$D$ & Brechkraft & $[D] = \mathrm{dpt}$ \\
	$f$ & Brennweite & $[f] = \mathrm{\frac{1}{s} = Hz}$ \\
	$R_i$ & Linsenradius & $[R_i] = \m$ \\
	$n$ & Brechungsindex & $[n] = 1$ \\
\end{tabular}


\subsubsection{Kombination von zwei Linsen}

\begin{minipage}{0.48\linewidth}
	$$ \boxed{\frac{1}{f} = \frac{1}{f_1} + \frac{1}{f_2} - \frac{d}{f_1f_2}} $$
\end{minipage}
\hfill
\begin{minipage}{0.48\linewidth}
	$$ \boxed{f = \frac{f_1f_2}{f_1 + f_2 - d}} $$
\end{minipage}

\subsubsection{Kombination von zwei dünnen Linsen ohne Zwischenraum}

Die Kombination von zwei dünnen Linsen ohne Zwischenraum ist \\
wie folgt definiert:

$$ \boxed{D = D_1 + D_2} $$




\subsection{Konvexe und Konkave Linsen}

\textbf{Konvex:} nach aussen gewölbt

\textbf{Konkav:} nach innen gewölbt

\includegraphics[width=0.9\linewidth]{Bilder/Wellen-Optik/Konvex}

\qquad\qquad Sammellinsen \qquad\qquad\qquad Zerstreuungslinsen



\subsection{Aberration}
Unter dem Begriff Aberration versteht man die Abweichung\\
vom idealisierten Fall. \\

\includegraphics[width=0.9\linewidth]{Bilder/Wellen-Optik/Aberration}



\subsubsection{Astigmatismus}

\includegraphics[width=0.9\linewidth]{Bilder/Wellen-Optik/Astigmatismus}





\subsection{Polarisation}


\subsubsection{Lineare Polarisation}

$E_x$ und $E_y$ können \textbf{unterschiedliche Amplituden} haben. \\
Die \textbf{Phasen müssen gleich} sein.\\
\\
EM-Wellen können mit dem Herzsches Gitter oder mit dem \textbf{Brewster Winkel} linear polarisiert werden.

\begin{minipage}{0.48\linewidth}
\includegraphics[width=0.9\linewidth]{Bilder/Wellen-Optik/lineare_Polarisation}
\end{minipage}
\hfill
\begin{minipage}{0.48\linewidth}
$$ \boxed{  \vec{E} = \begin{pmatrix} E_x \\  E_y \\  0  \end{pmatrix}  } $$

\center{  $\vec{E}$ kann zu $\vec{0}$ werden }

\end{minipage}




\subsubsection{Brewster Winkel}

Unter dem \textbf{Brewster Winkel} wird nur \textbf{linear polarisiertes Licht} zurückgeworfen. \\
Der ins Medium 2 eindringende Strahl steht dabei \textbf{senkrecht} auf dem reflektierten Strahl \\

\begin{minipage}{0.48\linewidth}
\includegraphics[width=0.75\linewidth]{Bilder/Wellen-Optik/brewster_winkel} \\
$n_1 < n_2$ \\
\end{minipage}
\hfill
\begin{minipage}{0.48\linewidth}
$$ \boxed{  \tan(\varepsilon) = \frac{n_2}{n_1}  }$$

\end{minipage}


% \vfill\null
% \columnbreak



\subsubsection{Zirkulare Polarisation}

$x$ und $y$ Komponenten haben die \textbf{gleiche Amplitude} und eine \textbf{Phasendifferenz von 90°} \\

Positive zirkulare Polarisation $\sigma ^+$ \\
Negative zirkulare Polarisation $\sigma ^-$ \\


\begin{minipage}{0.48\linewidth}
\includegraphics[width=0.8\linewidth]{Bilder/Wellen-Optik/Zirkulare_Polarisation}
\end{minipage}
\hfill
\begin{minipage}{0.48\linewidth}
$$ \boxed{ \vec{E} = \begin{pmatrix} E_0 \cos(2 \pi ft-kz) \\ E_0 \sin(2 \pi ft -kz) \\ 0 \end{pmatrix}  } $$

\center{ $\vec{E}$ kann nicht zu $\vec{0}$ werden }
\end{minipage}

\subsubsection{Elliptische Polarisation}

$x$ und $y$ Komponenten haben \textbf{unterschiedliche Amplituden} und eine \textbf{beliebige Phasendifferenz}.

$$ \boxed{  \vec{E} = \begin{pmatrix} E_0 \cos(2 \pi ft-kz) \\ E_0 \cos(2 \pi ft -kz + \varphi) \\ 0 \end{pmatrix}  } $$





\subsubsection{Doppelbrechung}

Doppelbrechung ist eine anisotropische Eigenschaft von Kristrallen \\
Diese Kristalle haben unterschiedliche Brechungsindizes in unterschiedliche Richtungen \\
\\
Nach einer gewissen Zeit $t$ haben die $x$- und $y$-Komponente einen Phasenunterschied. \\
$\Rightarrow$ $x$ und $y$ bewegen sich unterschiedlich schnell fort \\

\includegraphics[width=0.8\linewidth]{Bilder/Wellen-Optik/kristall_mit_doppelbrechung}






\subsection{Streuung}

\subsubsection{Diffuse Streuung}

Streuung des Lichts an Teilchen von Dimensionen $d >> \lambda$ \\ 

\begin{minipage}{0.45\linewidth}
\includegraphics[width=\linewidth]{Bilder/Wellen-Optik/diffuse_streuung}
\end{minipage}
\hfill
\begin{minipage}{0.53\linewidth}

\begin{itemize}
\item Licht als \textbf{Strahlen}
\item Reflexion in alle Richtungen
\item Keine bevorzugte Richtung, \\
	  unabhängig von $\lambda$ 
\item Beispiele:
	\begin{itemize}
		\item Wolken, Nebel 
		\item milchige Lösungen
	\end{itemize}	  
\end{itemize}

\end{minipage}




\subsubsection{Rayleigh-Streuung}

Streuung des Lichts an Teilchen von Dimensionen $d < \lambda$ \\
(atomare Grösse) \\ 

\begin{minipage}{0.38\linewidth}
\includegraphics[width=0.9\linewidth]{Bilder/Wellen-Optik/rayleigh_streuung}
\end{minipage}
\hfill
\begin{minipage}{0.58\linewidth}
$$ \boxed{ I(\theta) = A \, f^4 \, \frac{\sin^2 \theta}{r^2} } $$

$\Rightarrow$ hochfrequentes Licht wird viel stärker abgestrahlt! \\

\begin{itemize}

\item Licht als \textbf{Wellen} senkrecht 
	zur Ausbreitungsrichtung 
\item Abstrahlmuster eines Dipols \\
		  %\\
\item Himmel tagsüber blau 
\item Himmel abends rötlich	  

%\endMyItemize
\end{itemize}

\end{minipage}


\subsection{Abbildungssysteme - Auge}

\textbf{Brechzahl Augenlinse}: 1.3\\
\textbf{Tiefe}: $\sim$25mm

\subsubsection{Terminologie des Auges}

\textbf{Sehwinkel $\varepsilon$}

\begin{minipage}{0.48\linewidth}
\includegraphics[width=0.9\linewidth]{Bilder/Wellen-Optik/sehwinkel}
\end{minipage}
\hfill
\begin{minipage}{0.48\linewidth}
Die Grösse, in der ein Gegenstand dem betrachtenden Auge erscheint \textbf{(in Bogenminuten)} \\

\begin{tabular}{l c l}
\textcolor{blue}{$1$°} & \textcolor{blue}{$\Leftrightarrow$} & \textcolor{blue}{$60'$} \\
\end{tabular}

\end{minipage}



\textbf{Auflösung} 

Minimaler Winkelabstand $\varepsilon _{min}$, den zwei Punkte haben müssen, damit sie noch getrennt wahrgenommen werden. \\
Normalsichtiges Auge: Auflösung ca. 1 Bogenminute ($1'$))\\


\textbf{Sehschärfe}

Reziprokwert der Auflösung

$$ \boxed{ S = \frac{1}{\varepsilon _{min}} \qquad \text{beim Menschen also } S = 1  } $$ 



\textbf{Deutliche Sehweite  $s$ (normierte Betrachtungsdistanz)} \\

Damit die Vergrösserungen von Lupen und Mikroskopen eindeutig \\
bestimmt werden können, wird eine \textbf{deutliche Sehweite} definiert:

$$ \boxed{ s = 25 \, \mathrm{cm} = 0.25 \, \mathrm{m}  } $$ 




\subsubsection{Kurzsichtigkeit vs. Weitsichtigkeit}

\begin{minipage}{0.48\linewidth}
\textbf{Kurzsichtigkeit} 
\raggedright
\begin{itemize}

\item Augapfel zu lang
\item Konkave Streulinse als\\
	 Korrektur
\item Brille rückt Gegenstand näher heran 

\end{itemize}


\end{minipage}
\hfill
\begin{minipage}{0.48\linewidth}
\raggedright
\textbf{Weitsichtigkeit}
\begin{itemize}

\item Augapfel zu kurz
\item Konvexe Sammellinse als Korrektur
\item Brille rückt Gegenstand weiter weg 

\end{itemize}


\end{minipage}



\subsection{Abbildungssysteme - Fotoapparat}

\textbf{Bildgrösse $B [m]$} 

\begin{minipage}{0.48\linewidth}
Die Bildweite $b$ ist normalerweise viel kleiner als die
Gegenstandsweite $g$ und daraus folgt: \\
\end{minipage}
\hfill
\begin{minipage}{0.48\linewidth}
$$ \boxed{ B = \frac{f}{g} \, G }$$ \\
\end{minipage}


\textbf{Lichtstärke $H [\frac{W}{m^2}]$} 

\begin{minipage}{0.48\linewidth}
Die Intensität des Lichts auf dem Film ist gegeben durch \\
\end{minipage}
\hfill
\begin{minipage}{0.48\linewidth}
$$ \boxed{ H = \Big(  \frac{d}{f} \Big)^2 = q^2 } $$ \\
\end{minipage}




\textbf{Blendenzahl  $Z$} 

\begin{minipage}{0.48\linewidth}
z.B.: 1, 1.4, 2, 2.8, 4, 5.6, ... \\
\end{minipage}
\hfill
\begin{minipage}{0.48\linewidth}
$$ \boxed{ Z = \frac{1}{q} = \frac{1}{\sqrt{H}}  } $$ \\
\end{minipage}



\textbf{Belichtung $E$, Belichtungszeit $t$} 

\begin{minipage}{0.4\linewidth}
	$$ \boxed{ E = Ht \approx q^2t } $$
\end{minipage}
\hfill
\begin{minipage}{0.55\linewidth}
	$$ \boxed{t = \frac{G}{v} = \frac{g}{f}\frac{B}{v}} $$
\end{minipage} 


\textbf{Schärfentiefe} 

In der Filmebene ergibt sich vom Punkt $G$ kein scharfer Bildpunkt,
sondern ein \textbf{Unschärfekreis} mit dem Durchmesser $u$. \\


Es wird folgende Gegenstandsweite $g$ in den Unschärfekreis \\
abgebildet:


\begin{minipage}{0.65\linewidth}
\includegraphics[width=\linewidth]{Bilder/Wellen-Optik/schaerfentiefe}
\end{minipage}
\hfill
\begin{minipage}{0.3\linewidth}
$$\boxed{ \frac{1}{g} = \frac{1}{g_0} \pm \frac{u}{q \, f^2}  } $$
\end{minipage}

% \vfill\null
% \columnbreak


\begin{tabular}{c l c}
	$b$ & Bildweite & $[b] = \m$ \\
	$b_0$ & $\text{Bilddlistanz}_0$ & $[b_0] = \m$ \\
	$B$ & $\text{Bildgrösse}_0$ & $[B] = \m$ \\
	$B_0$ & $\text{Bild}_0$ & $[B_0] = \m$ \\
	$g$ & Gegenstandsweite ($G$ zu Objektiv) & $[g] = \m$ \\
	$g_0$ & $\text{Gegenstandsdistanz}_0$ ($G_0$ zu Objektiv) & $[g_0] = \m$ \\
	$G$ & Gegenstandsgrösse & $[G] = \m$ \\
	$G_0$ & $\text{Gegenstandsgrösse}_0$ & $[G_0] = \m$ \\
	$f$ & Brennweite & $[f] = \m$ \\
	$u$ & Durchmesser Unschärfekreis & $[d] = \m$ \\
	$d$ & Durchmesser Blendenöffnung & $[d] = \m$ \\
	$v$ & Geschw. des zu fotograf. Objekts & $[v] = \frac{m}{s}$
\end{tabular}






\subsection{Abbildungssysteme - Lupe}


\begin{minipage}{0.48\linewidth}
\includegraphics[width=0.9\linewidth]{Bilder/Wellen-Optik/ohne_lupe}

$$ \boxed{ \tan(\varepsilon_0) = \frac{G}{s} } $$
\end{minipage}
\hfill
\begin{minipage}{0.48\linewidth}
\includegraphics[width=0.9\linewidth]{Bilder/Wellen-Optik/mit_lupe}

$$ \boxed{ \tan(\varepsilon) = \frac{G}{f} } $$
\end{minipage}

$$ \boxed{ V = \frac{\tan(\varepsilon)}{\tan(\varepsilon_0)} = \frac{s}{f} } $$


\begin{tabular}{c l c}
	$\varepsilon_i$ & Sehwinkel & $[\varepsilon_i] =$° \\
	$s$ & Deutliche Sehweite $s =0.25 \, \m$ & $[s] = \m$ \\
	$f$ & Brennweite & $[f] = \m$ \\
	$V$ & Vergrösserung & $[V] = 1$ \\
\end{tabular}


\subsection{Abbildungssysteme - Fernrohr}

Es wird zuerst ein vergrössertes Bild erzeugt, welchs selber wiederum mit einer Lupe betrachtet wird. \\

\includegraphics[width=0.9\linewidth]{Bilder/Wellen-Optik/fernrohr} \\

$$ \boxed{ V = \frac{\tan(\varepsilon)}{\tan(\varepsilon')} = \frac{\frac{B}{f_2}}{\frac{B}{f_1} } = \frac{f_1}{f_2}  } $$


% \vfill\null
% \columnbreak


\subsection{Abbildungssysteme - Mikroskop}

Es wird zuerst ein vergrössertes Bild erzeugt, welchs selber wiederum mit einer Lupe betrachtet wird. \\

\includegraphics[width=0.9\linewidth]{Bilder/Wellen-Optik/mikroskop} \\

$$ \boxed{ V = \frac{\Delta \cdot s}{f_1 \cdot f_2} = \frac{\tan(\varepsilon)}{\tan(\varepsilon_0)} = \frac{\frac{B}{f_2}}{\frac{G}{s}} = \frac{B}{G} \frac{s}{f_2} = \frac{b_1}{g_1} \frac{s}{f_2} } $$


\begin{tabular}{c l c}
	$\varepsilon_i$ & Sehwinkel & $[\varepsilon_i] =$° \\
	$s$ & Deutliche Sehweite $s =0.25 \, \m$ & $[s] = \m$ \\
	$f_i$ & Brennweite & $[f_i] = \m$ \\
	$V$ & Vergrösserung & $[V] = 1$ \\
	$\Delta$ & optische Tubuslänge (Abstand der Brennpunkte) & $[\Delta] = \m$ \\
	$b_1$ & Bildweite & $[b_1] = \m$ \\
	$g_1$ & Gegenstandsweite & $[g_1] = \m$ \\
	$B$ & Bildgrösse & $[B] = \m$ \\
	$G$ & Gegenstandsgrösse & $[G] = \m$
\end{tabular}






\subsection{Farbentheorie}

\begin{tabularx}{\linewidth}{lX}
	Spektralfarben: & Die gesamten Lichtfarben \\
	Mischfarben:    & Mischung von versch. Spektralfarben \\
	Grundfarben:    & Ein Set, um alle Mischfarben zu erzeugen \\
	Komplementärfarben: & Mischfarbe, die bleibt, wenn von weissem Licht eine Farbe ausgeblendet wird \\
	Monochromatisches Licht: & Nur eine einzige Wellenlänge (Farbe) 
\end{tabularx}



\subsubsection{Farbmischungen}

\begin{minipage}{0.48\linewidth}
\textbf{Additive Farbmischung} \\

\includegraphics[width=0.9\linewidth]{Bilder/Wellen-Optik/additiveFarbmischung}

\end{minipage}
\hfill
\begin{minipage}{0.48\linewidth}
\textbf{Subtraktive Farbmischung} \\

\includegraphics[width=0.9\linewidth]{Bilder/Wellen-Optik/SubtraktiveFarbmischung}
\end{minipage}

		\section{Schwingungen}

\textbf{Freie Schwingung} \\
Wird ein schwingungsfähiges System aus dem Gleichgewichtszustand gebracht und dann sich
selbst überlassen, so führt es \textit{freie \\
Schwingungen} oder \textit{Eigenschwingungen} aus. \\


\textbf{Erzwungene Schwingung} \\
Wird ein System von aussen durch periodische oder auch nichtperiodische Störungen zum
Schwingen veranlasst, wird von \textit{fremderregten Schwingungen} gesprochen. \\



\textbf{Selbsterregte Schwingung} \\
Ein schwingungsfähiges System kann unter Umständen einer Energiequelle Energie entziehen
und diese der eigenen Schwingung selbst zuführen, so dass die Schwingung trotz einer
eventuell vorhandenen Dämpfung nicht abklingt. 






\subsection{Freie Schwingungen}

\subsubsection{Terminologie}

$$ \boxed{  y(t) = A \cdot \sin(\omega \, t + \varphi) } $$
\begin{minipage}{0.48\linewidth}
$$ \dot{y}(t) =  v(t) = A \cdot \omega \cdot \cos(\omega \, t + \varphi) $$
\end{minipage} 
\hfill
\begin{minipage}{0.48\linewidth}
$$ \ddot{y}(t) = a(t) = - A \cdot \omega^2 \cdot \sin(\omega \, t + \varphi) $$
\end{minipage}

\begin{minipage}{0.48\linewidth}
$$ \boxed{ f = \frac{\omega}{2 \, \pi}  } $$\\
\end{minipage}
\hfill
\begin{minipage}{0.48\linewidth}
$$ \boxed{ T = \frac{1}{f} = \frac{2 \, \pi}{\omega}  } $$ \\
\end{minipage}



\begin{tabular}{lll}
$y(t)$ & Position zum Zeitpunkt $t$ & $[y(t)] = \m$ \\
$\dot{y}(t)$ & Geschwindigkeit zum Zeitpunkt $t$ & $[\dot{y}(t)] = \frac{\m}{\s}$ \\ 
$\ddot{y}(t)$ & Beschleunigung  zum Zeitpunkt $t$ & $[\ddot{y}(t)] = \frac{\m}{\s^2}$ \\ 
$A$ & Amplitude & $[A] = \m$ \\
$\omega$ & Winkelgeschwindingkeit & $[\omega] \frac{\mathrm{rad}}{\s}$ \\
$\varphi$ & Phase & $[\varphi] = \mathrm{rad}$ \\ 
$T$ & Periodendauer & $[T] = \s$ \\
$f$ & Frequenz & $[f] = \frac{1}{\s}$ \\
\end{tabular}


\subsection{Beispiel - Federpendel}

\begin{minipage}{0.48\linewidth}
$$F_{res} = m \cdot a = m \cdot \ddot{x}$$ 
\end{minipage}
\hfill
\begin{minipage}{0.48\linewidth}
$$F_{Feder} = -k \cdot x$$
\end{minipage}

$$ \text{Kräftegleichgewicht: } m \cdot \ddot{x} = -k \cdot x$$


$$ \boxed{ \text{DGL: } \ddot{x} = - \omega^2 \cdot x \quad \text{mit } \omega^2 = \frac{k}{m}  }$$

$$ \text{Allgemeine Lösung: } x(t) = A \cdot \sin(\omega \, t + \varphi)$$


\begin{tabular}{lll}
$m$ & Masse & $[m] = \m$ \\
$k=c$ & Federkonstante & $[k] = \frac{\N}{\m}$
\end{tabular}





\subsubsection{Harmonische Schwingung - Energiebetrachtung}

\begin{minipage}{0.48\linewidth}
$$ \textcolor{blue}{x(t) = A \cdot \sin(\omega \, t + \varphi)} $$ 
\end{minipage}
\hfill
\begin{minipage}{0.48\linewidth}
$$  \textcolor{violet}{\dot{x}(t) = \omega \cdot A \cdot \cos(\omega \, t + \varphi)} $$ 
\end{minipage}



\begin{align*}
E_{tot} &= E_{pot} + E_{kin} = \frac{k \cdot \textcolor{blue}{x^2} }{2} + \frac{m \cdot \textcolor{violet}{\dot{x^2} } }{2}   \\
&= \frac{k}{2} A^2 \cdot \sin^2(\omega \, t + \varphi) + \frac{m}{2} \omega^2  A^2 \cdot \cos^2(\omega \, t + \varphi) \\
&= \frac{k A^2}{2} ( \sin^2(\omega \, t + \varphi) + \cos^2(\omega \, t + \varphi) )
\end{align*}


$$ \boxed{ E_{tot} = \frac{k \cdot A^2}{2}  } $$






\subsection{Beschreibung einer 1D-Schwingung}

\subsubsection{Zeitbreich}

\begin{minipage}{0.38\linewidth}
\includegraphics[width=0.9\linewidth]{Bilder/Wellen-Optik/zeitbereich}
\end{minipage}
\hfill
\begin{minipage}{0.58\linewidth}
Auslenkung in Abhängigkeit der Zeit \\
Beispiel: Oszilloskop 

$$ x(t) = A \cdot \sin(\omega \, t + \varphi) $$
\end{minipage}


\subsubsection{Zeigerdarstellung}

\begin{minipage}{0.38\linewidth}
\includegraphics[width=0.9\linewidth]{Bilder/Wellen-Optik/zeigerdarstellung}
\end{minipage}
\hfill
\begin{minipage}{0.58\linewidth}
Auslenkung als Zeiger (komplexe Zahl), der um den Ursprung rotiert 

$$ z(t) = \underbrace{ x(t) }_{\substack{\mathcal{R} (z)}} + i \, \underbrace{ y(t) }_{\substack{ \mathcal{F} (z)}} $$
\end{minipage}


\subsubsection{Phasenraum}

\begin{minipage}{0.38\linewidth}
\includegraphics[width=0.7\linewidth]{Bilder/Wellen-Optik/phasenraum}
\end{minipage}
\hfill
\begin{minipage}{0.58\linewidth}
Darstellung der Position $y$ und der \\
Ableitung (Geschwindigkeit)
\end{minipage}



\subsection{Pendel}

\subsubsection{Fadenpendel}

\begin{minipage}{0.3\linewidth}
\includegraphics[width=0.9\linewidth]{Bilder/Wellen-Optik/fadenpendel}
\end{minipage}
\hfill
\begin{minipage}{0.66\linewidth}

\center{ $ F_R = F_G \cdot \sin(\varphi) = m \cdot g \cdot \sin(\varphi) \approx m \cdot g \cdot \varphi  $ } \\

\center{ $ x = \varphi \cdot l \quad \rightarrow \quad \varphi = \frac{x}{l} $ } \\

 \center{ $ \Rightarrow F_G = m \cdot g \cdot \frac{x}{l}$ } 

$$ \text{Kräftegleichgewicht: } F = m \cdot \ddot{x} = - m \cdot g \, \frac{x}{l} $$


$$ \boxed{ \text{DGL: } \ddot{x} = - \omega^2 \cdot x \quad \text{mit } \omega^2 = \frac{g}{l}  }$$

\end{minipage}



\subsubsection{Drehpendel}

\begin{minipage}{0.4\linewidth}
\includegraphics[width=0.9\linewidth]{Bilder/Wellen-Optik/drehpendel}
\end{minipage}
\hfill
\begin{minipage}{0.56\linewidth}

Analogie ohne Rotation: 
\center{ $F = - k \cdot x \qquad F = m \cdot a = m \cdot \ddot{x} $ }\\

\center{ $ M = - c^* \, \varphi \qquad M = J \cdot \ddot{\varphi} $ } 

\center{Gleichgewicht:  $J \cdot \ddot{\varphi} = - c^* \, \varphi $ } 


$$ \boxed{ \text{DGL: } \ddot{\varphi} = - \omega^2 \cdot\varphi \quad \text{mit } \omega^2 = \frac{c^*}{J}  }$$

$\varphi$ folgt der gleichen DGL wie $x$ im Fall des Federpendels \\

\center{ $ \varphi(t) = \varphi_0 \cdot \sin(\omega \, t + \delta) $} \\
\end{minipage}

\vspace{0.2cm}


\begin{tabular}{lll}
$J$ & Trägheitsmoment & $[J] = \kg \cdot \m^2$ \\
\rule{0pt}{15pt}$c^*$ & Winkelrichtgrösse & $[c^*] = \mathrm{\frac{N \, m}{rad}}$ \\
\end{tabular}


\subsubsection{Torsionspendel}
\begin{minipage}{0.3\linewidth}
\includegraphics[width=0.5\linewidth]{Bilder/Wellen-Optik/torsionspendel} \\
\end{minipage}
\hfill
\begin{minipage}{0.62\linewidth}

\center{ Variante des Drehpendels mit der Winkelrichtgrösse \\

$$ \boxed{ c^* = \frac{\pi r^4 G}{2l} } $$  


\begin{tabular}{ll}
$G$ & Torsionsmodul \\
$l$ & Länge \\
$r = \frac{d}{2}$ & Radius \\
\end{tabular}

}
\end{minipage}





\subsubsection{Physikalisches Pendel}

\begin{minipage}{0.32\linewidth}
\includegraphics[width=0.98\linewidth]{Bilder/Wellen-Optik/physikalisches_pendel} \\
\end{minipage}
\hfill
\begin{minipage}{0.66\linewidth}

\center{ $ M = - a \cdot \sin(\varphi) \cdot F_G = -a \cdot m \cdot g \cdot  \sin(\varphi) $ }\\

\center{Bewegungsgleichung: $ M = J_A \cdot \ddot{\varphi}$ } 

\center{Gleichgewicht:  $-a \cdot m \cdot g \cdot  \sin(\varphi) =  J_A \cdot \ddot{\varphi} $ } 
\center{Kleine Winkel:  $-a \cdot m \cdot g \cdot \varphi =  J_A \cdot \ddot{\varphi} $ }


$$ \boxed{ \text{DGL: } \ddot{\varphi} = -\omega^2 \, \varphi  \quad \text{mit } \omega^2 =  \frac{g}{L^*} = \frac{g \cdot a \cdot m}{J_A}  }$$

\center{$ \boxed{L^* = \frac{J_A}{a \cdot m}} \qquad \boxed{J_A = J_s + m \cdot a^2 } $  } \\

\vspace{0.2cm}

$\varphi$ folgt der gleichen DGL wie $x$ im Fall des Federpendels \\

\end{minipage}

\textbf{Auch gültig für mehrere Massen:}
$$ \boxed{  T = \frac{2 \, \pi}{\omega} = 2 \, \pi \sqrt{\frac{J_{A1} + J_{A2}}{(a_1 \cdot m_1 + a_2 \cdot m_2) \cdot g}} } $$

$\boldsymbol{ \Rightarrow} $ \textbf{J-Tabelle im Anhang Abschnitt \ref{Massenträgheitsmomente}} \\



\begin{tabular}{lll}
$S$ &  Schwerpunkt des Körpers &  \\
$J_s$ & Trägheitsmoment bzgl. $S$ & $[J_S] = \kg \cdot \m^2$ \\
$a$ & Abstand Schwerpunkt - Drehpunkt & $[a] = \m$ \\
$L^*$ & Reduzierte Länge & $[L^*] = \m$ \\
$J_A$ & Trägheitsmoment um Aufhängepunkt & $[J_A] = \kg \cdot \m^2$ \\
$T$ & Schwingungsdauer & $[T] = s $ \\
\end{tabular}



\subsection{Perkussionszentrum}

\textbf{Frage:} Wie weit vom Drehpunkt $A$ muss ein Impuls auf einen Körper ausgeübt werden, damit keine Kraft auf die Achse ausgeübt wird? \\

\textbf{Antwort:} Auf Höhe der reduzierten Länge $L^* = \frac{J_A}{a \cdot m}$ \\
\includegraphics[width=0.8\linewidth]{Bilder/Wellen-Optik/perkussionszentrum} \\


% \vfill\null
% \columnbreak


\subsection{Periodische Schwingung}

\begin{minipage}{0.48\linewidth}
\includegraphics[width=0.8\linewidth]{Bilder/Wellen-Optik/periodische_schwingung} 
\end{minipage}
\hfill
\begin{minipage}{0.48\linewidth}

Muster wiederholt sich 

$$ \boxed{ f(t) = f(t - T ) }$$
\end{minipage}


Periodische Schwingungen können im \textbf{Frequenzbereich} in \\
eine \textbf{Grundschwingung} und \textbf{Oberschwingungen}\\
\textbf{(Harmonische)} zerlegt werden. 

$$ \boxed{ f(t) = A_0 + \sum\limits_{n=1}^{\infty}  A_n \, \sin(n \cdot \omega + \varphi_n)  } $$

\begin{tabular}{ll}
$\omega_0 = \frac{2 \, \pi}{T}$ & Grundschwingung \\
$\omega_n = n \cdot \omega_0$ & n-te Harmonische \\
\end{tabular}



\subsubsection{Fourier-Analyse}

$$ \boxed{ f(t) = A_0 +  \sum\limits_{n=1}^{\infty}  A_n \, \cos  \Big(\frac{ 2 \, \pi \, n}{T} t \Big) + \sum\limits_{n=1}^{\infty}  B_n \, \sin \Big(\frac{ 2 \, \pi \, n}{T} t \Big)  } $$

\begin{minipage}{0.6\linewidth}
\center{ $ A_n = \frac{2}{T} \int\limits_{-T/2}^{T/2} f(t) \, \cos \Big(\frac{ 2 \, \pi \, n \, t}{T}  \Big)  \, dt $}
\end{minipage}
\hfill
\begin{minipage}{0.33\linewidth}
\center{$A_0 = \frac{1}{T} \int\limits_{-T/2}^{T/2} f(t) \, dt $}
\end{minipage}


\begin{minipage}{0.6\linewidth}
\center{ $ B_n = \frac{2}{T} \int\limits_{-T/2}^{T/2} f(t) \, \sin \Big(\frac{ 2 \, \pi \, n \, t}{T}  \Big)  \, dt $}
\end{minipage}
\hfill
\begin{minipage}{0.33\linewidth}

\end{minipage}





\subsection{Signalmodulationen}

$$ \boxed{ x(t) = A \sin ( \omega t + \varphi)} $$ 

\begin{tabular}{ll}
Amplitudenmodulation (AM) & Veränderung von $ A $ \\
Frequenzmodulation (FM) & Veränderung von $ \omega $ \\
\\
\end{tabular}

\includegraphics[width=0.65\linewidth]{Bilder/Wellen-Optik/AM_FM}

% \vfill\null
% \columnbreak


\subsection{Gedämpfte Schwingungen}

\subsubsection{Gedämpfte Schwingung - Konstante Reibungskraft}

$$ m \ddot{x}(t) = -k x(t) - \mu F_N \quad \text{für } \ddot{x}(t) > 0 $$

$$ m \ddot{x}(t) = -k x(t) + \mu F_N \quad \text{für } \ddot{x}(t) < 0 $$


\includegraphics[width=0.7\linewidth]{Bilder/Wellen-Optik/gedaempfte_schwingungen_konst_reibung} 




\begin{minipage}{0.48\linewidth}
$$ \boxed{ E = \frac{k \, A^2}{2} } $$ \\
\end{minipage}
\hfill
\begin{minipage}{0.48\linewidth}
$$ \boxed{ \Delta A = 4 \frac{F_R}{k}} $$ \\
\end{minipage}


\begin{tabular}{c l c}
$E$ & Energie bei max. Auslenkung & $[E] = \J$ \\
$k=c$ & Federkonstante & $[k] = \frac{\N}{\m}$ \\
$A$ & Amplitude bei max. Auslankung & $[A] = \m$ \\
$\Delta A$ & Amplitudenänderung pro Periode & $[\Delta A] = \m$ \\
$F_R$ & Reibungskraft & $[F_R] = \N$ 
\end{tabular}



\subsubsection{Gedämpfte Schwingungen - Dämpfung proportional zur Geschwindigkeit}
$$ \boxed{\ddot{x}(t) + 2\delta \dot{x}(t) + \omega^2x(t) = 0, \quad D = \frac{\delta}{\omega}, \quad f = \frac{\Omega}{2 \, \pi}} $$
\tiny
\setlength{\tabcolsep}{3pt}
\begin{tabular}{|l|c|l|l|}\hline
        1. Fall: & $\delta^2 - \omega^2 > 0$, D $>$ 1 & $ x(t) = A \e^{\lambda_1t} + B\e^{\lambda_2t} $ & Aperiod. Schwingung \\ 
        & & & \\ \hline
        2. Fall: & $\delta^2 - \omega^2 < 0$, D $<$ 1 & $ x(t) = A \e^{-\delta t} cos(\Omega t + \varphi) $ & Period. Schwingung \\
        & & & \\\hline
        3. Fall: & $\delta^2 - \omega^2 = 0$, D $=$ 1 & $ x(t) = (A + Bt)\e^{-\delta t} $ & Grenzfall ($\lambda = -\delta$) \\
        & & & \\\hline
\end{tabular}
\normalsize


% \includegraphics[width=0.98\linewidth]{Bilder/Wellen-Optik/gedaempfte_schwingungen} \\

\begin{minipage}{0.5\linewidth}
    $$ \boxed{ \lambda_{1,2} = -\omega ( D \pm \sqrt{D^2 -1}) }$$ 
    $$ \boxed{ \Omega^2 =  \omega^2 - \delta^2 }$$ 
\end{minipage}
\hfill
\begin{minipage}{0.45\linewidth}
    $$ \boxed{ z \cdot \Lambda = \ln \Big( \frac{A_n}{A_{n+z}} \Big) = z \cdot \delta T }$$ \\
\end{minipage}

\begin{tabular}{c l c}
$\frac{\kappa}{2} = \delta$ & Abklingkonstante & $[\kappa = \delta] = \frac{1}{\s}$ \\
$\omega$ & Kreisfrequenz ungedämpfte Schwingung & $[\omega] = \frac{\rad}{\s}$ \\
$D$ & Dämpfungsgrad & $[D] = 1$ \\
$\Omega$ & Kreisfrequenz gedämpfte Schwingung & $[\Omega] = \frac{\rad}{\s}$ \\


$\Lambda$ & Logarithmisches Dekrement & $[\Lambda] = 1$ \\
$A_n$ & Amplitude zum Zeitpunkt $t$ & $[A_n] = \m$ \\
$A_{n+z}$ & Amplitude zum Zeitpunkt $t + z \cdot T$ & $[A_{n+z}] = \m$ \\
$z$ & Anzahl verstrichene Schwingungen & $[z] = 1$ \\
$T$ & Periodendauer & $[T] = \s$ \\
$f$ & Frequenz der gedämpften Schwingung & $[f] = \Hz$ 
\end{tabular}


\vfill\null
\columnbreak


\subsection{Fremderregte Schwingung}

\subsubsection{Definition}

Erzwungene Schwingungen sind Schwingungen, die durch eine \\
\textbf{periodische Störung} verursacht werden. (EN: \textit{driven oscillation})

\subsubsection{Übersicht über Hilfsgrössen}

\fbox{\parbox{0.8\linewidth}{

\begin{tabular}{c c c c c}
$\omega_0 = \sqrt{\frac{k}{m}}$ & &  $\delta = \frac{\kappa}{2} = \frac{b}{2 \, m}$ & & $ D = \frac{\delta}{\omega_0} =  \frac{\kappa}{2 \, \omega_0}$\\
\\
\end{tabular}
 
 
\begin{tabular}{c c c}
$\eta = \frac{\omega}{\omega_0}$ & & $\Omega_d = \sqrt{\omega_0^2 - \big( \frac{\kappa}{2} \big) ^2} =  \omega_0 \sqrt{1 - D^2}$\\
\end{tabular}
}}

\begin{tabular}{c l c}
\rule{0pt}{10pt} $\omega_0$ & Kreisfrequenz ungedämpfte Schwingung & $[\omega_0] = \frac{\rad}{\s}$ \\
\rule{0pt}{10pt} $\Omega_d$ & Kreisfrequenz gedämpfte Eigenfrequenz & $[\Omega_d] = \frac{\rad}{\s}$ \\
\rule{0pt}{10pt} $\omega_r$ & Resonanzkreisfrequenz & $[\omega_r] = \frac{\rad}{\s}$ \\
\rule{0pt}{10pt} $\omega$ & Kreisfrequenz der Störung (Erreger) & $[\omega] = \frac{\rad}{\s}$ \\
\rule{0pt}{10pt} $\frac{\kappa}{2} = \delta $ & Abklingkonstante & $[\kappa] = \frac{1}{\s}$ \\
\rule{0pt}{10pt} $D$ & Dämpfungsgrad & $[D] = 1$ \\
$\eta$ & Dimensionslose Frequenz & $[\eta] = 1$  \\
$k = c$ & Federkonstante & $[k] = \frac{\N}{\m}$
\end{tabular}

\subsubsection{Resonanz}

Die Amplitude $A$ wird maximal, wenn der Nenner von $A(\omega)$ minimal wird 

\begin{minipage}{0.48\linewidth}
    \begin{center}
        Resonanzkreisfrequenz:
        $$ \boxed{ \omega_r = \omega_0 \sqrt{1 - 2 \, D^2} } $$
    \end{center}
\end{minipage}
\hfill
\begin{minipage}{0.48\linewidth}
    \begin{center}
        Resonanzamplitude:
        $$ \boxed{ A_r = \frac{u_0}{2 \, D \sqrt{1 - D^2}} }$$
    \end{center}    
\end{minipage}



\subsection{Fremderregte Schwingungen - Krafterregung}
\includegraphics[width=0.75\linewidth]{Bilder/Wellen-Optik/krafterregung} 

$$ \boxed{ \text{DGL: } m \, \ddot{y} + b \, \dot{y} + c \, y = F_0 \, \sin(\omega \, t)  } $$  


$$ y(t) = \underbrace{ A(\omega) \, \sin(\omega \, t - \varphi) }_{\substack{y_p(t)}} + \underbrace{ B \, e^{- \delta t} \sin(\Omega_d \, t + \varphi_0)}_{\substack{y_h(t)}} $$

\begin{minipage}{0.3\linewidth}
$$ \boxed{ \varphi = \arctan \Big( \frac{2 D \omega_0 \omega}{\omega_0^2 - \omega^2}   \Big) } $$ \\
\end{minipage}
\hfill
\begin{minipage}{0.68\linewidth}
$$ \boxed{ A(\omega) = \frac{F_0}{m \sqrt{(\omega_0^2- \omega^2)^2 + (2 D \omega_0 \omega)^2 } } }$$ \\
\end{minipage}


\begin{tabular}{c l c}
$A(\omega)$ & Amplitudenverlauf & $[A(\omega)] = \m$ \\
$\varphi$ & Phasenverschiebung & $[\varphi] = \rad$
\end{tabular}


% \vfill\null
% \columnbreak


\subsubsection{Vergrösserungsfunktion / Phasenverschiebung}

\begin{tabular}{c c}
$ \boxed{ V = \frac{1}{\sqrt{(1- \eta^2)^2 + (2 \, D \, \eta)^2} } } $ & $ \boxed{ \varphi = \arctan \Big( \frac{2 D \eta}{1 - \eta^2}  \Big) } $ \\
\\
$ \boxed{ V_r = \frac{1}{\sqrt{ 1 - \eta_r^4} } \quad \text{mit } \eta_r = \sqrt{1 - 2 \, D^2}  } $ \\
\\
\end{tabular}


\begin{tabular}{c l c}
$\varphi$ & Phasenverschiebung & $[ \varphi] = \rad$ \\
$V$ & Vergrösserungsfunktion & $[V] = 1$ \\
$V_r$ & Vergrösserungsfunktion & $[V_r] = 1$ \\
$\eta$ & Dimensionslose Frequenz & $[\eta] = 1$  \\
$D$ & Dämpfungsgrad & $[D] = 1$ 
\end{tabular}




\subsection{Fremderregte Schwingungen - Dämpfererregung}

\begin{minipage}{0.25\linewidth}
\includegraphics[width=0.9\linewidth]{Bilder/Wellen-Optik/daempfererregung} \\
\\
\end{minipage}
\hfill
\begin{minipage}{0.72\linewidth}
$$ \boxed{ \text{DGL: } m \, \ddot{y} + b \, \dot{y} + c \, y = b \, \omega  \, u_0 \, \cos(\omega \, t)  } $$  

$$ A(\omega) =  \frac{b \, \omega \, u_0}{m\sqrt{(\omega_0^2 -\omega^2)^2 + (2D \omega_0 \omega)^2}} $$ 

$$ \boxed{ V = \frac{2 \, D \, \eta}{\sqrt{(1- \eta^2)^2 + (2 \, D \, \eta)^2} } } $$

$$ \boxed{ \varphi = \arctan \Big( \frac{2 D \eta}{1 - \eta^2} \Big) - \frac{\pi}{2} } $$ 


\end{minipage}


\begin{tabular}{c l c}
$A(\omega)$ & Amplitude der Schwingung & $[A(\omega)] = \m$ \\
$\varphi$ & Phasenverschiebung & $[ \varphi] = \rad$ \\
$V$ & Vergrösserungsfunktion & $[V] = 1$ \\
$\eta$ & Dimensionslose Frequenz & $[\eta] = 1$  \\
$D$ & Dämpfungsgrad & $[D] = 1$ \\
\end{tabular}

% \vfill\null
% \columnbreak



\subsection{Fremderregte Schwingungen - Stützenerregung}

$$ \boxed{ \text{DGL: } m \, \ddot{q} + b \, \dot{q} + c \, q = -m \, \omega^2 \, u_0 \, \sin(\omega \, t)  \quad \text{mit } q = y - u } $$

\begin{minipage}{0.25\linewidth}
\includegraphics[width=0.9\linewidth]{Bilder/Wellen-Optik/stuetzenerregung} 
\end{minipage}
\hfill
\begin{minipage}{0.72\linewidth}
$$ A(\omega) = \frac{\omega^2 u_0}{\sqrt{(\omega_0^2 -\omega^2)^2 + (2D \omega_0 \omega)^2}} $$ 
\end{minipage}

\begin{minipage}{0.48\linewidth}
$$ \boxed{ V = \frac{\eta^2}{\sqrt{(1- \eta^2)^2 + (2 \, D \, \eta)^2} } } $$
\end{minipage}
\hfill
\begin{minipage}{0.48\linewidth}
$$ \boxed{ \varphi = \arctan \Big( \frac{2 D \eta}{1 - \eta^2} \Big) - \pi } $$ 
\end{minipage}

\vspace{0.2cm}

\begin{tabular}{c l c}
$\varphi$ & Phasenverschiebung & $[ \varphi] = \rad$ \\
$V$ & Vergrösserungsfunktion & $[V] = 1$ \\
$\eta$ & Dimensionslose Frequenz & $[\eta] = 1$  \\
$D$ & Dämpfungsgrad & $[D] = 1$  \\
\rule{0pt}{10pt} $\omega_0$ & Kreisfrequenz ungedämpfte Schwingung & $[\omega_0] = \frac{\rad}{\s}$ \\
\rule{0pt}{10pt} $\omega$ & Kreisfrequenz der Störung (Erreger) & $[\omega] = \frac{\rad}{\s}$ \\
$A(\omega)$ & Amplitude der Schwingung & $[A(\omega)] = \m$
\end{tabular}



\subsection{Fremderregte Schwingung - Unwuchterregung}
\textbf{Unwucht:} Schwerpunkt $S$ des Rotors der Masse $m_R$ bewegt sich auf einem Kreis mit Radius $e$ \\

\begin{minipage}{0.52\linewidth}
\includegraphics[width=0.9\linewidth]{Bilder/Wellen-Optik/unwucht} \\
\end{minipage}
\hfill
\begin{minipage}{0.44\linewidth}
\includegraphics[width=0.89\linewidth]{Bilder/Wellen-Optik/unwucht_gehause} \\
\end{minipage}

$$ \boxed{ \text{DGL in y-Richtung: } m \ddot{y} + b \dot{y} + c \, y = -m_R \, \omega^2 \, e \sin(\omega \, t) }  $$

Radiale Beschleunigung des Schwerpunkts des Rotors: $a_R = \omega^2 e$ \\

Kraft des Rotors auf die Maschine: $\boxed{ F_U = m_R \cdot a_R = m_R \cdot \omega^2 e } $ 

$$ A(\omega) = \frac{m_R}{m} \frac{e \, \omega^2}{\sqrt{(\omega_0^2 -\omega^2)^2 + (2D \omega_0 \omega)^2}}  $$ 


$$ A_R = \frac{m_R}{m} \frac{e}{2D \sqrt{1 - D^2}}  $$


% \vfill\null
% \columnbreak


\subsubsection{Kraft auf die Basis des Gehäuses} 

\begin{minipage}{0.48\linewidth}
$$ F_B = c y + b \dot{y} = F_{B0} \sin(\omega t - \varphi + \psi) $$ 
\end{minipage}
\hfill
\begin{minipage}{0.48\linewidth}
$$\boxed{ F_{B0} = \frac{m_R \, e \, \omega^2 \sqrt{1+ (2D \eta)^2}}{\sqrt{(1 - \eta^2)^2 + (2 D \eta)^2}} } $$ 
\end{minipage}
\vspace{0.2cm}


\begin{tabular}{c l c}
$m_R$ & Masse des Rotors & $[m_R] = \kg$ \\
$a_R$ & Radiale Beschleinigung Schwerpunkt & $[a_R] = \frac{\m}{\s}$ \\
$e$ & Abstand Mittelpunkt - Schwerpunkt & $[e] = \m$ \\
$\varphi$ & Phasenverschiebung & $[ \varphi] = \rad$ \\
$\eta$ & Dimensionslose Frequenz & $[\eta] = 1$  \\
$D$ & Dämpfungsgrad & $[D] = 1$  \\
$A(\omega)$ & Amplitude & $[A(\omega)] = \m$ \\
$A_R$ & Resonanzamplitude & $[A_R] = \m$
\end{tabular}




\subsection{Fremderregte Schwingung - Schwingkreis}

\begin{minipage}{0.3\linewidth}
\includegraphics[width=\linewidth]{Bilder/Wellen-Optik/schwingkreis}  \\
\end{minipage}
\hfill
\begin{minipage}{0.68\linewidth}
$$ \boxed{ \text{DGL: } L \ddot{I} + R \dot{I} + \frac{1}{C} \, I = \omega \, U_0 \,\sin(\omega \, t + \frac{\pi}{2}) } $$

$\Rightarrow$ Gleiche DGL wie bei \textcolor{blue}{Dämpfererregung} \\
\end{minipage}


$ \textcolor{blue}{\text{DGL: } m \ddot{y} + b \dot{y} + c \, y = b \, \omega \, u_0 \, \cos(\omega \, t) }, \textcolor{blue}{A(\omega) =  \frac{b \, \omega \, u_0}{m\sqrt{(\omega_0^2 -\omega^2)^2 + (2D \omega_0 \omega)^2}} } $ \\

\vspace{0.1cm}


$ \boxed{ I(\omega) =  \frac{\omega}{L\sqrt{(\omega_0^2 -\omega^2)^2 + (2D \omega_0 \omega)^2}}U_0, \, \omega_0 = \frac{1}{\sqrt{LC}} , \, D = \frac{R}{2}\sqrt{\frac{C}{L}}} $ 

\vspace{0.1cm}

$$ \boxed{ V = \frac{U_{L0}}{U_0} = \frac{\eta^2}{\sqrt{(1 - \eta^2)^2 + (2 \, D \, \eta)^2}} }$$



\subsubsection{Resonanz}

\begin{minipage}{0.48\linewidth}
\textbf{Resonanzfrequenz}
$$ \boxed{ \omega_r = \omega_0 = \frac{1}{\sqrt{LC}} } $$ 
\end{minipage}
\hfill
\begin{minipage}{0.48\linewidth}
\textbf{Amplitude @ Resonanz}
$$ \boxed{ I_{0r} = \frac{U_0}{R} } $$ 
\end{minipage}


\vspace{0.2cm}


\begin{tabular}{c l c}
\rule{0pt}{10pt} $\omega_0$ & Kreisfrequenz ungedämpfte Schwingung & $[\omega_0] = \frac{\rad}{\s}$ \\
\rule{0pt}{10pt} $\omega$ & Kreisfrequenz der Störung (Erreger) & $[\omega] = \frac{\rad}{\s}$ \\
\rule{0pt}{10pt} $\omega_r$ & Resonanzfrequenz & $[\omega_r] = \frac{\rad}{\s}$ \\
$I_{0r}$ & Strom-Amplitude @ Resonanz & $[I_{0r}] = \A$ \\
$U_0$ & Amplitude der Erregerspannung & $[U_0] = \V$ \\
$U_{L0}$ & Amplitude Spulenspannung & $[U_{L0}] = \V$ \\
$A(\omega)$ & Amplitude der Schwingung & $[A(\omega)] = \m$ \\
$V$ & Vergrösserungsfunktion & $[V] = 1$ \\
$\eta$ & Dimensionslose Frequenz & $[\eta] = 1$  \\
$D$ & Dämpfungsgrad & $[D] = 1$ 
\end{tabular}





\subsection{Fremderregte Schwingung - Güte $Q$}
\subsubsection{Definition}

Die relative Abnahme der Schwingungsenergie $E(t)$ pro Schwingdauer wird als \textbf{Güte} oder \textbf{Gütefaktor} bezeichnet 

$$ \boxed{ Q = 2 \pi \frac{E(t)}{E(t) - E(t + T)} }$$ 



\subsubsection{Beziehungen}

\begin{minipage}{0.4\linewidth}
\includegraphics[width=0.95\linewidth]{Bilder/Wellen-Optik/guete_resonanz} 
\end{minipage}
\hfill
\begin{minipage}{0.58\linewidth}

\begin{minipage}{0.48\linewidth}
$$ \boxed{Q = \frac{1}{2 \, D} } $$ 
\end{minipage}
\hfill
\begin{minipage}{0.48\linewidth}
$$ \boxed{ Q = \frac{\omega_0}{\Delta \omega} } $$
\end{minipage}

\vspace{0.2cm}

Breite der Resonanzkurve bei $U_0 = \frac{U_{0r}}{\sqrt{2}}$ \\


\textbf{breite Kurve $\Rightarrow$ tiefe Güte}
\end{minipage}



\subsection{Gekoppelte Pendel}
Zwei Pendel sind durch eine Feder miteinander verbunden. \\
\textbf{Die Bewegung eines Pendels hat Auswirkungen auf die \\
Bewegung des anderen Pendels.} \\
Gesucht ist eine Beschreibung der Bewegung des Pendels. \\

\begin{minipage}{0.3\linewidth}
    \includegraphics[width=0.95\linewidth]{Bilder/Wellen-Optik/gekoppelte_pendel} 
\end{minipage}
\hfill
\begin{minipage}{0.66\linewidth}
    $ J_1 \ddot{\varphi_1} = -m_1 \, g \, a_1 \, \varphi_1 + c \cdot h2 \, (\varphi_2 - \varphi_1)$ \\
    $ J_2 \ddot{\varphi_2} = -m_2 \, g \,  a_2 \, \varphi_2 - c \cdot h^2 \, (\varphi_2 - \varphi_1)$ \\

    \vspace{0.2cm}

    \textbf{Spezialfall $J_1 = J_2 = J$ und $m_1 = m_2 = m$ und $a_1 = a_2 = a$} \\

    $ \ddot{\varphi_1} = - \omega^2 \varphi_1 + k \, (\varphi_2 - \varphi_1)$ \\
    $ \ddot{\varphi_2} = - \omega^2 \varphi_2 - k \, (\varphi_2 - \varphi_1)$ \\

    \vspace{0.2cm}
    mit $ \omega^2 = \frac{mga}{J}$ \quad $k = \frac{c \cdot h^2}{J}$ \quad $\omega_k = \sqrt{\omega^2 + 2 \, k}$ \\
    
    mit $\Phi_+ = \varphi_1 + \varphi_2$ \qquad $\Phi_- = \varphi_2 + \varphi_1$ \\
\end{minipage}


\begin{tabular}{llll}
Symm: & $\ddot{\Phi_+} = - \omega^2 \Phi_+ $ & \quad &  $\varphi_1(t) = \varphi_2(t) = \frac{\Phi_0}{2} \cos{ \omega t} $\\
Antisymm: & $\ddot{\Phi_-} = - (\omega^2 + 2k) \Phi_- $ & \quad & $\varphi_1(t) = -\varphi_2(t) = \frac{\Phi_0}{2} \cos{ \omega_k t} $ \\
\\
\end{tabular}





\textbf{Allgemeine Lösunge des gekoppelten Systems:} \\
Lineare Kombination der symmetrischen und asymmetrischen Lösung \\

\renewcommand{\arraystretch}{1.3}
\begin{tabular}{lll}
$\varphi_1(t) = \Phi \, \sin( \Omega\, t) \cdot \cos(\overline{\omega} \, t)$ & & $\Omega = \frac{\omega_k - \omega}{2}$ \\
$\varphi_2(t) = \Phi \, \cos( \Omega\, t) \cdot \sin(\overline{\omega} \, t)$ & & $\overline{\omega} = \frac{\omega + \omega_k}{2}$ \\
\\
\end{tabular}
\renewcommand{\arraystretch}{1}


\begin{tabular}{c l c}
$k$ & Kopplungsfaktor & $[k] = 1$ \\
$h$ & Abstand zur Aufhängung & $[h] = \m$ \\
$J$ & Massenträgheitsmoment & $[J] = \kg \cdot \m^2 $ \\
$c$ & Federkonstante & $[c] = \frac{\N}{\m}$ \\
\end{tabular}


% \vfill\null
% \columnbreak

		\section{Wellen}

\subsection{Definition}

Eine Welle ist eine \textbf{Störung eines Gleichgewichtszustandes}, die sich \textbf{im Raum ausbreitet}.


\subsubsection{Bemerkungen zur Definition}

\begin{itemize}
	\itemsep0em
	\item Voraussetzung für die \textbf{Ausbreitung} einer Welle ist die \\
		\textbf{Kopplung} benachbarter Teilchen.	
	\item \textbf{Eine Welle transportiert Energie (keine Materie)} 
	\item Die Störung kann von ganz unterschiedlicher Natur sein: 
		\begin{itemize}
			\item Druck in Luft
			\item Auslenkung einer Position entlang einem Seil (Saite)
			\item Elektrische Signale  
		\end{itemize}
\end{itemize}


\textbf{Die Störung wird mit $\xi$ beschrieben: $\xi = \xi(x, y, z, t)$}



\subsection{Klassifizierung von Wellen}

\begin{minipage}{0.45\linewidth}
Welle breitet sich \textbf{senkrecht} zur Störung aus \\

\includegraphics[width=0.95\linewidth]{Bilder/Wellen-Optik/welle_transversal} \\
z.B. Lichtwellen
\end{minipage}
\hfill
\begin{minipage}{0.45\linewidth}
Welle breitet sich \textbf{parallel} zur Störung aus \\

\includegraphics[width=0.95\linewidth]{Bilder/Wellen-Optik/welle_longitudinal}\\
z.B. Schallwellen

\end{minipage}

\subsection{Wellengeschwindigkeit / Phasengeschwindigkeit $u$}
Die Störung an der Position $x_0$ zum Zeitpunkt $t=0$ breitet sich mit der Geschwindigkeit $u$ aus und erreicht nach einer Zeit $t$ die Position $x$ \\

\center{\includegraphics[width=0.7\linewidth]{Bilder/Wellen-Optik/ausbreitung_stoerung}} \\

\raggedright
\textbf{In einem Medium mit grösserer Dichte breiten sich Wellen schneller aus!} $\Rightarrow$ Bessere Kopplung der Moleküle \\
\vspace{0.5cm}

Man schaut bei der Beschreibung der Fortbewegung auf den Ort. Die Verschiebung des Ortes wird mit der Zeit hineingebracht

\subsubsection{Typische Wellengeschwindigkeiten}
\begin{tabular}{| l | c |}
	\hline
	\textbf{Material}   & \textbf{$m/s$} \\ \hline
	Wasser (20°C)       & 1300 \\ \hline
	Luft (20°C)         & 344 \\ \hline
	Kohlendioxid (20°C) & 258 \\ \hline
	Aluminium           & 5200 \\ \hline
	Eisen               & 5000 \\ \hline
	Tannenholz          & 3320 \\ \hline
	Beton               & 3100 \\ \hline
	Polystrol           & 1800 \\ \hline
	Kork                & 500 \\ \hline
	elektromagn. Welle  & 299'792'458 \\ \hline
\end{tabular}

% \vfill\null
% \columnbreak

\subsubsection{Verschiedene Wellengeschwindigkeiten}

\begin{minipage}{0.45\linewidth}
Schallwellen in Fluiden: 
\end{minipage}
\hfill
\begin{minipage}{0.48\linewidth}
$$\boxed{ u = \sqrt{\frac{1}{\rho \, \kappa}} }$$
\end{minipage}


\begin{minipage}{0.45\linewidth}
Schallwellen in Gasen:
\end{minipage}
\hfill
\begin{minipage}{0.48\linewidth}
$$\boxed{ u = \sqrt{\frac{\varkappa \, p}{\rho}} = \sqrt{\frac{\varkappa \, R \, T}{M}} }$$
\end{minipage}


\begin{minipage}{0.45\linewidth}
Elastische Longitudinalwellen in einem schlanken Stab
\end{minipage}
\hfill
\begin{minipage}{0.48\linewidth}
$$\boxed{ u = \sqrt{\frac{E}{\rho}} }$$
\end{minipage}

\begin{minipage}{0.45\linewidth}
Elastische Transversalwellen
\end{minipage}
\hfill
\begin{minipage}{0.48\linewidth}
$$\boxed{ u = \sqrt{\frac{G}{\rho}} }$$
\end{minipage}

\begin{minipage}{0.45\linewidth}
Transversalwellen auf einem Seil oder einer Saite
\end{minipage}
\hfill
\begin{minipage}{0.48\linewidth}
$$\boxed{ u = \sqrt{\frac{F}{\rho \, A}} }$$ 
\end{minipage}

\begin{minipage}{0.45\linewidth}
Elektromagnetische Wellen (transversal) \\
(z.B. Lichtwellen)
\end{minipage}
\hfill
\begin{minipage}{0.48\linewidth}
$$\boxed{ u = \frac{c}{n} }$$ \\
\end{minipage}


\renewcommand{\arraystretch}{1.1}
\begin{tabular}{c l c}
$u$ & Wellengeschwindigkeit & $[u] = \frac{\m}{\s}$ \\
$A$ & Querschnittsfläche & $[A] = \m^2$ \\
$E$ & Elastizitätsmodul & $[E] = \frac{\N}{\m^2} = \Pa$\\
$F$ & Spannkraft des Seils / der Saite & $[F] = \N$ \\
$G$ & Schubmodul & $[G] = \frac{\N}{\m^2} = \Pa$ \\
$M$ & Molmasse & $[M] = \frac{\kg}{\mol}$ \\
$p$ & Druck & $[p] = \Pa$ \\
$R$ & Universelle Gaskonstante: $R = 8.314 \mathrm{\frac{J}{mol \cdot K}}$ & $[R] = \mathrm{\frac{J}{mol \cdot K}} $ \\
$T$ & \textbf{Absolut-}Temperatur (in K) & $[T] = \mathrm{K}$ \\
$\kappa$ & Kompressibilität & $[\kappa] = \frac{1}{\Pa}$ \\
$\varkappa$ & Adiabatenexponent & $[\varkappa] = 1$ \\
$\rho$ & Dichte & $[\rho] = \frac{\kg}{\m^3}$ \\
$n$ & Brechungsindex & $[n] = 1$ \\
$c$ & Lichtgeschwindigkeit: $c = 300 \cdot 10^6 \, \frac{\m}{\s}$ & $[c] = \frac{\m}{\s}$ \\
\end{tabular}
\renewcommand{\arraystretch}{1}



\subsection{Wellengleichungen}
Die Wellengleichungen stellen eine \textbf{Verbindung zwischen Zeit und Ort} her \\

\vspace{0.5cm}


\begin{minipage}{0.45\linewidth}
Eindimensional \\
Welle breitet sich in 1D aus \\
\end{minipage}
\hfill
\begin{minipage}{0.48\linewidth}
$$\boxed{ \frac{\partial^2 \xi}{\partial x^2} = \frac{1}{u^2} \frac{\partial^2 \xi}{\partial t^2}  }$$  \\
\end{minipage}


\begin{minipage}{0.45\linewidth}
Zweidimensional \\
Welle breitet sich in 2D aus \\
\end{minipage}
\hfill
\begin{minipage}{0.48\linewidth}
$$\boxed{ \frac{\partial^2 \xi}{\partial x^2} + \frac{\partial^2 \xi}{\partial y^2} = \frac{1}{u^2} \frac{\partial^2 \xi}{\partial t^2}  }$$  \\
\end{minipage}


% \vfill\null
% \columnbreak


\begin{minipage}{0.45\linewidth}
Dreidimensional \\
Welle breitet sich in 3D aus \\
\end{minipage}
\hfill
\begin{minipage}{0.48\linewidth}
$$\boxed{ \frac{\partial^2 \xi}{\partial x^2} + \frac{\partial^2 \xi}{\partial y^2} + \frac{\partial^2 \xi}{\partial z^2} = \frac{1}{u^2} \frac{\partial^2 \xi}{\partial t^2}  }$$ \\
\end{minipage}



\subsubsection{Wichtige Lösung der Wellengleichung (1D)}

\begin{minipage}{0.38\linewidth}
$$\frac{\partial^2 \xi}{\partial x^2} = \frac{1}{u^2} \frac{\partial^2 \xi}{\partial t^2} $$ \\
\end{minipage}
\hfill
\begin{minipage}{0.58\linewidth}
Ansatz: $\xi(x,t) = \xi_0 \cdot \sin(\omega \, t - k \, x)$ \\
\end{minipage}

$$ \underbrace{- k^2 \, A \, \sin(\omega \, t - k \, x)}_{\substack{\frac{\partial^2 \xi}{\partial x^2} }} = - \frac{1}{u^2} \cdot \underbrace{ \omega^2 \, A \,\sin(\omega \, t - k \, x)}_{\substack{\frac{\partial^2 \xi}{\partial t^2} }}  $$ 


$$ \boxed{ \text{mit Lösung } u^2 = \frac{\omega^2}{k^2} } $$



\subsection{Harmonische Wellen}

$$ \boxed{ \xi(x, t) = \xi_0 \cdot \sin( \omega \, t - k \, x + \varphi) }  $$


\subsubsection{Terminologie}

\center{\includegraphics[width=0.7\linewidth]{Bilder/Wellen-Optik/harmonische_wellen_terminologie}} \\
\raggedright

\renewcommand{\arraystretch}{1.1}
\begin{tabular}{c l c}
$\xi_0$ & Amplitude & $[\xi_0]$ \\
$\omega$ & Kreisfrequentz & $[\omega] = \frac{\rad}{\s}$ \\
$T$ & Periodendauer & $[T] = \s$ \\
$k$ & Wellenzahl & $[k] = \frac{1}{\m}$ \\
$\lambda$ & Wellenlänge & $\lambda = \m $ \\
$u$ & Wellengeschwindigkeit & $[u] = \frac{\m}{\s}$ \\
$\varphi$ & Phasenverschiebung & $[\varphi] = \rad$ 
\end{tabular}
\renewcommand{\arraystretch}{1}



\subsubsection{Zusammenhänge}
\includegraphics[width=0.8\linewidth]{Bilder/Wellen-Optik/harmonische_wellen_zusammenhaenge}






\subsection{Wellenflächen / Wellenfronten}
Die Gesamtheit aller Punkte, die zu einer bestimmten Zeit im gleichen Schwingungszustand sind, bilden eine Fläche im Raum. \\

Diese \textbf{Flächen mit gleicher Phase} werden als \textbf{Wellenflächen} oder \textbf{Wellenfronten} genannt.\\
\vspace{0.2cm}

Eine Welle kann sich in 3 Dimensionen ausbreiten und dabei \textbf{verschiedene Wellenflächen} zeigen. 




\subsection{Wellenausbreitung}

\begin{minipage}{0.25\linewidth}
Wellengleichung (3D)
\end{minipage}
\hfill
\begin{minipage}{0.73\linewidth}
$$\boxed{ \frac{\partial^2 \xi}{\partial x^2} + \frac{\partial^2 \xi}{\partial y^2} + \frac{\partial^2 \xi}{\partial z^2} = \frac{1}{u^2} \frac{\partial^2 \xi}{\partial t^2}  }$$
\end{minipage}

\begin{minipage}{0.25\linewidth}
Lösungsansatz \\
\end{minipage}
\hfill
\begin{minipage}{0.73\linewidth}
$$ \boxed{ \xi = \xi_0 \e^{\jimg \, (\omega \, t - \vec{k} \bullet \vec{r})} }$$ \\
\end{minipage}

mit Wellenvektor $\vec{k} = \begin{pmatrix} k_x \\ k_y \\ k_z  \end{pmatrix}$ und Ortsvektor $ \vec{r} = \begin{pmatrix} r_x \\ r_y \\ r_z  \end{pmatrix}$




\subsubsection{Ebene Wellen}

\begin{minipage}{0.3\linewidth}
\includegraphics[width=0.98\linewidth]{Bilder/Wellen-Optik/Ebene_Welle}
\end{minipage}
\hfill
\begin{minipage}{0.68\linewidth}

\begin{itemize}
	\item Wellenfronten sind Ebenen im Raum 
	\item Wellenvektor $\vec{k}$ steht senkrecht auf \\
		der Ebenen 
	\item \textbf{Abstand} zw. zwei Wellenfronten ist $\lambda$ 

\end{itemize}


Die Ebenen bewegen sich mit der \\
Wellengeschwindigkeit
$$ \boxed{ u = \frac{\omega}{k} \text{ mit } k = ||\vec{k}|| = \sqrt{k_x^2 + k_y^2 + k_z^2} }$$  in die \textbf{Richtung}, die durch den \textbf{Wellenvektor} $\vec{k}$ gegeben ist.
\end{minipage}




\subsubsection{Kugelwellen}

\begin{minipage}{0.3\linewidth}
\includegraphics[width=0.98\linewidth]{Bilder/Wellen-Optik/Kugel_Welle}
\end{minipage}
\hfill
\begin{minipage}{0.68\linewidth}

\begin{itemize}
	\item Wellenfronten sind Kugeln 
	\item Wellenvektor $\vec{k}$ steht senkrecht auf  \\
 		Wellenfronten 
	\item Wellenfronten bewegen sich mit der \\
		 Wellengeschwindigkeit vom Zentrum weg 
	\item Amplitude nimmt mit $\frac{1}{r}$ ab \\
\end{itemize}
\end{minipage}

Für eine \textbf{punktförmige Quelle} und \textbf{keine Winkelabhängigkeit} gilt:

$$ \boxed{ \frac{1}{u^2} \frac{\partial^2 \xi}{\partial t^2} = \frac{2}{r} \Big( \frac{\partial \xi}{\partial r} \Big) + \frac{\partial^2 \xi}{\partial r^2} }$$

$$ \boxed{  \text{mit Lösungsansatz } \xi(t, r) = \frac{1}{r} \xi_0 \, \e^{\jimg \, (\omega \, t - k \, r)} }  $$

% \vfill\null
% \columnbreak




\subsection{Bewegte Quellen}

	\begin{minipage}{0.48\linewidth}
		\includegraphics[width=0.95\linewidth]{Bilder/Wellen-Optik/Bewegte_Quellen}
	\end{minipage}
	\hfill
	\begin{minipage}{0.48\linewidth}
		Die Quelle bewegt sich mit der \textcolor{blue}{Geschwindigkeit $v_Q$} in \\ eingezeichneter Richtung fort\\

		Die Quelle verschiebt sich in der Zeit $T$ um 
		$$ \boxed{ \Delta x = v_Q \cdot T }$$
	\end{minipage}

\subsubsection{Doppler Effekt}
	Die \textbf{Veränderung der Wellenlänge} $\lambda$ der von einer \textbf{bewegten Quelle} ausgesandten Wellen ist als \textbf{Doppler Effekt} bekannt.  \\

	\includegraphics[width=0.8\linewidth]{Bilder/Wellen-Optik/Doppler_effekt}

\subsubsection{Bewegte Quelle vs. unbewegte Quelle}

	\includegraphics[width=0.98\linewidth]{Bilder/Wellen-Optik/Unbewegte_Quelle}

\subsection{Frequenzverschiebung durch Bewegung}

	Die Quelle sendet eine Frequenz $f$ aus. Durch die Bewegung der Quelle ändert sich die Wellenlänge $\lambda$ und somit ergibt sich eine neue Frequenz $f'$, welche ein statischer Beobachter wahrnimmt. 

	Falls $v$ nicht parallel zur Beobachtungsrichtung ist, siehe \textbf{\ref{bewegte-Quelle-mit-Winkel}}

\begin{align*}
\lambda' &= \lambda -v_{Q_{||}} \cdot T  \\
\lambda'&= \lambda -v_{Q_{||}} \frac{\lambda}{u}\\
\lambda' &= \lambda \left(1 - \frac{v_{Q_{||}}}{u}\right)  \\
f'&= \frac{u}{\lambda'} = \frac{u}{\lambda\left(1-\frac{v_{Q_{||}}}{u}\right)} = \frac{f}{\left(1-\frac{v_{Q_{||}}}{u}\right)}
\end{align*}

\subsection{Bewegte Quelle oder bewegter Beobachter}

\begin{minipage}{0.48\linewidth}
	\begin{center}
		Bewegte Quelle:
		$$ \boxed{f' = \frac{1}{\left(1 \mp \frac{v_{Q_{||}}}{u}\right)}f}$$
	\end{center}
\end{minipage}
\hfill
\begin{minipage}{0.48\linewidth}
	\begin{center}
		Bewegter Beobachter:
		$$ \boxed{f' = \left( 1 \pm \frac{v_{B_{||}}}{u}  \right) \, f }$$\\
	\end{center}
\end{minipage}


\begin{tabular}{ll}
+ & Quelle bewegt sich weg / Beobachter bewegt sich hin\\
- & Quelle bewegt sich hin / Beobachter bewegt sich weg\\
\\
\end{tabular}


\begin{tabular}{clc}
$\lambda$ & Wellenlänge der aussendeten Welle & $[\lambda] = \m$ \\
$\lambda'$ & Wellenlänge der wahrgenommenen Welle & $[\lambda'] = \m$ \\
$v_{Q_{||}}$ & Geschwindigkeit der bewegten Quelle & $[v_{Q_{||}}] = \frac{\m}{\s}$ \\
$v_{B_{||}}$ & Geschwindigkeit des bewegten Beobachters & $[v_{Q_{||}}] = \frac{\m}{\s}$ \\
$u$ & Wellengeschwindigkeit & $[u] = \frac{\m}{\s}$ \\
$f$ & Frequenz der ausgesendeten Wellen & $[f] = \Hz$ \\
$f'$ & Frequenz der wahrgenommenen Wellen & $[f'] = \Hz$ \\
$T$ & Periodendauer (Dauer der Ausbreitung) & $[T] = \s$ \\
\end{tabular}






\subsection{Bewegte Quelle mit Winkel}\label{bewegte-Quelle-mit-Winkel}
Die Quelle bewegt sich nicht direkt auf den Beobachter zu, sondern sie bewegt sich am \textbf{Beobachter vorbei} \\


\begin{minipage}{0.48\linewidth}
\includegraphics[width=0.98\linewidth]{Bilder/Wellen-Optik/bewegte_quelle_winkel} \\
\end{minipage}
\hfill
\begin{minipage}{0.48\linewidth}
$$ \boxed{ f' =  \frac{1}{1 - \frac{v_Q}{u} \cdot \cos(v_Q)}  \, f} $$
\end{minipage}



\begin{tabular}{clc}
$v_Q$ & Geschwindigkeit der bewegten Quelle & $[v_Q] = \frac{\m}{\s}$ \\
$u$ & Wellengeschwindigkeit & $[u] = \frac{\m}{\s}$ \\
$f$ & Frequenz der ausgesendeten Wellen & $[f] = \Hz$ \\
$f'$ & Frequenz der wahrgenommenen Wellen & $[f'] = \Hz$ \\
\end{tabular}

% \vfill\null
% \columnbreak


\subsubsection{Beispiel Winkel zw. Quelle und Beobachter}

\begin{minipage}{0.48\linewidth}
\includegraphics[width=0.98\linewidth]{Bilder/Wellen-Optik/beispiel_bewegte_quelle} \\
\vspace{2.5cm}
\end{minipage}
\hfill
\begin{minipage}{0.48\linewidth}
\textbf{Gegeben:} $v_Q, \, \theta, \,u, \,d, \,l$ \\
\textbf{Gesucht:} $\frac{f'}{f}$ \\

$f' =  \frac{1}{1 - \frac{v_Q}{u} \cdot \cos(v_Q)}$ \\
$\tan(\theta) = \frac{d}{l} =  \frac{\sin(\theta)}{\cos(\theta)}$ \\
$\sin^2(\theta) = 1 - \cos^2(\theta)$ \\
\vspace{0.2cm}


$\Rightarrow \tan(\theta) = \frac{d}{l} = \frac{\sqrt{ 1 - \cos^2(\theta)}}{\cos(\theta)} $ \\

$ \frac{1}{\cos^2(\theta)} = \frac{d^2}{l^2} + 1 $ \\

$\cos^2(\theta) = \frac{1}{1 + \frac{d^2}{l^2}}$ \\

$\boxed{ \Rightarrow \frac{f'}{f} = \frac{1}{1 - \frac{v_Q}{u} \cdot \sqrt{\frac{l^2}{l^2 + d^2} } } } $

\end{minipage}




\subsection{Bewegte Quelle und bewegter Beobachter}

\begin{minipage}{0.48\linewidth}
\includegraphics[width=0.98\linewidth]{Bilder/Wellen-Optik/akustischer_doppler_effekt} \\
\end{minipage}
\hfill
\begin{minipage}{0.48\linewidth}
$$ \boxed{f_B = \frac{u+ v_B \cos(\vartheta_B)}{u- v_Q \cos(\vartheta_Q)}f_Q} $$
\end{minipage}

\renewcommand{\arraystretch}{1.1}
\begin{tabular}{clc}
$\vartheta_B = \theta_B$ & siehe Skizze & $[\vartheta_B] = $°\\
$v_B$ & Geschwindigkeit bewegter Beobachter & $[v_B] = \frac{\m}{\s}$ \\
$\vartheta_Q = \theta_Q$ & siehe Skizze & $[\vartheta_Q] = $°\\
$v_Q$ & Geschwindigkeit bewegte Quelle & $[v_Q] = \frac{\m}{\s}$ \\
$u$ & Wellengeschwindigkeit & $[u] = \frac{\m}{\s}$ \\
$f_B$ & Frequenz beim bewegten Beobachter & $[f_B] = \Hz$ \\
$f_Q$ & Frequenz bei der bewegten Quelle & $[f_Q] = \Hz$ \\
\end{tabular}
\renewcommand{\arraystretch}{1}

% \vfill\null
% \columnbreak


\subsubsection{Optischer Doppler Effekt}

\textbf{Wird verwendet, wenn die Wellengeschwindigkeit $\boldsymbol{u}$ gleich der Lichtgeschwindigkeit $\boldsymbol{c}$ ist!} \\

Es spielt nur die \textbf{relative} Bewegung von Beobachter und Quellen eine Rolle

$$ \boxed{f' = \frac{\sqrt{1-\beta^2}}{1 - \beta \cos(\vartheta)}f \text{ mit } \beta = \frac{v}{c} } $$


$$ \text{für } \beta << 1 \quad f' \simeq \frac{1}{1- \beta \cdot \cos(\theta)} \cdot f  \quad \text{für } \beta << 1 \text{ und } \theta \approx \frac{\pi}{2} \quad f' \simeq \frac{1}{1- \beta } \cdot f $$



\subsubsection{Mach'scher Kegel}
Wenn sich die Quelle schneller fortbewegt als die Wellengeschwindigkeit, dann entsteht ein Mach'scher Kegel \\



\begin{minipage}{0.6\linewidth}
\includegraphics[width=0.95\linewidth]{Bilder/Wellen-Optik/machkegel_1} \\
\end{minipage}
\hfill
\begin{minipage}{0.38\linewidth}
\includegraphics[width=0.95\linewidth]{Bilder/Wellen-Optik/machkegel_2} \\
\end{minipage}


\begin{minipage}{0.48\linewidth}
$$ \boxed{ \sin(\vartheta) = \frac{u \, \Delta t}{v \, \Delta t} = \frac{u}{v} } $$ \\
\end{minipage}
\hfill
\begin{minipage}{0.48\linewidth}
$$ \boxed{ M = \frac{v}{u} } $$ \\
\end{minipage}


\renewcommand{\arraystretch}{1.1}
\begin{tabular}{clc}
$v$ & Geschwindigkeit der Quelle & $[v] = \frac{\m}{\s}$ \\
$u$ & Wellengeschwindigkeit (Schallgeschwindingkeit) & $[u] = \frac{\m}{\s}$ \\
$M$ & Machzahl & $[M] = 1$
\end{tabular}
\renewcommand{\arraystretch}{1}



\subsection{Wellenwiderstand, Energietransport -- Schallwellen}

\subsubsection{Terminologie Wellenwiderstand}
Der \textbf{Wellenwiderstand $Z$} (auch \textbf{Impedanz} genannt) beschreibt, wie ein Medium den Fluss
von Energie beeinflusst. \\
\vspace{0.3cm}
$\Rightarrow$ 'Wie gut können sich Wellen in einem Medium ausbreiten?'

$$ \boxed{ Z = \rho \cdot u = \frac{\Delta p_0}{v_0} } $$

\vspace{0.2cm}

\renewcommand{\arraystretch}{1.2}
\begin{tabular}{clc}
$Z$ & Wellenwiderstand bzw. Impedanz & $[Z] = \Ohm = \frac{\Pa}{\m / \s} = \frac{\N \, \s}{\m^3}$ \\
$\rho$ & Dichte des Mediums & $[\rho] = \frac{\kg}{\m^3}$ \\
$u$ & Wellengeschwindigkeit & $[u] = \frac{\m}{\s}$ \\
$\Delta p_0$ & Druckamplitude & $[\Delta p_0] = \Pa$ \\
$v_0$ & Schnellenamplitude & $[v_0] = \frac{\m}{\s}$
\end{tabular}
\renewcommand{\arraystretch}{1}




\subsubsection{Weitere Terminologien}

\renewcommand{\arraystretch}{1.2}
\begin{tabular}{lll}
Schalldruck & $p = \Delta p_0 \, \cos(\omega \, t - k \, x)$ & $[p] = \Pa$ \\
Druckamplitude & $  \Delta p_0 = \rho \, u \, v_0 $ & \\
\\
Schallschnelle (Schnelle) & $v = v_0 \,  \cos(\omega \, t - k \, x)$ & $[v] = \frac{\m}{\s}$ \\
Schnellenamplitude & $v_0 = \omega \, \xi_0$ 
\end{tabular}
\renewcommand{\arraystretch}{1}



\subsubsection{Intensität der Schallwelle (siehe auch \ref{Pegel})}

\textbf{Intensität = gemittelte Energieflussdichte} \\

\begin{minipage}{0.48\linewidth}
$$ E_{kin} = \frac{\rho \cdot v^2}{4} = \frac{\rho \cdot v_0^2}{4} $$
\end{minipage}
\hfill
\begin{minipage}{0.48\linewidth}
$$ E_{pot} = \frac{p^2 - p_0^2}{2 \, \rho \, u^2} = \frac{\rho \cdot v_0^2}{4} $$
\end{minipage}

$$ E_{tot} = E_{kin} + E_{pot} = \frac{\rho \cdot v_0^2}{2} $$


$$ \boxed{ I = u \cdot \overline{w} = \frac{1}{2} \rho \, v_0^2 \, u  =  \frac{1}{2} \rho \, (\omega \, \xi_0)^2  \, u = \frac{( \Delta p_0)^2}{2 \, Z} = \frac{P}{A} } $$

\vspace{0.2cm}

\renewcommand{\arraystretch}{1.1}
\begin{tabular}{clc}
$I$ &  Schallintensität & $[I] = \frac{\W}{\m^2}$ \\
$\overline{w}$ & Energieflussdichte & $[\overline{w}] = \frac{\W \, \s}{\m^3}$ \\
 & Pot. Energie $\rightarrow$ 'Kompression Gas'& \\
 & Kin. Energie $\rightarrow$ 'Geschw. Teilchen'& \\
$\rho$ & Dichte & $[\rho] = \frac{\kg}{\m^3}$ \\
$v_0$ & Schnellenamplitude & $[v_0] = \frac{\m}{\s}$ \\
$\xi_0$ & Amplitude & $[\xi_0]$ \\
$u$ & Wellengeschwindigkeit & $[u] = \frac{\m}{\s}$ \\
$\Delta p_0$ & Druckamplitude & $[\Delta p_0] = \Pa$ \\
$Z$ & Wellenwiderstand bzw. Impedanz & $[Z] = \Ohm = \frac{\Pa}{\m / \s} = \frac{\N \, \s}{\m^3}$ \\
$P$ & Leistung & $[P] = \W$ \\
$A$ & (Abstrahl-) Fläche & $[A] = \m^2$ 
\end{tabular}
\renewcommand{\arraystretch}{1}



\subsection{Dispersion}
Die \textbf{Abhängigkeit} der Wellengeschwindigkeit \textbf{von der Wellenlänge} wird als Dispersion bezeichnet. \\
\vspace{0.2cm}
$\Rightarrow$ Siehe Beispiel Optik Abschnitt \ref{Dispersion Optik}

\subsubsection{Dispersion bei Wasserwellen}

$$ \boxed{ u(\lambda) = \sqrt{\left( \frac{g \cdot \lambda}{2 \, \pi} + \frac{2 \, \pi \cdot \sigma}{\rho \cdot 
\lambda} \right) \cdot \tanh \left( \frac{2 \, \pi \cdot h}{\lambda} \right) }  } $$ \\


\begin{minipage}{0.48\linewidth}
tiefes Wasser ($\lambda << h$)
$$ \boxed{ u(\lambda) = \sqrt{\frac{g \cdot \lambda}{2 \, \pi}} } $$
\end{minipage}
\hfill
\begin{minipage}{0.48\linewidth}
flaches Wasser ($\lambda >> h$)
$$ \boxed{ u = \sqrt{ g \cdot h } } $$
\end{minipage}

\vspace{0.2cm}

\renewcommand{\arraystretch}{1.1}
\begin{tabular}{clc}
$g$ & Erdbeschleinigung $g = 9.81 \, \frac{\m}{\s^2}$ & $[g] = \frac{\m}{\s^2}$ \\
$\lambda$ & Wellenlänge & $[\lambda] = \m $ \\
$\sigma$ & Oberflächenspannung & $[\sigma] = \frac{\N}{\m} $ \\
$h$ & Wassertiefe & $[h] = \m$ \\
$\rho$ & Dichte & $[\rho] = \frac{\kg}{\m^3}$
\end{tabular}
\renewcommand{\arraystretch}{1}










\section{Superposition von Wellen}

\textbf{Superposition beschreibt die Überlagerung (Addition) \\
von Wellen} \\

\begin{itemize}

	\item Linearität der Wellengleichung 
	\item Die Summe zweier Lösungen der Wellengleichung ist auch eine \\
		Lösung der Wellengleichung. \\

\end{itemize}

Das Superpositionsprinzip erlaubt die Darstellung von periodischen Wellen als eine Summe von harmonischen Wellen.




\subsection{Schwebung}

Superposition von Wellen mit \textbf{unterschiedlichen} Frequenzen \\
$\Rightarrow$ Hörbar als ein 'Flattern' 

$$ \boxed{f_{\text{Schwebung}} = f_2 - f_1 } $$ \\


\begin{minipage}{0.48\linewidth}
$$ \xi_1 = A \cdot \sin(\omega_1 \, t) $$
\end{minipage}
\hfill
\begin{minipage}{0.48\linewidth}
$$ \xi_2 = A \cdot \sin(\omega_2 \, t) $$
\end{minipage}

$$ \boxed{ \xi = 2 \, A \cdot \sin( \overline{\omega} \, t) \cdot \cos( \Ohm \, t) }$$

mit $\overline{\omega} = \frac{\omega_1 + \omega_2}{2}$ und  $\Ohm = \frac{\omega_2 - \omega_1}{2}$



\subsection{Interferenz}

Superposition von Wellen mit \textbf{gleichen} Frequenzen \\
\vspace{0.2cm}

\begin{minipage}{0.4\linewidth}
\includegraphics[width=0.7\linewidth]{Bilder/Wellen-Optik/interferenz} \\
\end{minipage}
\hfill
\begin{minipage}{0.55\linewidth}
$$ \xi_1 = A \cdot \sin(\omega \, t - k \, r_1) $$

$$ \xi_2 = A \cdot \sin(\omega \, t - k \, r_2) $$

$$ \boxed{ \xi = 2 \, A \cdot \sin \left( \omega \, t - \frac{k (r_1 + r_2)}{2} \right) \,  \cos \left( k \frac{\Delta r}{2}  \right) } $$
\end{minipage}



$\Rightarrow$ Der $\cos$-Term hängt nur vom Ort ab! \\
$\Rightarrow$ Es gibt \textbf{Orte}, an denen Welle sich auslöscht!

% \vfill\null
% \columnbreak




\subsection{Kohärenz}

Zwei Wellen werden als kohärent bezeichnet, wenn eine \textbf{feste Phasendifferenz} zwischen den beiden Wellen besteht. \\
\vspace{0.2cm}
Kohärenz ist eine \textbf{Vorbedingung}, damit sich eine \textbf{Interferenz} bilden kann. \\
\vspace{0.2cm}
\textbf{Kohärenzlänge} ist der \textbf{maximale Streckenunterschied}, den zwei Wellen haben dürfen, damit eine \textbf{(stabile) Interferenz} beobachtet werden kann.


\subsection{Reflexion und Transmission}


\subsubsection{Verhalten von Wellem an Grenzflächen von zwei Medien}

Ein Teil der Welle wird reflektiert und ein Teil wird transmittiert \\

\includegraphics[width=0.95\linewidth]{Bilder/Wellen-Optik/reflexion_transmission} \\

\vspace{0.2cm}

\textbf{Physikalische Bedingung}\\
Stetigkeit der Wellenfunktion und der Ableitung an der Grenzfläche

$$ \xi_1(0) = \xi_2(0) \qquad \dot{\xi}_1(0) = \dot{\xi}_2(0) $$




\subsubsection{Intensität von Reflexion und Transmission}\label{Reflexionskoeffizient}

\begin{minipage}{0.48\linewidth}
	\includegraphics[width=0.9\linewidth]{Bilder/Wellen-Optik/reflexionskoeffizient_transmissionskoeffizient} \\
\end{minipage}
\hfill
\begin{minipage}{0.48\linewidth}
	$$ \boxed{ R = \left(  \frac{Z_1 - Z_2}{Z_1 + Z_2}  \right)^2  }  \quad \boxed{ T = \frac{ 4 \cdot Z_1 \cdot Z_2}{(Z_1 + Z_2)^2}   }$$
	$$ \boxed{ \sqrt{R} = \frac{Z_1 - Z_2}{Z_1 + Z_2} = \frac{u_R}{u_H} = \frac{A_R}{A_H} = \frac{p_R}{p_H} = ...} $$

	$$ \boxed{ \frac{2 \cdot Z_2}{(Z_1 + Z_2)}  = \frac{u_T}{u_H} = \frac{A_T}{A_H} = \frac{p_T}{p_H} = ... } $$
\end{minipage}

\vspace{0.2cm}


\renewcommand{\arraystretch}{1.1}
\begin{tabular}{clc}
$R$ & Reflexionskoeffizient  & $[R] = 1$ \\
$r$ & Amplitudenverhältnis  & $[r] = 1$ \\
$u_H$ & Geschw. hinlaufende Welle & $[u_H] = \frac{m}{s}$ \\
$u_R$ & Geschw. refl. Welle & $[u_R] = \frac{m}{s}$ \\
$u_T$ & Geschw. trans. Welle & $[u_T] = \frac{m}{s}$ \\
$p_H$ & Druck hinlaufende Welle & $[p_H] = Pa$ \\
$p_R$ & Druck refl. Welle & $[p_R] = Pa$ \\
$p_T$ & Druck trans. Welle & $[p_T] = Pa$ \\
$A_H$ & Ampl. hinlaufende Welle & $[A_H] = 1$ \\
$A_R$ & Ampl. refl. Welle & $[A_R] = 1$ \\
$A_T$ & Ampl. trans. Welle & $[A_R] = 1$ \\
$T$ & Transmissionskoeffizient & $[T] = 1 $ \\
$Z_n$ & Wellenwiderstand im Medium $n$ & $[Z] = \Ohm = \frac{\Pa}{\m / \s} = \frac{\N \, \s}{\m^3}$ \\
\end{tabular}
\renewcommand{\arraystretch}{1}



\subsubsection{Phasensprünge bei Reflexionen}

\begin{minipage}{0.4\linewidth}
	\includegraphics[width=0.9\linewidth]{Bilder/Wellen-Optik/reflexion_phasensprung} \\
\end{minipage}
\hfill
\begin{minipage}{0.58\linewidth}
	Reflexion an Material mit tieferer Wellenimpedanz\\
	$\Rightarrow$ \textbf{Phasensprung} \\
	(El. Impedanz : \(z\rightarrow 0\))\\

	dicht: $n_2 > n_1$ \\
	\textbf{kleinere} Wellengeschwindigkeit,\\
	\textbf{grösserer} Wellenwiderstand $Z$\\
	\vspace{0.2cm}

	Reflexion an Material mit höherer Wellenimpedanz\\
	$\Rightarrow$ \textbf{kein Phasensprung} \\
	(El. Impedanz : \(z\rightarrow \infty \))\\

\end{minipage}


\subsection{Anwendung: Elektromagnetische Wellen}

\subsubsection{Elektromagnetische Wellem in Doppelleiter}

\begin{minipage}{0.48\linewidth}
$$ \boxed{ u_r = \frac{Z_2 - Z_1}{Z_1 + Z_2} \, u_h } $$
\end{minipage}
\hfill
\begin{minipage}{0.48\linewidth}
$$ \boxed{ u_t = \frac{2 \, Z_2}{Z_1 + Z_2} \, u_h } $$
\end{minipage}


\begin{minipage}{0.48\linewidth}
$$ \boxed{ i_r = -\frac{Z_2 - Z_1}{Z_1 + Z_2} \, i_h } $$ \\
\end{minipage}
\hfill
\begin{minipage}{0.48\linewidth}
$$ \boxed{ i_t = \frac{2 \, Z_1}{Z_1 + Z_2} \, i_h } $$ \\
\end{minipage}

\textbf{Kabel mit kurzgeschlossenem Ende $\boldsymbol{Z_2 = 0}$} \\
\vspace{0.2cm}
\begin{minipage}{0.3\linewidth}
$$ u_r = - u_h $$ \\
\end{minipage}
\hfill
\begin{minipage}{0.68\linewidth}
$$ u_1 = u_h + u_r  = u_h - u_h = 0 $$ \\
\end{minipage}
\vspace{-0.2cm}


\textbf{Kabel mit offenem Ende $\boldsymbol{Z_2 = \infty}$} \\
\vspace{0.2cm}
\begin{minipage}{0.3\linewidth}
$$ i_r = - i_h $$ 
\end{minipage}
\hfill
\begin{minipage}{0.68\linewidth}
$$ i_1 = i_r + i_h  = i_r - i_r = 0 $$  
\end{minipage}

\vspace{0.2cm}

\renewcommand{\arraystretch}{1.1}
\begin{tabular}{clc}
$u_r$ & Reflektierte Spannung & $[u_r] = \V$ \\
$u_h$ & Eintreffende Spannung & $[u_h] = \V$ \\
$i_r$ & Reflektierter Strom & $[i_r] = \A$ \\
$i_h$ & Eintreffender Strom & $[i_h] = \A$ \\
$Z_n$ & Wellenwiderstand im Medium $n$ & $[Z] = \Ohm = \frac{\Pa}{\m / \s} = \frac{\N \, \s}{\m^3}$ \\
\end{tabular}
\renewcommand{\arraystretch}{1}




\subsubsection{Elektromagnetische Wellen in homogenem Milieu}

$$ \boxed{ Z = \frac{E}{H} = \sqrt{ \frac{\mu_r \mu_0}{\varepsilon_r \varepsilon_0}} = \sqrt{\frac{\mu_r}{\varepsilon_r}} Z_0 = Z_0 \frac{c}{n}} \quad \boxed{ R = \left(  \frac{Z_1 - Z_2}{Z_1 + Z_2}  \right)^2 =  \left(  \frac{n_1 - n_2}{n_1 + n_2}  \right)^2 }$$


$$ \boxed{ E_{r0} = \frac{Z_2 - Z_1}{Z_1 + Z_2} E_{h0} } \quad \boxed{E_{t0} = \frac{2 \cdot Z_2}{Z_1 + Z_2} E_{h0}} $$

\renewcommand{\arraystretch}{1.1}
\begin{tabular}{clc}
$R$ & Reflexionskoeffizient  & $[R] = 1$ \\
$Z_n$ & Wellenwiderstand im Medium $n$ & $[Z] = \Ohm = \frac{\Pa}{\m / \s} = \frac{\N \, \s}{\m^3}$ \\
$\mu_r$ & Permeabilitätszahl & $[\mu_r] = 1$ \\
$\varepsilon$ & Dielektizitätszahl & $[\varepsilon_r] = 1$ \\
$n_n$ & Brechungsindex von Medium $n$ & $[n_1] = 1$ \\
\end{tabular}


\renewcommand{\arraystretch}{1.2}
\begin{tabular}{ll}
$\varepsilon_0$ & El. Feldkonstante $\varepsilon_0 = 8.854 \cdot 10^{-12} \, \frac{\A \, \s}{\V \, \m}$ \\
$\mu_0$ & Magn. Feldkonstante $\mu_0 = 4 \pi \cdot 10^{-7} \frac{\N}{\A^2}$\\
$Z_0$ & Wellenwiderstand Vakuum $Z_0 \approx 376.73 \, \Ohm$ \\
$c$ & Lichtgeschwindigkeit $c = 300 \cdot 10^6 \, \mathrm{\frac{m}{s}}$ \\ 
\end{tabular}
\renewcommand{\arraystretch}{1}


% \vfill\null
% \columnbreak



\section{Stehende Wellen}

\subsubsection{Terminologie}

Eine \textbf{stehende Welle} ist eine Welle, bei der Orte maximaler Auslenkung (oder minimaler Auslenkungen) sich
\textbf{nicht fortbewegen} 


\begin{itemize}
	\item Ort- und Zeitabhängigkeit sind separiert 
	\item Die Welle bewegt sich nicht im Raum ('Muster bleibt stehen') 
\end{itemize}



$$ \boxed{ \xi(x, t) = \xi_0 \cdot \cos(\omega t) \cdot \cos(k x) } $$

$\Rightarrow$ $\sin()$-Terme sind auch erlaubt!\\
\vspace{0.2cm}

Orte, wo die Welle für alle Zeit $= 0$ ist heissen \textbf{Wellenknoten} \\
$\Rightarrow$ \textbf{Zwei benachbarte Knoten sind $\boldsymbol{\frac{\lambda}{2}}$ auseinander}\\
\vspace{0.2cm}

Orte, wo die Welle eine maximale Auslenkung erreicht, heissen \textbf{Wellenbauch} \\
$\Rightarrow$ \textbf{Zwei benachbarte Bäuche sind $\boldsymbol{\frac{\lambda}{2}}$ auseinander}





\subsection{Entstehung von stehenden Wellen}

\includegraphics[width=0.9\linewidth]{Bilder/Wellen-Optik/stehende_wellen}



\subsection{Prinzip von Huygens}

\begin{minipage}{0.48\linewidth}
\includegraphics[width=0.9\linewidth]{Bilder/Wellen-Optik/huygens}
\end{minipage}
\hfill
\begin{minipage}{0.48\linewidth}
\raggedright
Jedes Flächenelement auf einer Wellenfläche kann als Zentrum einer Kugelwelle betrachtet\\
werden. \\
Die Wellenfläche zu einem späteren Zeitpunkt ist die Einhüllende all dieser Elementarwellen.
\end{minipage}

\subsection{Eigenschwingungen - 1D}

\begin{minipage}{0.3\linewidth}
$ \boxed{ f_n = \frac{n}{2 \, l} \cdot u = \frac{u}{\lambda_n} } $
\end{minipage}
\hfill
\begin{minipage}{0.66\linewidth}
$ \boxed{ \text{Auslenkung = 0: } k_n \cdot l = n \cdot \pi } $ \\
$ \boxed{ \text{Auslenkung max: } k_n \cdot l = \frac{\pi}{2} + n \cdot \pi  } $
\end{minipage}


\subsubsection{Saite}

\begin{itemize}
	\item Reflexion an einer Grenzfläche $\rightarrow$  Stehende Welle 
	\item Die stehende Welle muss in den vorhandenen Raum passen \\
		 $\rightarrow$ Geometrische Bedingung 
	\item \textbf{Knoten an beiden Enden} 
\end{itemize}



\begin{minipage}{0.48\linewidth}
\includegraphics[width=0.98\linewidth]{Bilder/Wellen-Optik/saite} \\
\end{minipage}
\hfill
\begin{minipage}{0.48\linewidth}
$ \boxed{ \xi = \xi_0 \, \sin(\omega t) \cdot \sin(k x)  } $ \\

$ \boxed{ \text{Auslenkung = 0: } k_n \cdot l = n \cdot \pi } $ \\

$ \boxed{ f_n = \frac{u}{\lambda_n} = \frac{n}{2 \, l} \cdot u =  \frac{n}{2 \, l} \sqrt{\frac{F}{\rho \cdot A}} } $ \\
\end{minipage}

\vspace{0.2cm}

\renewcommand{\arraystretch}{1.1}
\begin{tabular}{clc}
$k_n$ & Wellenzahl & $[k_n] = \frac{1}{\m} $ \\
$l$ & Länge der Saite & $[l] = \m$ \\
$u$ & Wellengeschwindigkeit & $[u] \frac{\m}{\s}$ \\
$\lambda_n$ & Wellenlänge & $[\lambda_n]= \m$ \\
$n$ & Ganze Zahl & $[n] = 1$ \\
$\omega$ & Kreisfrequenz & $[\omega] = \frac{\rad}{\s}$ \\
$A$ & Querschnitt der Saite & $[A] = \m^2$
\end{tabular}



\subsubsection{Pfeifen}

\begin{minipage}{0.48\linewidth}
	\textbf{Offene Pfeife}\\
	Länge $l = \frac{\lambda}{2}$ \\

	\includegraphics[width=0.8\linewidth]{Bilder/Wellen-Optik/offene_pfeiffe}\\

	Auslenkung an Enden: \\
	Wellenbauch \\
\end{minipage}
\hfill
\begin{minipage}{0.48\linewidth}
	\textbf{Gedackte Pfeife} \\
	Länge $l = \frac{\lambda}{4}$ \\

	\includegraphics[width=0.85\linewidth]{Bilder/Wellen-Optik/gedackte_pfeiffe} \\

	Auslenkung offenes Ende: Wellenbauch \\
	Auslenkung gedacktes Ende: Knoten \\
	(max. Auslenkung) \\
\end{minipage}

\begin{center}
	$ \boxed{f_n = \underbrace{\frac{n}{2l}\sqrt{\frac{\varkappa R T}{M}}}_{\text{offen}}, \qquad f_n = \underbrace{\frac{2n+1}{4l}\sqrt{\frac{\varkappa R T}{M}}}_{\text{gedackt}}} $
\end{center}


\subsection{Eigenschwingungen - 2D}

\subsubsection{Rechteckige Membrane}


\begin{minipage}{0.3\linewidth}
	\includegraphics[width=0.98\linewidth]{Bilder/Wellen-Optik/membrane} \\
	\includegraphics[width=0.98\linewidth]{Bilder/Wellen-Optik/membrane-3d.jpg} \\
\end{minipage}
\hfill
\begin{minipage}{0.68\linewidth}
	$$ \boxed{ \xi(x,y,t) = \xi_0 \, \sin(\omega t) \, \sin(k_x x) \, \sin(k_y y)  } $$

	$$ \boxed{ g(t)\frac{\partial^2f}{\partial x^2} + g(t)\frac{\partial^2f}{\partial y^2} = \frac{1}{u^2}f(x,y)g(t)\frac{\partial^2g}{\partial t^2} } $$

	$$ \boxed{ \text{Wellenvektor: } k = |\vec{k}| = \sqrt{k_x^2 + k_y^2}} $$

	$$ \boxed{ u^2 = \frac{\omega^2}{k^2}} $$
\end{minipage}

\textbf{Randbedingungen:} %\\
\begin{tabular}{l c c }
	Auslenkung = 0 &  $ \boxed{ k_x = \frac{n \, \pi}{a} }$ & $ \boxed{ k_y = \frac{m \, \pi}{b} }$
\end{tabular} 

\begin{minipage}{0.45\linewidth}
	\begin{tabular}{clc}
		$u$ & Wellengeschwindigkeit & $[u] = \frac{\m}{\s}$ \\
		$m$ & Ganze Zahl & $[m] = 1$ \\
		$n$ & Ganze Zahl & $[n] = 1$ \\
		$\omega$ & Kreisfrequenz & $[\omega] = \frac{\rad}{\s}$ \\
		$k_{...}$ & Wellenzahl & $[k_{...}] = \frac{1}{\m} $ \\
		$a,b$ & Seitenlänge Membran & $[a,b] = \m$ \\
		$A$ & Querschnitt der Membrane & $[A] = \m^2$
	\end{tabular}
\end{minipage}
\hfill
\begin{minipage}{0.45\linewidth}
	$$ \boxed{f_{mn} = \frac{1}{2}\sqrt{\frac{F}{\rho A}}\sqrt{\frac{m^2}{a^2} + \frac{n^2}{b^2}}} $$
\end{minipage}



\section{Beugung}

\subsubsection{Terminologie}

Die \textbf{Richtungsänderungen} der Wellenausbreitung in einem homogenen Medium durch \textbf{Hindernisse} wird als \textbf{Beugung (Diffraktion)} bezeichnet. \\
\begin{itemize}
	\item Das Hindernis kann eine Kante, ein Spalt oder ein kleines \\
	Objekt sein sein 
	\item Beugung tritt auf, wenn das \textbf{Hindernis} von \textbf{ähnlicher}\\
	\textbf{Grösse} ist, wie die \textbf{Wellenlänge} 
\end{itemize}


$\Rightarrow$ \textbf{Beugung tritt auf, wenn eine Welle limitiert wird!} \\
\vspace{0.2cm}

Dies gilt insbesondere für: \\
Spalt, Kante, Loch (Pinhole), Objektiv-Öffnung


\subsection{Beugung - Spalt}

\subsubsection{Beschreibung Setup}\label{Beschreibung Setup}

\includegraphics[width=0.7\linewidth]{Bilder/Wellen-Optik/beugung_spalt} 


\subsubsection{Intensität nach dem Spalt}\label{Spalt-Intensitaet}


\begin{minipage}{0.58\linewidth}
\includegraphics[width=0.98\linewidth]{Bilder/Wellen-Optik/beugung_spalt_intensitaet} \\
\end{minipage}
\hfill
\begin{minipage}{0.4\linewidth}
$$ \boxed{ I_s \varpropto \frac{A^2}{r^2} \frac{\sin^2\Big( \frac{k \cdot s \cdot \sin(\varphi)}{2} \Big) }{\Big( \frac{k \cdot s \cdot \sin(\varphi)}{2} \Big)^2 } } $$


\textbf{Minima der Intensität} 

$$ \boxed{ \sin(\varphi) = n \, \frac{\lambda}{s}} $$ \\
\end{minipage}



\renewcommand{\arraystretch}{1.1}
\begin{tabular}{clc}
$I_s$ & Intensität & $[I_s] = \frac{W}{\m^2}$ \\
$s$ & Länge des Spalts & $[s] = \m$ \\
$\lambda$ & Wellenlänge & $[\lambda]= \m$ \\
$n$ & Ordnung (typ. 1) & $[n] = 1$ \\
$A$ & Amplitude & $[A]$ \\
$r$ & siehe Bild Abschnitt \ref{Beschreibung Setup} & $[r] = \m$ \\
$\varphi$ & Einfallswinkel der Welle zum Spalt & $[\varphi] =$° \\
 & \textcolor{blue}{Hinweis: Oft muss gegebenes $\varphi$ durch 2} & \\
 & \textcolor{blue}{geteilt werden!} \\
\end{tabular}
\renewcommand{\arraystretch}{1}

% \vfill\null
% \columnbreak


\subsection{Beugung - Runde Öffnung (Pinhole)}

\begin{minipage}{0.48\linewidth}
\includegraphics[width=0.98\linewidth]{Bilder/Wellen-Optik/beugung_loch} \\
\end{minipage}
\hfill
\begin{minipage}{0.48\linewidth}
\textbf{Nullstelle erster Ordnung:} 

$$ \boxed{ \sin(\varphi) = 1.22 \frac{\lambda}{D} } $$ 
\end{minipage}


\renewcommand{\arraystretch}{1.1}
\begin{tabular}{clc}
$\lambda$ & Wellenlänge & $[\lambda]= \m$ \\
$D$ & Loch-Durchmesser & $[D] = \m$
\end{tabular}
\renewcommand{\arraystretch}{1}



\subsection{Beugung - Gitter} % TODO: Formeln aus Buch eintragen

\subsubsection{Beschreibung Setup}
\includegraphics[width=0.9\linewidth]{Bilder/Wellen-Optik/beugung_gitter} 


\subsubsection{Intensität nach dem Gitter}

$$ \boxed{ I_G = \frac{A^2}{r^2} \textcolor{red}{A_s^2} \, \textcolor{green}{B^2} }$$


% \begin{minipage}{0.48\linewidth}
% $$\textcolor{blue}{A_s^2(\varphi) = \frac{\sin^2\Big( \frac{k \cdot s \cdot \sin(\varphi)}{2} \Big) }{\Big( \frac{k \cdot s \cdot \sin(\varphi)}{2} \Big) } } $$
% \end{minipage}
% \hfill
% \begin{minipage}{0.48\linewidth}
% $$ \textcolor{orange}{B^2(\varphi) = \frac{\sin^2 \Big( Z  \frac{k \cdot d \cdot \sin(\varphi)}{2} \Big) }{\Big( \frac{k \cdot d \cdot \sin(\varphi)}{2} \Big) }  }$$
% \end{minipage}


\includegraphics[width=0.9\linewidth]{Bilder/Wellen-Optik/beugung_gitter_intensitaet} \\
\vspace{0.2cm} 
\renewcommand{\arraystretch}{1.2}
\begin{tabular}{ll}
$A_s^2(\varphi)$ & hat Nullstellen bei $n \, \frac{\lambda}{s}$ (hängt von $s$ ab), gleiches $A_s$ wie in \ref{Spalt-Intensitaet} \\
$B^2(\varphi)$ & hat Hauptmaxima bei $n \, \frac{\lambda}{d}$ und $Z - 2$ Nebenmaxima\\
& dazwischen (hängt von $Z$ und $d$ ab)\\
\end{tabular}
\renewcommand{\arraystretch}{1}


\vspace{0.2cm}

\renewcommand{\arraystretch}{1.3}
\begin{tabular}{clc}
$I_G$ & Intensität nach Gitter & $[I_G] = \frac{W}{\m^2}$ \\
$k$ & Wellenzahl & $[k] = \frac{1}{\m}$ \\
$s$ & Spalt & $[s] = \m$ \\
$d$ & Gitterkonstante & $[d] = \m$ \\
$Z$ & Anzahl der Spalten & $[Z] = 1$
\end{tabular}
\renewcommand{\arraystretch}{1}

% \vfill\null
% \columnbreak


\subsubsection{Auflösungsvermögen}

Verschiedene Wellenlängen können getrennt (aufgelöst) werden \\


\includegraphics[width=0.9\linewidth]{Bilder/Wellen-Optik/aufloesungsvermoegen} \\

\vspace{0.2cm}

\textbf{Kriterium}\\

Zwei Wellenlängen werden gerade
noch aufgelöst, wenn das
Hauptmaximum von $\lambda_2$ mit dem Minimum von $\lambda_1$ zusammenfällt

$$ \boxed{ \frac{\lambda}{\Delta \lambda} = n \, Z } $$


\renewcommand{\arraystretch}{1.3}
\begin{tabular}{clc}
$\lambda$ & Wellenlänge & $[\lambda] = \m$ \\
$\Delta \lambda$ & Unterschied der Wellenlängen & $[ \Delta \lambda] = \m$ \\
$n$ & Ordnung der Beugung (typ. 1) & $[n] = 1$ \\
$Z$ & Anzahl der Spalten & $[Z] = 1$ 
\end{tabular}
\renewcommand{\arraystretch}{1}




\subsection{Babinet-Prinzip}

\begin{minipage}{0.48\linewidth}
\includegraphics[width=0.98\linewidth]{Bilder/Wellen-Optik/babinet_prinzip} \\
\end{minipage}
\hfill
\begin{minipage}{0.48\linewidth}

Ausserhalb des Bereichs \\
geometrisch-optischer Abbildung\\
produzieren komplementäre\\
Beugungsschirme gleiche \\
Beugungsbilder
\end{minipage}





\section{Akustik}

\subsubsection{Terminologie}

\textbf{Ton}\\
Eine harmonische Schallwelle wird als \textbf{Ton} bezeichnet. \\
Ein (reiner) Ton entspricht also einer \textbf{Schallschwingung}, die \textbf{eine
einzige Frequenz} enthält. \\
\vspace{0.2cm}

\textbf{Klang} \\
Eine \textbf{Überlagerung} von harmonischen Schwingungen, deren
\textbf{Frequenzen} in einem \textbf{ganzzahligen Verhältnis} zur tiefsten
Frequenz, zur Frequenz des \textbf{Grundtons}, stehen, wird \textbf{Klang}
genannt. \\
\vspace{0.2cm}

\textbf{Geräusch} \\
Bei einem \textbf{Geräusch} besteht das Frequenzspektrum nicht mehr
aus einzelnen diskreten Linien, sondern weist in einem
bestimmten Frequenzbereich eine  \textbf{kontinuierliche Verteilung} auf. \\




\subsection{Pegel}\label{Pegel}

\begin{minipage}{0.48\linewidth}
\center{\textbf{'Einheit': Bel}} 
$$ \boxed{ \text{Pegel} = \log \Big( \frac{x}{b_0} \Big) } $$
$$ \boxed{ x = b_0 \cdot 10^{\text{Pegel} } } $$ 

\end{minipage}
\hfill
\begin{minipage}{0.48\linewidth}
\center{\textbf{'Einheit': Dezibel}}
$$ \boxed{ \text{Pegel} = 10 \cdot \log \Big( \frac{x}{b_0} \Big) } $$ 
$$ \boxed{ x = b_0 \cdot 10^{\Big( \frac{\text{Pegel}}{10} \Big) } } $$ \\
\end{minipage}

\renewcommand{\arraystretch}{1.3}
\begin{tabular}{clc}
Pegel & Dimensionslose Grösse & $[\text{Pegel}] =  1$ \\
$x$ & Zu vergleichende Grösse & $[x]$ \\
$b_0$ & Basisgrösse & $[b_0] = [x]$ \\
\end{tabular}
\renewcommand{\arraystretch}{1}

\raggedright



\subsection{Schallintensität}

\begin{minipage}{0.28\linewidth}
	$$ \boxed{ L_I = 10 \cdot \log \left( \frac{I}{I_0} \right) } $$ 
\end{minipage}
\hfill
\begin{minipage}{0.40\linewidth}
	$$ \boxed{I = \frac{1}{2}\rho v_0^2 u = \frac{\Delta p_0^2}{2 \rho u} }$$ 
\end{minipage}
\hfill
\begin{minipage}{0.28\linewidth}
	$$ \boxed{ p_{eff} = \frac{\Delta p_0}{\sqrt{2}} } $$ 
\end{minipage}

\begin{minipage}{0.28\linewidth}
	$$ \boxed{ L_p = 20 \cdot \log \left( \frac{p_{eff}}{p_{eff0}} \right) } $$ 
\end{minipage}
\hfill
\begin{minipage}{0.28\linewidth}
	$$ \boxed{ L_P = 10 \cdot log \left(\frac{P}{P_0}\right)} $$ 
\end{minipage}

\renewcommand{\arraystretch}{1.3}
\begin{tabular}{clc}
	$L_I$ & Schallintensitätspegel & $[L_I] = \dB$ \\
	$I$ & Intensität & $[I] = \frac{\W}{\m^2}$\\
	$I_0$ & Bezugsintensität  & $I_0 = 10^{-12} \frac{\W}{\m^2}$ \\
	$L_p$ & Schalldruckpegel & $[L_p] = \dB$ \\
	$p_{eff}$ & Schalldruck (Effektivwert) & $[p_{eff}] = \Pa$ \\
	$p_{eff0}$ & Bezugsschalldruck & $p_{eff0} = 2 \cdot 10^{-5} \Pa$ \\
	$L_P$ & Schallleistungspegel & $[L_P] = \dB$ \\
	$P$ & Schallleistung & $[P] = \W$ \\
	$P_0$ & Bezugsschallleistung & $P_0 = 10^{-12} \W$ \\
	$\Delta p_0$ & Druckamplitude & $[\Delta p_0] = \Pa$ \\
\end{tabular}
\renewcommand{\arraystretch}{1}

\subsection{Intensität bei Kugelwellen}
\begin{tabular}{lll}
					 & \textbf{ohne Dämpfung} & \textbf{mit Dämpfung} \\
	Energieerhaltung & $ \boxed{ I(r) = \frac{P}{4 \, \pi \, r^2} }$ & $ \boxed{ I(r) = \frac{P}{4 \, \pi \, r^2} \, \e^{- \alpha r}}$\\

	Verhältnis 		 & $\boxed{ \frac{I_2}{I_1} = \frac{r_1^2}{r_2^2} }$ & $\boxed{ \frac{I_2}{I_1} = \frac{r_1^2}{r_2^2} \, \e^{- \alpha (r_2 - r_1)} }$ \\

	Pegel 			 & $\boxed{ L_2 = L_1 - 20 \cdot \log \left( \frac{r_2}{r_1} \right)}$ & $\boxed{ L_2 = L_1 - 20 \cdot \log \left( \frac{r_2}{r_1} \right) - K (r_2 - r_1)  }$\\
\end{tabular}

\begin{tabular}{clc}
	$I(r)$ & Intensität & $[I(r)] = \frac{\W}{\m^2}$ \\
	$P$ & Leistung & $[P] = \W$ \\
	$r_i$ & Abstand (Radius) zum Wellenursprung & $[r_i] = \m^2$\\
	$L_i$ & Pegel & $[L_i] = \dB$ \\
	$\alpha$ & Absorptionsgrad ($\alpha = 1-R$, siehe \ref{Reflexionskoeffizient} ) & $[\alpha] = 1$ \\
	$K$ & Dämpfung & $[K] = \frac{\dB}{\m}$
\end{tabular}


\subsection{Verschiedene Schallquellen}

\includegraphics[width=0.9\linewidth]{Bilder/Wellen-Optik/schallquellen} \\

\subsection{PHON-Skala}
\includegraphics[width=0.95\linewidth]{Bilder/Wellen-Optik/phon_skala}



\subsection{Schalldämpfung / Schalldämmung}

\subsubsection{Schalldämpfung}

Schalldämpfung bedeutet eine \textbf{Abschwächung} der Schallwellen durch
\textbf{Absorption}. \\ Schallenergie wird in \textbf{'Wärme' umgewandelt}, d.h. durch die
Absorption von Energie werden das von den Schallwellen
durchdrungene Medium oder die das Schallfeld begrenzenden Körper
\textbf{erwärmt}. \\
\vspace{0.2cm}

\textbf{Verschiedene Arten von Absorption für Schalldämpfung Innen} \\

\begin{itemize}
	\itemsep0em
	\item Poröse Schicht (mit oder ohne perforierte Abdeckung) 
	\item Akustikplatte 
	\item Plattenresonator
	\item Helmholtz-Resonator
\end{itemize}

% \vfill\null
% \columnbreak

\subsubsection{Schalldämmung} 

Schalldämmung ist die \textbf{Behinderung der Schallausbreitung} durch
\textbf{reflektierende} Hindernisse. Mauern, Türen und Fenster bewirken eine
Schalldämmung für den von aussen in das Gebäude eindringenden
Schall. Auch die Ausbreitung von Schall innerhalb eines Gebäudes wird
durch die Schalldämmung von Zwischenwänden und Türen
abgeschwächt. Im Freien wird durch Schallschutzwände eine
Schalldämmung für die dahinterliegenden Gebäude erreicht.



\subsection{Schalldämmung / Schalldämm-Mass}

Bei der Schallübertragung muss zwischen \textbf{Luftschall} und \textbf{Körperschall}
unterschieden werden.

\subsubsection{Luftschalldämmung}

Es gibt Schallquellen, die ihre Schallenergie (fast)
ausschliesslich in die Luft abstrahlen.\\
Beispiele: menschliche Stimme, Geigen, Lautsprecher und Blasinstrumente

\subsubsection{Körperschalldämmung}

Andere Schallerzeuger übertragen die Schallschwingungen nicht nur auf die
Luft, sondern auch direkt auf feste Körper. Streichinstrumente, wie Cello oder
Bassgeige, und Klavier oder Flügel übertragen die Schallschwingungen auch
direkt auf den Fussboden. Wird ein Nagel in die Wand geschlagen, so wird ein
grosser Anteil des erzeugten Schalls als Körperschall übertragen. \\
Beispiele: Trittschall, Wasserleitungsgeräusche \\

% \vfill\null
% \columnbreak



\subsection{Schalldämm-Mass}

$$ \boxed{ \mathcal{R} = 10 \cdot \log\left( \frac{P_1}{P_2} \right)  } $$



\subsection{Anhall / Nachhall}

\subsubsection{Anhall}

Es dauert eine gewisse Zeit, bis sich eine \text{konstante Energiedichte} der Schallwellen \text{im Raum aufgebaut} hat. Dieser Vorgang wird Anhall genannt. Wegen der logarithmischen Empfindlichkeit des Ohrs \text{wird er praktisch nicht wahrgenommen}.

\subsubsection{Nachhall}

Die von den Begrenzungsflächen des Raumes mehrfach reflektierten Wellen bewirken andererseits, dass beim \text{plötzlichen Abschalten} einer Schallquelle der Schall nicht sofort verschwindet, sondern \text{allmählich abklingt}. Dieses Phänomen wird Nachhall genannt und ist im Gegensatz zum Anhall \text{deutlich wahrnehmbar}. \\
\vspace{0.2cm}

\textbf{Per Definition ist die Nachhallzeit jene Zeitspanne, in welcher der Schallpegel im Raum um 60 dB sinkt.}


\subsubsection{Nachhallzeit $T_N$}


$$ \boxed{ T_N = 0.16 \, \frac{\s}{\m} \frac{V}{\sum\limits_i \alpha_i A_i } } $$ \\

% \renewcommand{\arraystretch}{1.3}
\begin{tabular}{clc}
$T_N$ & Nachhallzeit & $[T_N] = \s$ \\
$V$ & Raumvolumen & $[V] = \m^3$ \\
$\alpha_i$ & Absorptionsgrad & $[\alpha_i] = 1$ \\
$A_i$ & Teilfläche der Raumbegrenzung & $[A_i] = \m^2$ \\ 
& mit Absorption $\alpha_i$ \\
\end{tabular}
% \renewcommand{\arraystretch}{1}

% \vfill\null
% \columnbreak

		\section{Anhang}

% Massenträgheitstabelle in Physik I eingepflanzt

\subsection{Messunsicherheit}

Abhängigkeit der Messgrösse: $ y = y(x_1,x_2,...,x_n) $

Unsicherheit der Messgrösse: \( \Delta y = \sqrt{\sum_{i} \left( \frac{\partial y}{\partial x_i}\right)^2 \Delta x_i^2} \)

Beispiel: $ f = \frac{c}{\lambda} $

\begin{align*}
	\Delta f &= \sqrt{\left( \frac{\partial f}{\partial c}\right)^2 \Delta c^2 + \left( \frac{\partial f}{\partial \lambda}\right)^2 \Delta \lambda^2} \\
	         &= \sqrt{\left( \frac{1}{\lambda}\right)^2 \Delta c^2 + \left( - \frac{c}{\lambda^2}\right)^2 \Delta \lambda^2} 
\end{align*}



\subsection{Trigonometrie}
		
		% TODO: add arcsin, ...
		\definecolor{tablebordercolor}{HTML}{D7D7D7}% adjust to taste

		{\tabcolsep=0.05cm% might need adjusting
			\resizebox{\columnwidth}{!}{%
				\begin{tabular}{cccccccccccccccccc}
				$\alpha$                                                     \begin{tabular}{c}\\\\\end{tabular}  & \cellcolor{tablebordercolor}0         & \cellcolor{tablebordercolor}$\frac{\pi}{6}$   & \cellcolor{tablebordercolor}$\frac{\pi}{4}$   & \cellcolor{tablebordercolor}$\frac{\pi}{3}$   & \cellcolor{tablebordercolor}$\frac{\pi}{2}$ & \cellcolor{tablebordercolor}$\frac{2\pi}{3}$   & \cellcolor{tablebordercolor}$\frac{3\pi}{4}$   & \cellcolor{tablebordercolor}$\frac{5\pi}{6}$   & \cellcolor{tablebordercolor}$\pi$     & \cellcolor{tablebordercolor}$\frac{7\pi}{6}$   & \cellcolor{tablebordercolor}$\frac{5\pi}{4}$   & \cellcolor{tablebordercolor}$\frac{4\pi}{3}$   & \cellcolor{tablebordercolor}$\frac{3\pi}{2}$ & \cellcolor{tablebordercolor}$\frac{5\pi}{3}$   & \cellcolor{tablebordercolor}$\frac{7\pi}{4}$   & \cellcolor{tablebordercolor}$\frac{11\pi}{6}$  & \cellcolor{tablebordercolor}$2\pi$    \\ \cline{2-18} 
				\multicolumn{1}{c|}{\cellcolor{tablebordercolor}{$\alpha$°}  \begin{tabular}{c}\\\\\end{tabular}} & \multicolumn{1}{c|}{0°}               & \multicolumn{1}{c|}{30°}                      & \multicolumn{1}{c|}{45°}                      & \multicolumn{1}{c|}{60°}                      & \multicolumn{1}{c|}{90°}                    & \multicolumn{1}{c|}{120°}                      & \multicolumn{1}{c|}{135°}                      & \multicolumn{1}{c|}{150°}                      & \multicolumn{1}{c|}{180°}             & \multicolumn{1}{c|}{210°}                      & \multicolumn{1}{c|}{225°}                      & \multicolumn{1}{c|}{240°}                      & \multicolumn{1}{c|}{270°}                    & \multicolumn{1}{c|}{300°}                      & \multicolumn{1}{c|}{315°}                      & \multicolumn{1}{c|}{330°}                      & \multicolumn{1}{c|}{360°}             \\ \cline{2-18} 
				\multicolumn{1}{c|}{\cellcolor{tablebordercolor}{sin}        \begin{tabular}{c}\\\\\end{tabular}} & \multicolumn{1}{c|}{$0$}              & \multicolumn{1}{c|}{$\frac{1}{2}$}            & \multicolumn{1}{c|}{$\frac{\sqrt{2}}{2}$}     & \multicolumn{1}{c|}{$\frac{\sqrt{3}}{2}$}     & \multicolumn{1}{c|}{$1$}                    & \multicolumn{1}{c|}{$\frac{\sqrt{3}}{2}$}      & \multicolumn{1}{c|}{$\frac{\sqrt{2}}{2}$}      & \multicolumn{1}{c|}{$\frac{1}{2}$}             & \multicolumn{1}{c|}{$0$}              & \multicolumn{1}{c|}{$-\frac{1}{2}$}            & \multicolumn{1}{c|}{$-\frac{\sqrt{2}}{2}$}     & \multicolumn{1}{c|}{$-\frac{\sqrt{3}}{2}$}     & \multicolumn{1}{c|}{$-1$}                    & \multicolumn{1}{c|}{$-\frac{\sqrt{3}}{2}$}     & \multicolumn{1}{c|}{$-\frac{\sqrt{2}}{2}$}     & \multicolumn{1}{c|}{$-\frac{1}{2}$}            & \multicolumn{1}{c|}{$0$}              \\ \cline{2-18} 
				\multicolumn{1}{c|}{\cellcolor{tablebordercolor}{cos}        \begin{tabular}{c}\\\\\end{tabular}} & \multicolumn{1}{c|}{$1$}              & \multicolumn{1}{c|}{$\frac{\sqrt{3}}{2}$}     & \multicolumn{1}{c|}{$\frac{\sqrt{2}}{2}$}     & \multicolumn{1}{c|}{$\frac{1}{2}$}            & \multicolumn{1}{c|}{$0$}                    & \multicolumn{1}{c|}{$-\frac{1}{2}$}            & \multicolumn{1}{c|}{$-\frac{\sqrt{2}}{2}$}     & \multicolumn{1}{c|}{$-\frac{\sqrt{3}}{2}$}     & \multicolumn{1}{c|}{$-1$}             & \multicolumn{1}{c|}{$-\frac{\sqrt{3}}{2}$}     & \multicolumn{1}{c|}{$-\frac{\sqrt{2}}{2}$}     & \multicolumn{1}{c|}{$-\frac{1}{2}$}            & \multicolumn{1}{c|}{$0$}                     & \multicolumn{1}{c|}{$\frac{1}{2}$}             & \multicolumn{1}{c|}{$\frac{\sqrt{2}}{2}$}      & \multicolumn{1}{c|}{$\frac{\sqrt{3}}{2}$}      & \multicolumn{1}{c|}{$1$}              \\ \cline{2-18} 
				\multicolumn{1}{c|}{\cellcolor{tablebordercolor}{tan}        \begin{tabular}{c}\\\\\end{tabular}} & \multicolumn{1}{c|}{$0$}              & \multicolumn{1}{c|}{$\frac{1}{\sqrt{3}}$}     & \multicolumn{1}{c|}{$1$}                      & \multicolumn{1}{c|}{$\sqrt{3}$}               & \multicolumn{1}{c|}{$\tilde{\infty}$}       & \multicolumn{1}{c|}{$-\sqrt{3}$}               & \multicolumn{1}{c|}{$-1$}                      & \multicolumn{1}{c|}{$-\frac{1}{\sqrt{3}}$}     & \multicolumn{1}{c|}{$0$}              & \multicolumn{1}{c|}{$\frac{1}{\sqrt{3}}$}      & \multicolumn{1}{c|}{$1$}                       & \multicolumn{1}{c|}{$\sqrt{3}$}                & \multicolumn{1}{c|}{$\tilde{\infty}$}        & \multicolumn{1}{c|}{$-\sqrt{3}$}               & \multicolumn{1}{c|}{$-1$}                      & \multicolumn{1}{c|}{$-\frac{1}{\sqrt{3}}$}     & \multicolumn{1}{c|}{$0$}              \\ \cline{2-18} 
				\multicolumn{1}{c|}{\cellcolor{tablebordercolor}{cot}        \begin{tabular}{c}\\\\\end{tabular}} & \multicolumn{1}{c|}{$\tilde{\infty}$} & \multicolumn{1}{c|}{$\sqrt{3}$}               & \multicolumn{1}{c|}{$1$}                      & \multicolumn{1}{c|}{$\frac{1}{\sqrt{3}}$}     & \multicolumn{1}{c|}{$0$}                    & \multicolumn{1}{c|}{$-\frac{1}{\sqrt{3}}$}     & \multicolumn{1}{c|}{$-1$}                      & \multicolumn{1}{c|}{$-\sqrt{3}$}               & \multicolumn{1}{c|}{$\tilde{\infty}$} & \multicolumn{1}{c|}{$\sqrt{3}$}                & \multicolumn{1}{c|}{$1$}                       & \multicolumn{1}{c|}{$\frac{1}{\sqrt{3}}$}      & \multicolumn{1}{c|}{$0$}                     & \multicolumn{1}{c|}{$-\frac{1}{\sqrt{3}}$}     & \multicolumn{1}{c|}{$-1$}                      & \multicolumn{1}{c|}{$-\sqrt{3}$}               & \multicolumn{1}{c|}{$\tilde{\infty}$} \\ \cline{2-18} 
				\end{tabular}%
			}
		}
		
		\subsubsection{Beziehungen zwischen $\sin(x)$ und $\cos(x)$}
		\begin{tabular}{ll}
		$\sin(-a) = -\sin(a)$ & $\cos(-a) = \cos(a)$ \\
		$\sin(\pi - a) = \sin(a)$ & $\cos(\pi -a) = - \cos(a)$ \\		
		$\sin(\pi + a) = -\sin(a)$ & $\cos(\pi + a) = - \cos(a)$ \\	
		$\sin(\frac{\pi}{2} - a) = \sin(\frac{\pi}{2} + a) = \cos(a)$ \\
		$\cos(\frac{\pi}{2} - a) = - \cos(\frac{\pi}{2} + a) = \sin(a)$\\		 
		\end{tabular}
		
		
		% \vfill\null
		% \columnbreak

		\subsubsection{Additionstheoreme}
		
		$\sin(a \pm b) = \sin(a) \cdot \cos(b) \pm \cos(a) \cdot \sin(b)$ \\
		$\cos(a \pm b) = \cos(a) \cdot \cos(b) \mp \sin(a) \cdot \sin(b) $ \\
		$\tan(a \pm b) = \frac{\tan(a) \pm \tan(b)}{1 \mp \tan(a) \cdot \tan(b)}$
		
	
		\subsubsection{Summen und Differenzen}
		
		$\sin(a) + \sin(b) = 2 \cdot \sin \big( \frac{a+b}{2} \big) \cdot \cos \big( \frac{a-b}{2} \big)$ \\
		$\sin(a) - \sin(b) = 2 \cdot \sin \big( \frac{a-b}{2} \big) \cdot \cos \big( \frac{a+b}{2} \big)$ \\
		$\cos(a) + \cos(b) = 2 \cdot \cos \big( \frac{a+b}{2} \big) \cdot \cos \big( \frac{a-b}{2} \big)$ \\
		$\cos(a) - \cos(b) = -2 \cdot \sin \big( \frac{a+b}{2} \big) \cdot \sin \big( \frac{a-b}{2} \big)$ \\
		$\tan(a) \pm \tan(b) = \frac{\sin(a \pm b)}{\cos(a) \cdot \cos(b)}$ 
		
			
		
		\subsubsection{Produkte}
		$\sin(a) \cdot \sin(b) = \frac{1}{2} \big( \cos(a-b) - \cos(a+b) \big) $ \\
		$\cos(a) \cdot \cos(b) = \frac{1}{2} \big( \cos(a-b) + \cos(a+b) \big) $ \\	
		$\sin(a) \cdot \cos(b) = \frac{1}{2} \big( \sin(a-b) + \sin(a+b) \big) $ 
			
			
		\subsubsection{Winkelvielfache und Halbwinkel}
		$\sin(2a) = 2 \, \sin(a) \cdot \cos(a)$ \\
		$\sin(3a) = 3 \, \sin(a) -4 \, \sin^3(a)$ \\ 
		$\sin(4a) = 8 \, \cos^3(a) \cdot \sin(a) - 4 \, \cos(a) \cdot \sin(a)$ \\
		\vspace{0.2cm}
		$\cos(2a) = \cos^2(a) - \sin^2(a)$ \\
		$\cos(3a) = 4 \, \cos^3(a) -3 \, \cos(a)$ \\ 
		$\cos(4a) = 8 \, \cos^4(a) - 8 \, \cos^2(a) + 1$ \\
		\vspace{0.2cm}
		$\sin \big( \frac{a}{2} \big) = \sqrt{\frac{1}{2}(1-\cos(a))} $ \qquad $\cos \big( \frac{a}{2} \big) = \sqrt{\frac{1}{2}(1+\cos(a))} $ 
		
		
		\subsubsection{Potenzen}
		$\sin^2(a) = \frac{1}{2}(1-\cos(2a))$ \\	
		$\sin^3(a) = \frac{1}{4}(3\, \sin(a) - \sin(3a))$ \\	
		$\sin^4(a) = \frac{1}{8}(\cos(4a) -4 \, \cos(2a) +3)$ \\	
		\vspace{0.2cm}
		$\cos^2(a) = \frac{1}{2}(1+\cos(2a))$ \\	
		$\cos^3(a) = \frac{1}{4}(\cos(3a) + 3 \, \cos(a)$ \\	
		$\cos^4(a) = \frac{1}{8}(\cos(4a) +4 \, \cos(2a) +3)$ 

		% \section{Vektorrechnung}
		
	\subsection{Betrag eines Vektors}
	$$ \vert \vec{A} \vert =  \sqrt{A_x^2 + A_y^2 + A_z^2}$$
		
		
	\subsection{Gleichheit zweier Vektoren}
	Zwei Vektoren sind gleich, wenn alle Komponenten identisch sind: \\
	\\
	\begin{tabular}{ll}
	$\bullet$ & $A_x = B_x$\\
	$\bullet$ & $A_y = B_y$\\
	$\bullet$ & $A_z = B_z$\\
	\end{tabular}		
		

		
		
	\subsection{Negative eines Vektors}
	\begin{minipage}{0.6\linewidth}
	\begin{tikzpicture}
		[
		x=1cm, y=1cm, scale=0.65, font=\footnotesize, >=latex 
		%Voreinstellung für Pfeilspitzen
		]
		
		%Raster links im Hintergrund
		\draw[step=1, gray, very thin] (0,0) grid (3.25,2.25);
		
		%Länge x Achse
		\draw [-latex] (0,0) -- ++(3.5,0) node[below] {$x$};
		
		%Länge y Achse
		\draw [-latex] (0,0) -- ++(0,2.5) node[left] {$y$};	
		
		%Vektor a
		\draw[-latex, red, very thick] (0.5,0.5) -- ++(2,1) node [midway, above] {$\vec{a}$} node (a) {}; 
		
		\draw[dashed] (a.center) ++ (-3,0) node (c) {};
		
		%Raster links im Hintergrund
		\draw[step=1, gray, very thin] (4,0) grid (7.25,2.25);
		
		%Länge x Achse
		\draw [-latex] (4,0) -- ++(3.5,0) node[below] {$x$};
		
		%Länge y Achse
		\draw [-latex] (4,0) -- ++(0,2.5) node[left] {$y$};	
		
		%Vektor a
		\draw[-latex, red, very thick] (6.5,1.5) -- ++(-2,-1) node [midway, above] {$\vec{b}$} node (a) {}; 
		
	\end{tikzpicture}
	\end{minipage}
	\hfill
	\begin{minipage}{0.3\linewidth}
	$b_x = - a_x$ \\
	\\
	$b_y = - a_y$ \\
	\\
	$b_z = - a_z$ \\
	\end{minipage}	
		
		
	
	\subsection{Addition zweier Vektoren}
	
	\begin{minipage}{0.6\linewidth}
	\begin{tikzpicture}
		[
		x=1cm, y=1cm, scale=0.7, font=\footnotesize, >=latex 
		%Voreinstellung für Pfeilspitzen
		]
		
		%Raster im Hintergrund
		\draw[step=1, gray, very thin] (0,0) grid (5.5,3.5);
		
		%Zahlen auf x-Achse
		\foreach \x in {0,1,2,3,4,5}
		\draw[shift={(\x,0)},color=black] (0pt,2pt) -- (0pt,-2pt) node[below]
		{\footnotesize $\x$};
		%Länge x Achse
		\draw [-latex] (0,0) -- ++(5.5,0) node[below] {$x$};
		%Länge y Achse
		\draw [-latex] (0,0) -- ++(0,3.5) node[left] {$y$};
		
		%Zahlen auf y-Achse
		%\foreach \y in {0,...,1}
		\foreach \y in {0,1,2,3}
		\draw[shift={(0,\y)},color=black] (2pt,0pt) -- (-2pt,0pt) node[left]
		{\footnotesize $\y$};		
		
		%Vektor a
		\draw[-latex, thick] (0,0) -- (2,3) node [midway, above] {$\vec{a}$} node (a) {}; 
		%Vektor b
		\draw[-latex, thick] (0,0) -- (3,0) node [midway, above] {$\vec{b}$} node (b) {}; 
		\draw[dashed] (a.center) -- ++ (3,0) node (c) {};
		\draw[dashed] (b.center) -- ++ (2,3);
		\draw[very thick, red, -latex] (0,0) -- (c.center) node [midway, above] {$\vec{c}$};
	\end{tikzpicture}
	\end{minipage}
	\hfill
	\begin{minipage}{0.3\linewidth}
	$c_x = a_x + b_x$ \\
	\\
	$c_y = a_y + b_y$ \\
	\\
	$c_z = a_z + b_z$ \\
	\end{minipage}
	
	
		
		
		
	\subsection{Subtraktion zweier Vektoren}
	
	\begin{minipage}{0.6\linewidth}
	\begin{tikzpicture}
		[
		x=1cm, y=1cm, scale=0.8, font=\footnotesize, >=latex 
		%Voreinstellung für Pfeilspitzen
		]
		
		%Raster im Hintergrund
		\draw[step=1, gray, very thin] (-0.5,-0.5) grid (6.5,2.5);
		
		%Zahlen auf x-Achse
		%\foreach \x in {0,1,2,3}
		%\draw[shift={(\x,0)},color=black] (0pt,2pt) -- (0pt,-2pt) node[below]
		%{\footnotesize $\x$};
		
		%Länge x Achse
		\draw [-latex] (0,0) -- (6.5,0) node[below] {$x$};
		
		%Länge y Achse
		\draw [-latex] (0,0) -- ++(0,2.5) node[left] {$y$};	
		
		%Vektor a
		\draw[-latex, very thick] (0,0) -- (3,2) node [midway, above] {$\vec{a}$} node (a) {}; 
		
		%Vektor b
		\draw[-latex, dashed, very thick] (3,2) -- ++(3,0) node [midway, above] {$\vec{b}$} node (b) {};
		\draw[-latex, very thick, blue] (3,2) -- ++(-3,0) node [midway, above] {$-\vec{b}$} node (-b) {}; 
		\draw[dashed] (a.center) ++ (-3,0) node (c) {};
		
		%\draw[dashed] (-b.center) -- ++ (2,3);
		%\draw[dashed] (b.center) -- ++ (-1,3);
		\draw[very thick, red, -latex] (0,0) -- (c.center) node [midway, left] {$\vec{c}$};
		
		%Zahlen auf y-Achse 
		%\foreach \y in {0,...,1}
		%\foreach \y in {1,2,3}
		%\draw[shift={(0,\y)},color=black] (2pt,0pt) -- (-2pt,0pt) node[left]
		%{\footnotesize $\y$};	
	\end{tikzpicture}
	\end{minipage}
	\hfill
	\begin{minipage}{0.3\linewidth}
	$c_x = a_x - b_x$ \\
	\\
	$c_y = a_y - b_y$ \\
	\\
	$c_z = a_z - b_z$ \\
	\end{minipage}	
		
		
	\subsection{Multiplikation eines Vektros mit einem Skalar}
	\begin{minipage}{0.68\linewidth}
	$$\vec{b} = s \, \vec{a} \quad \vert \vec{B} \vert = \vert s \vert \cdot \vert \vec{a} \vert$$ \\
	\end{minipage}
	\hfill
	\begin{minipage}{0.3\linewidth}
	$b_x = s \cdot a_x$ \\
	\\
	$b_y = s \cdot a_y$ \\
	\\
	$b_z = s \cdot a_y$ \\
	\end{minipage}	
	
		
		
		
		
	\subsection{Skalarprodukt}
	$$\vec{c} = \vec{a} \bullet \vec{b} = \vert \vec{a} \vert \cdot \vert \vec{b} \, \vert  \, \cos(\varphi)$$ 
		
		
		
		
		
	\subsection{Kreuzprodukt (nur in 3D)}
	 $$\vec{a} \times \vec{b} = \begin{pmatrix} a_2 b_3 - a_3 b_2 \\ -(a_1 b_3 - a_3 b_1) \\ a_1 b_2 - a_2 b_1 \end{pmatrix}$$ 
		
				


		% \section{Statistik}
	\subsection{Arithmetisches Mittel $\overline{x}_{arith}$}
		
	$\overline{x}_{arith} := \frac{1}{N} \sum \limits_{i = 1}^N  x_i$
	
		
		
	\subsection{Geometrisches Mittel $\overline{x}_{geom}$}
	\textbf{Nur für positive Zahlenreihen $x_i$ definiert!} \\

	$\overline{x}_{geom} := \sqrt[N]{\prod \limits_{i = 1}^N  x_i }$ \qquad \qquad $\Rightarrow \overline{x}_{geom} \leq \overline{x}_{arith}$
		
		
		
	\subsection{Quadratisches Mittel QMW (RMS)}
	\textbf{Wechselstromtechnik; Effektivwert} \\

	$QMW := \sqrt{ \frac{1}{N} \sum \limits_{i = 1}^N  x_i^2 }$
		
		
	\subsection{Harmonisches Mittel $\overline{x}_{harm}$}		
	
	$\overline{x}_{harm} := \frac{N}{ \sum \limits_{i = 1}^N   \frac{1}{x_i} } $ \\

	Kann sinnvoll eingesetzt werden, wenn man für die $i-te$ Teilstrecke $s_i$ eine Zeit $t_i$ benötigt (also eine Durchschnittsgeschwindigkeit von $v_i = \frac{s_i}{t_i}$ und eine Durchschnittsgeschwindigkeit über $N$ Teilstrecken ermitteln will: \\

	$\overline{v}_{harm} = \frac{\sum \limits_{i = 1}^N  s_i }{\sum \limits_{i = 1}^N  t_i} = \frac{\sum \limits_{i = 1}^N  s_i }{\sum \limits_{i = 1}^N \frac{s_i}{v_i} }  $ \qquad gewichtetes harm. Mittel
		
		
		
	\subsection{Standardabweichung $\sigma$}	
		
	\begin{tabular}{ll}
	Varianz: & $\sigma^2 := \frac{1}{N} \sum \limits_{i = 1}^N (x_i - \overline{x}_{arith} )^2 $ \\

	Standardabweichung: &  $\sigma := \sqrt{ \frac{1}{N} \sum \limits_{i = 1}^N (x_i - \overline{x}_{arith} )^2 }$ \\
	\end{tabular}
		
		
		
	\subsection{Standardabweichung des Mittelwerts}	
	\textbf{Gilt nur, wenn eine Normalverteilung vorliegt!} \\
	Beschreibt nur statistische Fehler \\

	$\sigma(\overline{x}_{arith}) = \frac{\sigma}{\sqrt{N}} $
		
		

		% \section{Mathematik-Hilfe}
		
	\subsection{Trigonometrie}
	\begin{tabular}{| c | c | c |}
	\hline
	Sinus & Cosinus & Tangens \\ 
	\hline 
	\rule{0pt}{10pt} $\frac{GK}{H}$ & $\frac{AK}{H}$ & $\frac{AK}{GK}$ \\ 
	\hline
	\end{tabular} 
		
		
		
	\subsection{Schwerpunkt}
	Die Koordinaten des Schwerpunkts müssen komponentenweise berechnet werden: \\

	$x_s = \frac{\sum  x_i \cdot m_i}{M}$ \qquad	$y_s = \frac{\sum  y_i \cdot m_i}{M}$ \qquad $z_s = \frac{\sum  z_i \cdot m_i}{M}$ \\

	\begin{tabular}{ll}
	$x_s \; , y_s \; , z_s$ & Koordinaten des Schwerpunkts \\
	$x_i \; , y_i \; , z_i$ & Koordinaten von kleinen Massepunkten \\
	$m_i$ & Kleine Massepunkte an entsprechenden Koordinaten \\
	$M$ & Gesamtmasse des Körpers
	\end{tabular}
	
	\subsection{Polarkoordinaten (Kreisbewegung)}

	
	\begin{tabular}{ll}
	polar $\rightarrow$ kartesisch & $\vec{P} =  \begin{pmatrix} x \\ y \end{pmatrix} =  \begin{pmatrix} r \cdot \cos(\phi)  \\ r \cdot \sin(\phi)  \end{pmatrix}$ \\
	\\
	kartesisch $\rightarrow$ polar & $\vec{P} =  \begin{pmatrix} r \\ \phi \end{pmatrix} =  \begin{pmatrix} \sqrt{x^2 + y^2}  \\ \tan \big( \frac{x}{y} \big)   \end{pmatrix}$\\
	\end{tabular}
	
	
	
	\subsection{Ableitungsregeln S .445-448}
			
			\subsubsection{Elementare Regeln}
			\begin{tabular}{lll}
			Potenzen: & $f(x) = x^3$ & $f'(x) = 3 \, x^2$ \\
			& $f(x) = x^\alpha$ & $f'(x) = \alpha \cdot x^{\alpha - 1}$ \\
			\\
			Linearität: & $f(x) = c \cdot x^2$ & $f'(x) = c \cdot 2 \, x $ \\
			\\
			Summe: & $(u(x) + v(x) - w(x))' $ & = $u'(x) + v'(x) - w'(x)$ \\
			\\
			Konstanten: & $c = const$ $\rightarrow$ $c' = 0$ \\
			\end{tabular}
			
			
			\subsubsection{Produktregel}
			$(f(x) \cdot g(x))' = f'(x) \cdot g(x) + f(x) \cdot g'(x)$ 
			
			\subsubsection{Quotientenregel}
			$\left( \frac{u(x)}{v(x)} \right) ' = \frac{u'(x) \cdot v(x) - u(x) \cdot v'(x)}{v(x) ^2}$ \quad $\rightarrow$ als Produkt schreiben \\

			$u(x) \cdot \left( \frac{1)}{v(x)} \right) ' =  u'(x) \cdot \frac{1}{v(x)} + u(x) \cdot \frac{- v'(x)}{v(x)^2}$
			
			\subsubsection{Kettenregel}
			$g(f(x))' =  f'(x) \cdot g'(x)$ \\
			
			\subsubsection{Umkehrfunktion}
			$(f^{-1}(y_0))' = \frac{1}{f'(x_0)} =  \frac{1}{f'(f^{-1}(y_0))}$ \\
			

			
			\subsection{Allgemeine Logarithmus-Ableitung}
			$(\log_b(x))' = \left( \frac{\ln(x)}{\ln(b)} \right)' = \frac{1}{\ln(b)} \cdot (\ln(x))' = \frac{1}{\ln(b)} \cdot \frac{1}{x} $
			
			
			
			
			\subsection{Integrationsregeln S. 494-496}
		Linearität: $\int \limits_{a}^{b} \alpha \, f(x) \, dx = \alpha \int \limits_{a}^{b} f(x) \, dx$ 
		
		
		\subsubsection{Rechenregeln mit Integralen S. 508-510}
		\begin{tabular}{ll}
		Zerlegung: & $\int \limits_{a}^{b} f_1(x) \, dx + f_2(x) \, dx = \int \limits_{a}^{b} f_1(x) \, dx + \int \limits_{a}^{b} f_2(x) \, dx$ \\
		\\		
		& $\int \limits_{a}^{c} f(x) \, dx = \int \limits_{a}^{b} f(x) \, dx + \int \limits_{b}^{c} f(x) \, dx$ \\
		\\
		Grenzen tauschen: & $ \int \limits_{a}^{b} f(x) \, dx = -  \int \limits_{b}^{a} f(x) \, dx $ \\
		\\
		Gleiche Grenzen: &  $\int \limits_{a}^{a} f(x) \, dx = 0$ \\
		\end{tabular}
		
		
		\subsection{Wichtige Integrale S. 495}
		\begin{tabular}{ll}
		$\int \limits_{a}^{b} x^2  \, dx = \frac{b^3}{3} - \frac{a^3}{3}$ & $\int \limits_{a}^{b} x \, dx = \frac{b^2}{2} - \frac{a^2}{2}$ \\
		\\
		$\int \limits_{a}^{b} 1 \, dx = b - a $ (Rechteck)& \\
		\end{tabular}
			
	
	
		
	\end{multicols*}	
\end{document}