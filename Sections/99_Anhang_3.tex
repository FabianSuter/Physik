\section{Anhang}

% Massenträgheitstabelle in Physik I eingepflanzt

\subsection{Messunsicherheit}

Abhängigkeit der Messgrösse: $ y = y(x_1,x_2,...,x_n) $

Unsicherheit der Messgrösse: \( \Delta y = \sqrt{\sum_{i} \left( \frac{\partial y}{\partial x_i}\right)^2 \Delta x_i^2} \)

Beispiel: $ f = \frac{c}{\lambda} $

\begin{align*}
	\Delta f &= \sqrt{\left( \frac{\partial f}{\partial c}\right)^2 \Delta c^2 + \left( \frac{\partial f}{\partial \lambda}\right)^2 \Delta \lambda^2} \\
	         &= \sqrt{\left( \frac{1}{\lambda}\right)^2 \Delta c^2 + \left( - \frac{c}{\lambda^2}\right)^2 \Delta \lambda^2} 
\end{align*}



\subsection{Trigonometrie}
		
		% TODO: add arcsin, ...
		\definecolor{tablebordercolor}{HTML}{D7D7D7}% adjust to taste

		{\tabcolsep=0.05cm% might need adjusting
			\resizebox{\columnwidth}{!}{%
				\begin{tabular}{cccccccccccccccccc}
				$\alpha$                                                     \begin{tabular}{c}\\\\\end{tabular}  & \cellcolor{tablebordercolor}0         & \cellcolor{tablebordercolor}$\frac{\pi}{6}$   & \cellcolor{tablebordercolor}$\frac{\pi}{4}$   & \cellcolor{tablebordercolor}$\frac{\pi}{3}$   & \cellcolor{tablebordercolor}$\frac{\pi}{2}$ & \cellcolor{tablebordercolor}$\frac{2\pi}{3}$   & \cellcolor{tablebordercolor}$\frac{3\pi}{4}$   & \cellcolor{tablebordercolor}$\frac{5\pi}{6}$   & \cellcolor{tablebordercolor}$\pi$     & \cellcolor{tablebordercolor}$\frac{7\pi}{6}$   & \cellcolor{tablebordercolor}$\frac{5\pi}{4}$   & \cellcolor{tablebordercolor}$\frac{4\pi}{3}$   & \cellcolor{tablebordercolor}$\frac{3\pi}{2}$ & \cellcolor{tablebordercolor}$\frac{5\pi}{3}$   & \cellcolor{tablebordercolor}$\frac{7\pi}{4}$   & \cellcolor{tablebordercolor}$\frac{11\pi}{6}$  & \cellcolor{tablebordercolor}$2\pi$    \\ \cline{2-18} 
				\multicolumn{1}{c|}{\cellcolor{tablebordercolor}{$\alpha$°}  \begin{tabular}{c}\\\\\end{tabular}} & \multicolumn{1}{c|}{0°}               & \multicolumn{1}{c|}{30°}                      & \multicolumn{1}{c|}{45°}                      & \multicolumn{1}{c|}{60°}                      & \multicolumn{1}{c|}{90°}                    & \multicolumn{1}{c|}{120°}                      & \multicolumn{1}{c|}{135°}                      & \multicolumn{1}{c|}{150°}                      & \multicolumn{1}{c|}{180°}             & \multicolumn{1}{c|}{210°}                      & \multicolumn{1}{c|}{225°}                      & \multicolumn{1}{c|}{240°}                      & \multicolumn{1}{c|}{270°}                    & \multicolumn{1}{c|}{300°}                      & \multicolumn{1}{c|}{315°}                      & \multicolumn{1}{c|}{330°}                      & \multicolumn{1}{c|}{360°}             \\ \cline{2-18} 
				\multicolumn{1}{c|}{\cellcolor{tablebordercolor}{sin}        \begin{tabular}{c}\\\\\end{tabular}} & \multicolumn{1}{c|}{$0$}              & \multicolumn{1}{c|}{$\frac{1}{2}$}            & \multicolumn{1}{c|}{$\frac{\sqrt{2}}{2}$}     & \multicolumn{1}{c|}{$\frac{\sqrt{3}}{2}$}     & \multicolumn{1}{c|}{$1$}                    & \multicolumn{1}{c|}{$\frac{\sqrt{3}}{2}$}      & \multicolumn{1}{c|}{$\frac{\sqrt{2}}{2}$}      & \multicolumn{1}{c|}{$\frac{1}{2}$}             & \multicolumn{1}{c|}{$0$}              & \multicolumn{1}{c|}{$-\frac{1}{2}$}            & \multicolumn{1}{c|}{$-\frac{\sqrt{2}}{2}$}     & \multicolumn{1}{c|}{$-\frac{\sqrt{3}}{2}$}     & \multicolumn{1}{c|}{$-1$}                    & \multicolumn{1}{c|}{$-\frac{\sqrt{3}}{2}$}     & \multicolumn{1}{c|}{$-\frac{\sqrt{2}}{2}$}     & \multicolumn{1}{c|}{$-\frac{1}{2}$}            & \multicolumn{1}{c|}{$0$}              \\ \cline{2-18} 
				\multicolumn{1}{c|}{\cellcolor{tablebordercolor}{cos}        \begin{tabular}{c}\\\\\end{tabular}} & \multicolumn{1}{c|}{$1$}              & \multicolumn{1}{c|}{$\frac{\sqrt{3}}{2}$}     & \multicolumn{1}{c|}{$\frac{\sqrt{2}}{2}$}     & \multicolumn{1}{c|}{$\frac{1}{2}$}            & \multicolumn{1}{c|}{$0$}                    & \multicolumn{1}{c|}{$-\frac{1}{2}$}            & \multicolumn{1}{c|}{$-\frac{\sqrt{2}}{2}$}     & \multicolumn{1}{c|}{$-\frac{\sqrt{3}}{2}$}     & \multicolumn{1}{c|}{$-1$}             & \multicolumn{1}{c|}{$-\frac{\sqrt{3}}{2}$}     & \multicolumn{1}{c|}{$-\frac{\sqrt{2}}{2}$}     & \multicolumn{1}{c|}{$-\frac{1}{2}$}            & \multicolumn{1}{c|}{$0$}                     & \multicolumn{1}{c|}{$\frac{1}{2}$}             & \multicolumn{1}{c|}{$\frac{\sqrt{2}}{2}$}      & \multicolumn{1}{c|}{$\frac{\sqrt{3}}{2}$}      & \multicolumn{1}{c|}{$1$}              \\ \cline{2-18} 
				\multicolumn{1}{c|}{\cellcolor{tablebordercolor}{tan}        \begin{tabular}{c}\\\\\end{tabular}} & \multicolumn{1}{c|}{$0$}              & \multicolumn{1}{c|}{$\frac{1}{\sqrt{3}}$}     & \multicolumn{1}{c|}{$1$}                      & \multicolumn{1}{c|}{$\sqrt{3}$}               & \multicolumn{1}{c|}{$\tilde{\infty}$}       & \multicolumn{1}{c|}{$-\sqrt{3}$}               & \multicolumn{1}{c|}{$-1$}                      & \multicolumn{1}{c|}{$-\frac{1}{\sqrt{3}}$}     & \multicolumn{1}{c|}{$0$}              & \multicolumn{1}{c|}{$\frac{1}{\sqrt{3}}$}      & \multicolumn{1}{c|}{$1$}                       & \multicolumn{1}{c|}{$\sqrt{3}$}                & \multicolumn{1}{c|}{$\tilde{\infty}$}        & \multicolumn{1}{c|}{$-\sqrt{3}$}               & \multicolumn{1}{c|}{$-1$}                      & \multicolumn{1}{c|}{$-\frac{1}{\sqrt{3}}$}     & \multicolumn{1}{c|}{$0$}              \\ \cline{2-18} 
				\multicolumn{1}{c|}{\cellcolor{tablebordercolor}{cot}        \begin{tabular}{c}\\\\\end{tabular}} & \multicolumn{1}{c|}{$\tilde{\infty}$} & \multicolumn{1}{c|}{$\sqrt{3}$}               & \multicolumn{1}{c|}{$1$}                      & \multicolumn{1}{c|}{$\frac{1}{\sqrt{3}}$}     & \multicolumn{1}{c|}{$0$}                    & \multicolumn{1}{c|}{$-\frac{1}{\sqrt{3}}$}     & \multicolumn{1}{c|}{$-1$}                      & \multicolumn{1}{c|}{$-\sqrt{3}$}               & \multicolumn{1}{c|}{$\tilde{\infty}$} & \multicolumn{1}{c|}{$\sqrt{3}$}                & \multicolumn{1}{c|}{$1$}                       & \multicolumn{1}{c|}{$\frac{1}{\sqrt{3}}$}      & \multicolumn{1}{c|}{$0$}                     & \multicolumn{1}{c|}{$-\frac{1}{\sqrt{3}}$}     & \multicolumn{1}{c|}{$-1$}                      & \multicolumn{1}{c|}{$-\sqrt{3}$}               & \multicolumn{1}{c|}{$\tilde{\infty}$} \\ \cline{2-18} 
				\end{tabular}%
			}
		}
		
		\subsubsection{Beziehungen zwischen $\sin(x)$ und $\cos(x)$}
		\begin{tabular}{ll}
		$\sin(-a) = -\sin(a)$ & $\cos(-a) = \cos(a)$ \\
		$\sin(\pi - a) = \sin(a)$ & $\cos(\pi -a) = - \cos(a)$ \\		
		$\sin(\pi + a) = -\sin(a)$ & $\cos(\pi + a) = - \cos(a)$ \\	
		$\sin(\frac{\pi}{2} - a) = \sin(\frac{\pi}{2} + a) = \cos(a)$ \\
		$\cos(\frac{\pi}{2} - a) = - \cos(\frac{\pi}{2} + a) = \sin(a)$\\		 
		\end{tabular}
		
		
		% \vfill\null
		% \columnbreak

		\subsubsection{Additionstheoreme}
		
		$\sin(a \pm b) = \sin(a) \cdot \cos(b) \pm \cos(a) \cdot \sin(b)$ \\
		$\cos(a \pm b) = \cos(a) \cdot \cos(b) \mp \sin(a) \cdot \sin(b) $ \\
		$\tan(a \pm b) = \frac{\tan(a) \pm \tan(b)}{1 \mp \tan(a) \cdot \tan(b)}$
		
	
		\subsubsection{Summen und Differenzen}
		
		$\sin(a) + \sin(b) = 2 \cdot \sin \big( \frac{a+b}{2} \big) \cdot \cos \big( \frac{a-b}{2} \big)$ \\
		$\sin(a) - \sin(b) = 2 \cdot \sin \big( \frac{a-b}{2} \big) \cdot \cos \big( \frac{a+b}{2} \big)$ \\
		$\cos(a) + \cos(b) = 2 \cdot \cos \big( \frac{a+b}{2} \big) \cdot \cos \big( \frac{a-b}{2} \big)$ \\
		$\cos(a) - \cos(b) = -2 \cdot \sin \big( \frac{a+b}{2} \big) \cdot \sin \big( \frac{a-b}{2} \big)$ \\
		$\tan(a) \pm \tan(b) = \frac{\sin(a \pm b)}{\cos(a) \cdot \cos(b)}$ 
		
			
		
		\subsubsection{Produkte}
		$\sin(a) \cdot \sin(b) = \frac{1}{2} \big( \cos(a-b) - \cos(a+b) \big) $ \\
		$\cos(a) \cdot \cos(b) = \frac{1}{2} \big( \cos(a-b) + \cos(a+b) \big) $ \\	
		$\sin(a) \cdot \cos(b) = \frac{1}{2} \big( \sin(a-b) + \sin(a+b) \big) $ 
			
			
		\subsubsection{Winkelvielfache und Halbwinkel}
		$\sin(2a) = 2 \, \sin(a) \cdot \cos(a)$ \\
		$\sin(3a) = 3 \, \sin(a) -4 \, \sin^3(a)$ \\ 
		$\sin(4a) = 8 \, \cos^3(a) \cdot \sin(a) - 4 \, \cos(a) \cdot \sin(a)$ \\
		\vspace{0.2cm}
		$\cos(2a) = \cos^2(a) - \sin^2(a)$ \\
		$\cos(3a) = 4 \, \cos^3(a) -3 \, \cos(a)$ \\ 
		$\cos(4a) = 8 \, \cos^4(a) - 8 \, \cos^2(a) + 1$ \\
		\vspace{0.2cm}
		$\sin \big( \frac{a}{2} \big) = \sqrt{\frac{1}{2}(1-\cos(a))} $ \qquad $\cos \big( \frac{a}{2} \big) = \sqrt{\frac{1}{2}(1+\cos(a))} $ 
		
		
		\subsubsection{Potenzen}
		$\sin^2(a) = \frac{1}{2}(1-\cos(2a))$ \\	
		$\sin^3(a) = \frac{1}{4}(3\, \sin(a) - \sin(3a))$ \\	
		$\sin^4(a) = \frac{1}{8}(\cos(4a) -4 \, \cos(2a) +3)$ \\	
		\vspace{0.2cm}
		$\cos^2(a) = \frac{1}{2}(1+\cos(2a))$ \\	
		$\cos^3(a) = \frac{1}{4}(\cos(3a) + 3 \, \cos(a)$ \\	
		$\cos^4(a) = \frac{1}{8}(\cos(4a) +4 \, \cos(2a) +3)$ 